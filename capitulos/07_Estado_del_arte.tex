\setcounter{chapter}{7}
\setcounter{section}{0}
\setcounter{subsection}{0}
\chapter{Estado del arte}

\section{Class Craft \cite{ClassCraft}}

Un ejemplo de una plataforma de docencia gamificada es Class Craft. En este caso, el profesor llega a los alumnos mediante una narrativa diferente pero con dinámicas y mecánicas muy parecidas. En el caso de esta plataforma podemos crear grupos de alumnos a modo de equipos, misiones, sistema de puntuaciones, sistema de experiencia. Como punto extra, la plataforma permite tener conversaciones con los padres. Además de estar considerado como uno de los mejores sistemas gamificados, es una herramienta online completamente gratuita.\\

Una de las principales ventajas es que nos ofrece la posibilidad de vivir su experiencia a través de diferentes formatos; aplicación web, aplicación móvil.

\section{Zoombiología \cite{Zoombiologia}}

Otro ejemplo que ofrece una narrativa completamente diferente es Zoombiologia. Aplicada al currículum de asignaturas de Biología y Geología de tercero de Educación Secundaria Obligatoria nos ofrece una web de docencia con fuerte narrativa Zombie pero, al ser un proyecto relativamente reciente, carece de muchas de las mecánicas y dinámicas de juego. Pero es un buen ejemplo de cómo gamificar una asignatura.

