\setcounter{chapter}{4}
\setcounter{section}{0}
\setcounter{subsection}{0}
\chapter{Plan de entregas}

\section{Breve descripción del alcance del sistema}
El desarrollo del proyecto \textbf{SinTime} consiste en la implementación de una aplicación web cuya implantación se realizará en el marco universitario con el fin de gamificar la experiencia del alumnado mediante la inmersión de ellos en un universo ficticio.\\

Los objetivos más importantes que debe cumplir la aplicación serán:

\begin{itemize}
	\item Debe permitir la administración de todos los usuarios con independencia de su rol.
	\item Debe facilitar al docente la creación de todo tipo de retos para el alumnado.
	\item Debe facilitar al docente la información de los movimientos de un alumno dentro de la app.
	\item Debe permitir al alumno ser evaluado por el docente.
	\item Debe permitir al alumno ser evaluado por el sistema.
	\item El alumno debe poder obtener su calificación.
\end{itemize}

\section{Listado inicial de Historias de Usuario}

A continuación se muestran las Historias de Usuario obtenidas durante las reuniones de planificación y entregas del producto realizadas entre el cliente y el equipo de desarrollo. La lista se divide en 4 partes: un identificador, una descripción de la historia, una estimación en días ideales y una prioridad. La prioridad se medirá por el cliente en un rango de 0 a 100, siendo 100 la prioridad más alta.\\

\begin{table}[h]
	\centering
	\begin{tabular}{| p{2.3cm} | p{5.1cm} | p{2cm} | p{1.6cm} |}
		\rowcolor[HTML]{329A9D} 
		{\color[HTML]{FFFFFF} \textbf{Identificador}} & {\color[HTML]{FFFFFF} \textbf{Historias de Usuario}} & {\color[HTML]{FFFFFF} \textbf{Estimación}} & {\color[HTML]{FFFFFF} \textbf{Prioridad}} \\ \hline
	\end{tabular}
\end{table}

\begin{table}[h]
	\centering
	\begin{tabular}{| p{2.3cm} | p{5.1cm} | p{2cm} | p{1.6cm} |}
		\hline 
		HU.1 & Plataformas y tecnologías usadas & 4 & 100 \\ \hline 
	\end{tabular}
\end{table}    

\begin{table}[h]
	\centering
	\begin{tabular}{| p{2.3cm} | p{5.1cm} | p{2cm} | p{1.6cm} |}
		\hline 
		HU.2 & Como administrador quiero preparar el
		entorno y las herramientas de uso & 11 & 100 \\ \hline 
	\end{tabular}
\end{table}   

\begin{table}[h]
	\centering
	\begin{tabular}{| p{2.3cm} | p{5.1cm} | p{2cm} | p{1.6cm} |}
		\hline 
		HU.3 & Creación del esqueleto de la aplicación Web & 15 & 100 \\ \hline 
	\end{tabular}
\end{table}   

\begin{table}[h]
	\centering
	\begin{tabular}{| p{2.3cm} | p{5.1cm} | p{2cm} | p{1.6cm} |}
		\hline 
		HU.4 & Como administrador quiero dar de alta a otros usuarios & 10 & 75 \\ \hline 
	\end{tabular}
\end{table}   


\begin{table}[h]
	\centering
	\begin{tabular}{| p{2.3cm} | p{5.1cm} | p{2cm} | p{1.6cm} |}
		\hline 
		HU.5 & Permitir al Maestro Guidoogway gestionar clanes entre los usuarios padawans & 4 & 50 \\ \hline 
	\end{tabular}
\end{table}   

\begin{table}[h]
	\centering
	\begin{tabular}{| p{2.3cm} | p{5.1cm} | p{2cm} | p{1.6cm} |}
		\hline 
		HU.6 & Permitir al Maestro Guidoogway gestionar retos & 64 & 100 \\ \hline 
	\end{tabular}
\end{table}      

\begin{table}[h]
	\centering
	\begin{tabular}{| p{2.3cm} | p{5.1cm} | p{2cm} | p{1.6cm} |}
		\hline 
		HU.8 & Permitir al Maestro Guidoogway gestionar las vacaciones de los usuarios Padawans & 8 & 50 \\ \hline 
	\end{tabular}
\end{table}   


\begin{table}[h]
	\centering
	\begin{tabular}{| p{2.3cm} | p{5.1cm} | p{2cm} | p{1.6cm} |}
		\hline 
		HU.9 & Permitir al Maestro Guidoogway gestionar los rangos de los usuarios Padawans & 8 & 50 \\ \hline 
	\end{tabular}
\end{table}   

\begin{table}[h]
	\centering
	\begin{tabular}{| p{2.3cm} | p{5.1cm} | p{2cm} | p{1.6cm} |}
		\hline 
		HU.10 & Permitir al Maestro Guidoogway gestionar las misiones de desactivación de minas & 12 & 75 \\ \hline 
	\end{tabular}
\end{table}   


\begin{table}[h]
	\centering
	\begin{tabular}{| p{2.3cm} | p{5.1cm} | p{2cm} | p{1.6cm} |}
		\hline 
		HU.11 & Permitir al Maestro Guidoogway gestionar los badges & 4 & 50 \\ \hline 
	\end{tabular}
\end{table}   

\begin{table}[h]
	\centering
	\begin{tabular}{| p{2.3cm} | p{5.1cm} | p{2cm} | p{1.6cm} |}
		\hline 
		HU.12 & Preparar la sección de usuarios Padawans & 20 & 100 \\ \hline 
	\end{tabular}
\end{table}   

\begin{table}[h]
	\centering
	\begin{tabular}{| p{2.3cm} | p{5.1cm} | p{2cm} | p{1.6cm} |}
		\hline 
		HU.13 & Control de accesos & 4 & 75 \\ \hline 
	\end{tabular}
\end{table}   

\begin{table}[h]
	\centering
	\begin{tabular}{| p{2.3cm} | p{5.1cm} | p{2cm} | p{1.6cm} |}
		\hline 
		HU.14 & Permitir el registro a la plataforma & 8 & 50 \\ \hline 
	\end{tabular}
\end{table}   

\begin{table}[h]
	\centering
	\begin{tabular}{| p{2.3cm} | p{5.1cm} | p{2cm} | p{1.6cm} |}
		\hline 
		HU.15 & Permitir al usuario Padawan recuperar su contraseña si la ha olvidado & 4 & 50 \\ \hline 
	\end{tabular}
\end{table}   

\begin{table}[h]
	\centering
	\begin{tabular}{| p{2.3cm} | p{5.1cm} | p{2cm} | p{1.6cm} |}
		\hline 
		HU.16 & Permitir al usuario Padawan modificar sus datos de perfil & 4 & 50 \\ \hline 
	\end{tabular}
\end{table}   

\begin{table}[h]
	\centering
	\begin{tabular}{| p{2.3cm} | p{5.1cm} | p{2cm} | p{1.6cm} |}
		\hline 
		HU.17 & Permitir al usuario Padawan ver su historial de cargos e ingresos de su cuenta de TdV & 4 & 20 \\ \hline 
	\end{tabular}
\end{table}   

\begin{table}[h]
	\centering
	\begin{tabular}{| p{2.3cm} | p{5.1cm} | p{2cm} | p{1.6cm} |}
		\hline 
		HU.18 & Permitir al usuario Padawan realizar retos de tipo Seguimiento del Maestro Jedi & 40 & 75 \\ \hline 
	\end{tabular}
\end{table}   

\begin{table}[h]
	\centering
	\begin{tabular}{| p{2.3cm} | p{5.1cm} | p{2cm} | p{1.6cm} |}
		\hline 
		HU.19 & Permitir al usuario Padawan realizar retos de tipo Pruebas de Nivel & 20 & 75 \\ \hline 
	\end{tabular}
\end{table}   

\begin{table}[h]
	\centering
	\begin{tabular}{| p{2.3cm} | p{5.1cm} | p{2cm} | p{1.6cm} |}
		\hline 
		HU.20 & Permitir al usuario Padawan realizar retos de tipo Dominio de la Fuerzas & 20 & 75 \\ \hline 
	\end{tabular}
\end{table}   

\begin{table}[h]
	\centering
	\begin{tabular}{| p{2.3cm} | p{5.1cm} | p{2cm} | p{1.6cm} |}
		\hline 
		HU.21 & Permitir al usuario Padawan realizar retos de tipo Audiencia ante el Senado Galáctico & 20 & 75 \\ \hline 
	\end{tabular}
\end{table}   

\FloatBarrier        


\section{Cálculo de la velocidad del equipo}

El equipo inicial está formado por un programador con una dedicación al proyecto del 50\%. La duración de cada Sprint será, como ya se ha comentado, de 3 semanas de las cuales dos semanas se destinarán a desarrollo y una a testing.\\

La estimación está basada en \textit{días ideales}. Un día ideal se compone de 4 horas de trabajo por lo que el resultado en horas de cada  sprint será:\\

\[
1 \textrm{Sprint} = 3 \textrm{Semanas} \quad \textrm{y} \quad 1 \textrm{Semana} = 5 \textrm{días ideales}
\]

\[
1 \textrm{día ideal} = 4 \textrm{horas reales} \quad \textrm{entonces} \quad 1 \textrm{Spint} = 60 \textrm{horas reales}
\]

\section{Descripción de las entregas}

\textbf{Esfuerzo total del proyecto} = 612 PH
\textbf{Duración del proyecto} = 11 meses y 1 semana
\textbf{Velocidad del equipo} = 40 a 41 PH

En base a las estimaciones de velocidad del equipo y al esfuerzo necesario para el desarrollo se ha decidido realizar 15 entregas. La semana laboral dedicada al proyecto se dividirá en 5 días (Lunes a Viernes) con una duración de 4 horas diarias que suman un total de 20 horas semanales. El desarrollo del proyecto comenzará el día \textbf{1 de Junio de 2017}.\\


El plan de entregas es el siguiente:

\begin{table}[h]
	\centering
	\begin{tabular}{| p{2cm} | p{9.3cm} |}
		\rowcolor[HTML]{329A9D} 
		{\color[HTML]{FFFFFF} \textbf{Entrega}} & {\color[HTML]{FFFFFF} \textbf{Fecha de entrega}} \\    
		Sprint 0 & 21 Junio 2017 \\ \hline
		Sprint 1 & 12 Julio 2017 \\ \hline
		Sprint 2 & 2 Agosto 2017 \\ \hline
		Sprint 3 & 23 Agosto 2017 \\ \hline
		Sprint 4 & 13 Septiembre 2017 \\ \hline
		Sprint 5 & 27 Septiembre 2017 \\ \hline
		Sprint 6 & 18 Octubre 2017 \\ \hline
		Sprint 7 & 8 Noviembre 2017 \\ \hline
		Sprint 8 & 29 Noviembre 2017 \\ \hline
		Sprint 9 & 20 Diciembre 2017 \\ \hline
		Sprint 10 & 10 Enero 2018 \\ \hline
		Sprint 11 & 31 Enero 2018 \\ \hline
		Sprint 12 & 21 Febrero 2018 \\ \hline
		Sprint 13 & 14 Marzo 2018 \\ \hline
		Sprint 14 & 4 Abril 2018 \\ \hline
	\end{tabular}
\end{table}

\newpage

\section{Lista Inicial del Producto}

La lista del producto, con las historias que se usaran en el inicio del desarrollo es la siguiente:\\

\begin{table}[h]
	\centering
	\begin{tabular}{| p{2.3cm} | p{5.1cm} | p{2cm} | p{1.6cm} |}
		\rowcolor[HTML]{329A9D} 
		{\color[HTML]{FFFFFF} \textbf{Identificador}} & {\color[HTML]{FFFFFF} \textbf{Historias de Usuario}} & {\color[HTML]{FFFFFF} \textbf{Estimación}} & {\color[HTML]{FFFFFF} \textbf{Prioridad}} \\ \hline
	\end{tabular}
\end{table}

\begin{table}[h]
	\centering
	\begin{tabular}{| p{2.3cm} | p{5.1cm} | p{2cm} | p{1.6cm} |}
		\hline 
		HU.1 & Plataformas y tecnologías usadas & 4 & 100 \\ \hline 
	\end{tabular}
\end{table}    

\begin{table}[h]
	\centering
	\begin{tabular}{| p{2.3cm} | p{5.1cm} | p{2cm} | p{1.6cm} |}
		\hline 
		HU.2 & Como administrador quiero preparar el
		entorno y las herramientas de uso & 11 & 100 \\ \hline 
	\end{tabular}
\end{table}   

\begin{table}[h]
	\centering
	\begin{tabular}{| p{2.3cm} | p{5.1cm} | p{2cm} | p{1.6cm} |}
		\hline 
		HU.3 & Creación del esqueleto de la aplicación Web & 15 & 100 \\ \hline 
	\end{tabular}
\end{table}   

\begin{table}[h]
	\centering
	\begin{tabular}{| p{2.3cm} | p{5.1cm} | p{2cm} | p{1.6cm} |}
		\hline 
		HU.4 & Como administrador quiero dar de alta a otros usuarios & 10 & 75 \\ \hline 
	\end{tabular}
\end{table}   


\begin{table}[h]
	\centering
	\begin{tabular}{| p{2.3cm} | p{5.1cm} | p{2cm} | p{1.6cm} |}
		\hline 
		HU.5 & Permitir al Maestro Guidoogway gestionar clanes entre los usuarios padawans & 4 & 50 \\ \hline 
	\end{tabular}
\end{table}   

\begin{table}[h]
	\centering
	\begin{tabular}{| p{2.3cm} | p{5.1cm} | p{2cm} | p{1.6cm} |}
		\hline 
		HU.6 & Permitir al Maestro Guidoogway gestionar retos & 64 & 100 \\ \hline 
	\end{tabular}
\end{table}      

\begin{table}[h]
	\centering
	\begin{tabular}{| p{2.3cm} | p{5.1cm} | p{2cm} | p{1.6cm} |}
		\hline 
		HU.8 & Permitir al Maestro Guidoogway gestionar las vacaciones de los usuarios Padawans & 8 & 50 \\ \hline 
	\end{tabular}
\end{table}   


\begin{table}[h]
	\centering
	\begin{tabular}{| p{2.3cm} | p{5.1cm} | p{2cm} | p{1.6cm} |}
		\hline 
		HU.9 & Permitir al Maestro Guidoogway gestionar los rangos de los usuarios Padawans & 8 & 50 \\ \hline 
	\end{tabular}
\end{table}   

\begin{table}[h]
	\centering
	\begin{tabular}{| p{2.3cm} | p{5.1cm} | p{2cm} | p{1.6cm} |}
		\hline 
		HU.10 & Permitir al Maestro Guidoogway gestionar las misiones de desactivación de minas & 12 & 75 \\ \hline 
	\end{tabular}
\end{table}   


\begin{table}[h]
	\centering
	\begin{tabular}{| p{2.3cm} | p{5.1cm} | p{2cm} | p{1.6cm} |}
		\hline 
		HU.11 & Permitir al Maestro Guidoogway gestionar los badges & 4 & 50 \\ \hline 
	\end{tabular}
\end{table}   

\begin{table}[h]
	\centering
	\begin{tabular}{| p{2.3cm} | p{5.1cm} | p{2cm} | p{1.6cm} |}
		\hline 
		HU.12 & Preparar la sección de usuarios Padawans & 20 & 100 \\ \hline 
	\end{tabular}
\end{table}   

\begin{table}[h]
	\centering
	\begin{tabular}{| p{2.3cm} | p{5.1cm} | p{2cm} | p{1.6cm} |}
		\hline 
		HU.13 & Control de accesos & 4 & 75 \\ \hline 
	\end{tabular}
\end{table}   

\begin{table}[h]
	\centering
	\begin{tabular}{| p{2.3cm} | p{5.1cm} | p{2cm} | p{1.6cm} |}
		\hline 
		HU.14 & Permitir el registro a la plataforma & 8 & 50 \\ \hline 
	\end{tabular}
\end{table}   

\begin{table}[h]
	\centering
	\begin{tabular}{| p{2.3cm} | p{5.1cm} | p{2cm} | p{1.6cm} |}
		\hline 
		HU.15 & Permitir al usuario Padawan recuperar su contraseña si la ha olvidado & 4 & 50 \\ \hline 
	\end{tabular}
\end{table}   

\begin{table}[h]
	\centering
	\begin{tabular}{| p{2.3cm} | p{5.1cm} | p{2cm} | p{1.6cm} |}
		\hline 
		HU.16 & Permitir al usuario Padawan modificar sus datos de perfil & 4 & 50 \\ \hline 
	\end{tabular}
\end{table}   

\begin{table}[h]
	\centering
	\begin{tabular}{| p{2.3cm} | p{5.1cm} | p{2cm} | p{1.6cm} |}
		\hline 
		HU.17 & Permitir al usuario Padawan ver su historial de cargos e ingresos de su cuenta de TdV & 4 & 20 \\ \hline 
	\end{tabular}
\end{table}   

\begin{table}[h]
	\centering
	\begin{tabular}{| p{2.3cm} | p{5.1cm} | p{2cm} | p{1.6cm} |}
		\hline 
		HU.18 & Permitir al usuario Padawan realizar retos de tipo Seguimiento del Maestro Jedi & 40 & 75 \\ \hline 
	\end{tabular}
\end{table}   

\begin{table}[h]
	\centering
	\begin{tabular}{| p{2.3cm} | p{5.1cm} | p{2cm} | p{1.6cm} |}
		\hline 
		HU.19 & Permitir al usuario Padawan realizar retos de tipo Pruebas de Nivel & 20 & 75 \\ \hline 
	\end{tabular}
\end{table}   

\begin{table}[h]
	\centering
	\begin{tabular}{| p{2.3cm} | p{5.1cm} | p{2cm} | p{1.6cm} |}
		\hline 
		HU.20 & Permitir al usuario Padawan realizar retos de tipo Dominio de la Fuerzas & 20 & 75 \\ \hline 
	\end{tabular}
\end{table}   

\begin{table}[h]
	\centering
	\begin{tabular}{| p{2.3cm} | p{5.1cm} | p{2cm} | p{1.6cm} |}
		\hline 
		HU.21 & Permitir al usuario Padawan realizar retos de tipo Audiencia ante el Senado Galáctico & 20 & 75 \\ \hline 
	\end{tabular}
\end{table}   

\FloatBarrier        


\section{Plan de entrega 0}

\subsection{Objetivos de la entrega}

El objetivo principal de esta entrega se basa en la búsqueda y selección de las tecnologías y herramientas usadas para el desarrollo de este proyecto. Además se hará una primera implementación para crear un apartado básico de gestión de usuarios.

\subsection{Listado de HU a desarrollar}

\begin{table}[h]
	\centering
	\begin{tabular}{| p{2.3cm} | p{6.7cm} | p{2cm} |}
		\rowcolor[HTML]{329A9D} 
		{\color[HTML]{FFFFFF} \textbf{Identificador}} & {\color[HTML]{FFFFFF} \textbf{Historias de Usuario}} & {\color[HTML]{FFFFFF} \textbf{Estimación}}  \\ \hline
		HU.1 & Plataformas y tecnologías usadas & 4 \\ \hline   
		HU.2 & Como administrador quiero preparar el entorno y las herramientas de uso & 11 \\ \hline   
		HU.3 & Creación del esqueleto de la aplicación Web & 15 \\ \hline
		HU.4 & Como administrador quiero dar de alta a otros usuarios & 10 \\ \hline
	\end{tabular}
\end{table}

\newpage

\subsection{Descomposición en tareas de desarrollo}

A continuación incluimos un desglose de las HU en tareas de desarrollo junto con la estimación realizada de su duración.\\

\begin{table}[h]
	\centering
	\begin{tabular}{| p{2.3cm} | p{6.7cm} | p{2cm} |}
		\rowcolor[HTML]{329A9D} 
		{\color[HTML]{FFFFFF} \textbf{HU.1}} & {\color[HTML]{FFFFFF} \textbf{Plataformas y tecnologías usadas}} & {\color[HTML]{FFFFFF} \textbf{4}}  \\ \hline
		Tarea 1.1 & Diagrama de clases & 4 \\ \hline
	\end{tabular}
\end{table}

\begin{table}[h]
	\centering
	\begin{tabular}{| p{2.3cm} | p{6.7cm} | p{2cm} |}
		\rowcolor[HTML]{329A9D} 
		{\color[HTML]{FFFFFF} \textbf{HU.2}} & {\color[HTML]{FFFFFF} \textbf{Como administrador quiero preparar el entorno y las herramientas de uso}} & {\color[HTML]{FFFFFF} \textbf{11}}  \\ \hline
		Tarea 2.1 & Configuración de la herramienta y elementos necesarios para comenzar el desarrollo & 11 \\ \hline
	\end{tabular}
\end{table}

\begin{table}[h]
	\centering
	\begin{tabular}{| p{2.3cm} | p{6.7cm} | p{2cm} |}
		\rowcolor[HTML]{329A9D} 
		{\color[HTML]{FFFFFF} \textbf{HU.3}} & {\color[HTML]{FFFFFF} \textbf{Creación del esqueleto de la aplicación Web}} & {\color[HTML]{FFFFFF} \textbf{15}}  \\ \hline
		Tarea 3.1 & Desarrollo de un prototipo básico real & 15 \\ \hline
	\end{tabular}
\end{table}

\begin{table}[h]
	\centering
	\begin{tabular}{| p{2.3cm} | p{6.7cm} | p{2cm} |}
		\rowcolor[HTML]{329A9D} 
		{\color[HTML]{FFFFFF} \textbf{HU.4}} & {\color[HTML]{FFFFFF} \textbf{Como administrador quiero dar de alta a otros usuarios}} & {\color[HTML]{FFFFFF} \textbf{10}}  \\ \hline
		Tarea 4.1 & Creación en la plataforma Web del usuario tipo Maestro Guidoogway & 3 \\ \hline
		Tarea 4.2 & Creación en la plataforma Web del usuario tipo Padawan & 3 \\ \hline
		Tarea 4.3 & Creación en la plataforma Web del panel de gestión de usuarios desde el BackOffice & 4 \\ \hline
	\end{tabular}
\end{table}

\newpage
\subsection{Carga prevista en los desarrolladores}

A continuación mostramos la carga de trabajo prevista para cada uno de los desarrolladores en relación a las tareas definidas anteriormente:

\begin{table}[h]
	\centering
	\begin{tabular}{| p{3cm} | p{2cm} | p{2cm} | p{2cm} | p{2cm} |}
		\rowcolor[HTML]{329A9D} 
		{\color[HTML]{FFFFFF} \textbf{Desarrollador}} & {\color[HTML]{FFFFFF} \textbf{Velocidad inicial (días ideales)}} & {\color[HTML]{FFFFFF} \textbf{Dedicación (\% del tiempo)}} & {\color[HTML]{FFFFFF} \textbf{Carga de trabajo (días ideales)}} & {\color[HTML]{FFFFFF} \textbf{\#Tareas aceptadas}}  \\ \hline
		Álvaro Fernández-Alonso Araluce & 15 & 50\% & 15 & 6 \\ \hline
	\end{tabular}
\end{table}

\newpage

\subsection{Planificación temporal de las tareas}

\begin{table}[h]
	\centering
	\begin{tabular}{| p{2cm} | p{2cm} | p{2cm} | p{2cm} | p{2cm} | p{2cm} |}
		\rowcolor[HTML]{329A9D} 
		 {\color[HTML]{FFFFFF} \textbf{Semana 1}} & {\color[HTML]{FFFFFF} \textbf{Día 1)}} & {\color[HTML]{FFFFFF} \textbf{Día 2}} & {\color[HTML]{FFFFFF} \textbf{Día 3}} & {\color[HTML]{FFFFFF} \textbf{Día 4}}  & {\color[HTML]{FFFFFF} \textbf{Día 5}} \\ \hline
		Álvaro Fernández-Alonso Araluce & 1.1 & 2.1 & 2.1 & 2.1 3.1 & 3.1 \\ \hline
	\end{tabular}
\end{table}

\begin{table}[h]
	\centering
	\begin{tabular}{| p{2cm} | p{2cm} | p{2cm} | p{2cm} | p{2cm} | p{2cm} |}
		\rowcolor[HTML]{329A9D} 
		{\color[HTML]{FFFFFF} \textbf{Semana 2}} & {\color[HTML]{FFFFFF} \textbf{Día 1)}} & {\color[HTML]{FFFFFF} \textbf{Día 2}} & {\color[HTML]{FFFFFF} \textbf{Día 3}} & {\color[HTML]{FFFFFF} \textbf{Día 4}}  & {\color[HTML]{FFFFFF} \textbf{Día 5}} \\ \hline
		Álvaro Fernández-Alonso Araluce & 3.1 & 3.1 & 3.1 4.1 & 4.1 4.2 & 4.3 \\ \hline
	\end{tabular}
\end{table}

\begin{table}[h]
	\centering
	\begin{tabular}{| p{2cm} | p{2cm} | p{2cm} | p{2cm} | p{2cm} | p{2cm} |}
		\rowcolor[HTML]{329A9D} 
		{\color[HTML]{FFFFFF} \textbf{Semana 3}} & {\color[HTML]{FFFFFF} \textbf{Día 1)}} & {\color[HTML]{FFFFFF} \textbf{Día 2}} & {\color[HTML]{FFFFFF} \textbf{Día 3}} & {\color[HTML]{FFFFFF} \textbf{Día 4}}  & {\color[HTML]{FFFFFF} \textbf{Día 5}} \\ \hline
		Álvaro Fernández-Alonso Araluce & Testing & Testing & Testing & Testing & Testing \\ \hline
	\end{tabular}
\end{table}


\newpage
\section{Plan de entrega 1}

\subsection{Objetivos de la entrega}

El objetivo principal de esta entrega consiste en seguir desarrollando funcionalidades para los usuarios padawans y empezar a desarrollar el apartado de gestión de retos desde el BackOffice.

\subsection{Listado de HU a desarrollar}

\begin{table}[h]
	\centering
	\begin{tabular}{| p{2.3cm} | p{6.7cm} | p{2cm} |}
		\rowcolor[HTML]{329A9D} 
		{\color[HTML]{FFFFFF} \textbf{Identificador}} & {\color[HTML]{FFFFFF} \textbf{Historias de Usuario}} & {\color[HTML]{FFFFFF} \textbf{Estimación}}  \\ \hline
		HU.5 & Permitir al Maestro Guidoogway gestionar clanes entre los usuarios padawans & 4 \\ \hline
		HU.6 & Permitir al Maestro Guidoogway gestionar retos & 64 \\ \hline
	\end{tabular}
\end{table}

\subsection{Descomposición en tareas de desarrollo}

A continuación incluimos un desglose de las HU en tareas de desarrollo junto con la estimación realizada de su duración.\\

\begin{table}[h]
	\centering
	\begin{tabular}{| p{2.3cm} | p{6.7cm} | p{2cm} |}
		\rowcolor[HTML]{329A9D} 
		{\color[HTML]{FFFFFF} \textbf{HU.5}} & {\color[HTML]{FFFFFF} \textbf{Permitir al Maestro Guidoogway gestionar clanes entre los usuarios padawans}} & {\color[HTML]{FFFFFF} \textbf{4}}  \\ \hline
		Tarea 5.1 & Permitir al Maestro Guidoogway agrupar usuarios Padawans en un clan & 3 \\ \hline
		Tarea 5.2 & Permitir al Maestro Guidoogway asignar un escudo a los clanes & 1 \\ \hline
	\end{tabular}
\end{table}

\newpage

\begin{table}[h]
	\centering
	\begin{tabular}{| p{2.3cm} | p{6.7cm} | p{2cm} |}
		\rowcolor[HTML]{329A9D} 
		{\color[HTML]{FFFFFF} \textbf{HU.6}} & {\color[HTML]{FFFFFF} \textbf{Permitir al Maestro Guidoogway gestionar retos}} & {\color[HTML]{FFFFFF} \textbf{64}}  \\ \hline
		Tarea 6.1 & Permitir al Maestro Guidoogway gestionar retos de Dominio de la Fuerza (Cuestionarios) & 8 \\ \hline
		Tarea 6.2 & Permitir al Maestro Guidoogway gestionar retos de Comida (Entregas individuales) & 4 \\ \hline
		Tarea 6.3 & Permitir al Maestro Guidoogway gestionar retos de Comida (Entregas de clanes) & 4 \\ \hline
		Tarea 6.4 & Permitir al Maestro Guidoogway gestionar retos de Agua (Entregas individuales) & 4 \\ \hline
		Tarea 6.5 & Permitir al Maestro Guidoogway gestionar retos de Agua (Entregas de clanes) & 4 \\ \hline
		Tarea 6.6 & Permitir al Maestro Guidoogway gestionar retos deportivos (Establecer parámetros deportivos a superar durante la semana)& 8 \\ \hline
		Tarea 6.7 & Permitir al Maestro Guidoogway gestionar retos de Felicidad & 8 \\ \hline
		Tarea 6.8 & Permitir al Maestro Guidoogway gestionar pruebas de nivel & 8 \\ \hline
		Tarea 6.9 & Permitir al Maestro Guidoogway gestionar pruebas de audiencia & 8 \\ \hline
		Tarea 6.10 & Permitir al Maestro Guidoogway asignar el coste en TdV de cada reto & 4 \\ \hline
		Tarea 6.11 & Permitir al Maestro Guidoogway asignar el beneficio obtenido por el Padawan en función de diferentes Badges & 4 \\ \hline
	\end{tabular}
\end{table}

\newpage
\subsection{Carga prevista en los desarrolladores}

A continuación mostramos la carga de trabajo prevista para cada uno de los desarrolladores en relación a las tareas definidas anteriormente:

\begin{table}[h]
	\centering
	\begin{tabular}{| p{3cm} | p{2cm} | p{2cm} | p{2cm} | p{2cm} |}
		\rowcolor[HTML]{329A9D} 
		{\color[HTML]{FFFFFF} \textbf{Desarrollador}} & {\color[HTML]{FFFFFF} \textbf{Velocidad inicial (días ideales)}} & {\color[HTML]{FFFFFF} \textbf{Dedicación (\% del tiempo)}} & {\color[HTML]{FFFFFF} \textbf{Carga de trabajo (días ideales)}} & {\color[HTML]{FFFFFF} \textbf{\#Tareas aceptadas}}  \\ \hline
		Álvaro Fernández-Alonso Araluce & 15 & 50\% & 15 & 9 \\ \hline
	\end{tabular}
\end{table}

\subsection{Planificación temporal de las tareas}

\begin{table}[h]
	\centering
	\begin{tabular}{| p{2cm} | p{2cm} | p{2cm} | p{2cm} | p{2cm} | p{2cm} |}
		\rowcolor[HTML]{329A9D} 
		 {\color[HTML]{FFFFFF} \textbf{Semana 1}} & {\color[HTML]{FFFFFF} \textbf{Día 1}} & {\color[HTML]{FFFFFF} \textbf{Día 2}} & {\color[HTML]{FFFFFF} \textbf{Día 3}} & {\color[HTML]{FFFFFF} \textbf{Día 4}}  & {\color[HTML]{FFFFFF} \textbf{Día 5}} \\ \hline
		Álvaro Fernández-Alonso Araluce & 5.1 5.2 & 6.1 & 6.1 & 6.2 & 6.3 \\ \hline
	\end{tabular}
\end{table}

\begin{table}[h]
	\centering
	\begin{tabular}{| p{2cm} | p{2cm} | p{2cm} | p{2cm} | p{2cm} | p{2cm} |}
		\rowcolor[HTML]{329A9D} 
		{\color[HTML]{FFFFFF} \textbf{Semana 2}} & {\color[HTML]{FFFFFF} \textbf{Día 1}} & {\color[HTML]{FFFFFF} \textbf{Día 2}} & {\color[HTML]{FFFFFF} \textbf{Día 3}} & {\color[HTML]{FFFFFF} \textbf{Día 4}}  & {\color[HTML]{FFFFFF} \textbf{Día 5}} \\ \hline
		Álvaro Fernández-Alonso Araluce & 6.4 & 6.5 & 6.6 & 6.6 & 6.7 \\ \hline
	\end{tabular}
\end{table}

\begin{table}[h]
	\centering
	\begin{tabular}{| p{2cm} | p{2cm} | p{2cm} | p{2cm} | p{2cm} | p{2cm} |}
		\rowcolor[HTML]{329A9D} 
		{\color[HTML]{FFFFFF} \textbf{Semana 3}} & {\color[HTML]{FFFFFF} \textbf{Día 1}} & {\color[HTML]{FFFFFF} \textbf{Día 2}} & {\color[HTML]{FFFFFF} \textbf{Día 3}} & {\color[HTML]{FFFFFF} \textbf{Día 4}}  & {\color[HTML]{FFFFFF} \textbf{Día 5}} \\ \hline
		Álvaro Fernández-Alonso Araluce & 6.7 & Testing & Testing & Testing & Testing \\ \hline
	\end{tabular}
\end{table}


\newpage
\input{capitulos/plan_de_entregas/02}
\newpage
\section{Plan de entrega 3}

\subsection{Objetivos de la entrega}

El objetivo principal de esta entrega consiste en crear las primeras secciones para los usuarios Padawans y terminar de desarrollar las funcionalidades que permiten al usuario Maestro Guidoogway crear y editar retos y misiones desde el BackOffice.

\subsection{Listado de HU a desarrollar}

\begin{table}[h]
	\centering
	\begin{tabular}{| p{2.3cm} | p{6.7cm} | p{2cm} |}
		\rowcolor[HTML]{329A9D} 
		{\color[HTML]{FFFFFF} \textbf{Identificador}} & {\color[HTML]{FFFFFF} \textbf{Historias de Usuario}} & {\color[HTML]{FFFFFF} \textbf{Estimación}}  \\ \hline
		HU.10 & Permitir al Maestro Guidoogway gestionar las misiones de desactivación de minas & 12 \\ \hline
		HU.11 & Permitir al Maestro Guidoogway gestionar los badges & 4 \\ \hline
		HU.12 & Preparar la sección de usuarios Padawans & 20 \\ \hline
		HU.13 & Control de accesos & 4 \\ \hline
	\end{tabular}
\end{table}

\subsection{Descomposición en tareas de desarrollo}

A continuación incluimos un desglose de las HU en tareas de desarrollo junto con la estimación realizada de su duración.\\

\begin{table}[h]
	\centering
	\begin{tabular}{| p{2.3cm} | p{6.7cm} | p{2cm} |}
		\rowcolor[HTML]{329A9D} 
		{\color[HTML]{FFFFFF} \textbf{HU.11}} & {\color[HTML]{FFFFFF} \textbf{Permitir al Maestro Guidoogway gestionar los badges}} & {\color[HTML]{FFFFFF} \textbf{4}}  \\ \hline
		Tarea 11.1 & Permitir al Maestro Guidoogway gestionar los badges & 3 \\ \hline
		Tarea 11.2 & Permitir al Maestro Guidoogway asignar un logo a cada badge  & 1 \\ \hline
	\end{tabular}
\end{table}

\newpage

\begin{table}[t]
	\centering
	\begin{tabular}{| p{2.3cm} | p{6.7cm} | p{2cm} |}
		\rowcolor[HTML]{329A9D} 
		{\color[HTML]{FFFFFF} \textbf{HU.10}} & {\color[HTML]{FFFFFF} \textbf{Permitir al Maestro Guidoogway gestionar las misiones de desactivación de minas}} & {\color[HTML]{FFFFFF} \textbf{12}}  \\ \hline
		Tarea 10.1 & Permitir al Maestro Guidoogway gestionar misiones de minas& 4 \\ \hline
		Tarea 10.2 & Permitir al Maestro Guidoogway asignar un código secreto para desactivar una mina & 2 \\ \hline
		Tarea 10.3 & Permitir al Maestro Guidoogway añadir pistas secretas para que los usuarios Padawans consigan la clave secreta & 2 \\ \hline
		Tarea 10.4 & Permitir al Maestro Guidoogway añadir un beneficio en TdV para los usuarios Padawans que consigan desactivar la mina & 2 \\ \hline
		Tarea 10.5 & Permitir al Maestro Guidoogway establecer la fecha y la hora de la explosión de la mina & 2 \\ \hline
	\end{tabular}
\end{table}

\begin{table}[t]
	\centering
	\begin{tabular}{| p{2.3cm} | p{6.7cm} | p{2cm} |}
		\rowcolor[HTML]{329A9D} 
		{\color[HTML]{FFFFFF} \textbf{HU.12}} & {\color[HTML]{FFFFFF} \textbf{Preparar la sección de usuarios Padawans}} & {\color[HTML]{FFFFFF} \textbf{20}}  \\ \hline
		Tarea 12.1 & Creación en la plataforma Web del apartado para usuarios Padawan & 4 \\ \hline
		Tarea 12.2 & Colocación de las imágenes facilitadas por el cliente para cada sección & 4 \\ \hline
		Tarea 12.3 & Preparación de estilos, fondos y fuentes estipulados por el cliente & 8 \\ \hline
		Tarea 12.4 & Colocación del contador regresivo en cada sección & 4 \\ \hline
	\end{tabular}
\end{table}

\begin{table}[t]
	\centering
	\begin{tabular}{| p{2.3cm} | p{6.7cm} | p{2cm} |}
		\rowcolor[HTML]{329A9D} 
		{\color[HTML]{FFFFFF} \textbf{HU.13}} & {\color[HTML]{FFFFFF} \textbf{Control de accesos}} & {\color[HTML]{FFFFFF} \textbf{4}}  \\ \hline
		Tarea 13.1 & Restringir el acceso a la sección de gestión solo para que puedan acceder usuarios Administradores  & 2 \\ \hline
		Tarea 13.2 & Restringir el acceso a la sección de usuario solo para que puedan acceder usuarios Padawans  & 2 \\ \hline
	\end{tabular}
\end{table}

\newpage
\FloatBarrier

\subsection{Carga prevista en los desarrolladores}

A continuación mostramos la carga de trabajo prevista para cada uno de los desarrolladores en relación a las tareas definidas anteriormente:

\begin{table}[h]
	\centering
	\begin{tabular}{| p{3cm} | p{2cm} | p{2cm} | p{2cm} | p{2cm} |}
		\rowcolor[HTML]{329A9D} 
		{\color[HTML]{FFFFFF} \textbf{Desarrollador}} & {\color[HTML]{FFFFFF} \textbf{Velocidad inicial (días ideales)}} & {\color[HTML]{FFFFFF} \textbf{Dedicación (\% del tiempo)}} & {\color[HTML]{FFFFFF} \textbf{Carga de trabajo (días ideales)}} & {\color[HTML]{FFFFFF} \textbf{\#Tareas aceptadas}}  \\ \hline
		Álvaro Fernández-Alonso Araluce & 15 & 50\% & 15 & 13 \\ \hline
	\end{tabular}
\end{table}

\subsection{Planificación temporal de las tareas}

\begin{table}[h]
	\centering
	\begin{tabular}{| p{2cm} | p{2cm} | p{2cm} | p{2cm} | p{2cm} | p{2cm} |}
		\rowcolor[HTML]{329A9D} 
		 {\color[HTML]{FFFFFF} \textbf{Semana 1}} & {\color[HTML]{FFFFFF} \textbf{Día 1}} & {\color[HTML]{FFFFFF} \textbf{Día 2}} & {\color[HTML]{FFFFFF} \textbf{Día 3}} & {\color[HTML]{FFFFFF} \textbf{Día 4}}  & {\color[HTML]{FFFFFF} \textbf{Día 5}} \\ \hline
		Álvaro Fernández-Alonso Araluce & 10.1 & 10.2 10.3 & 10.4 10.5 & 11.1 11.2 & 12.1 \\ \hline
	\end{tabular}
\end{table}

\begin{table}[h]
	\centering
	\begin{tabular}{| p{2cm} | p{2cm} | p{2cm} | p{2cm} | p{2cm} | p{2cm} |}
		\rowcolor[HTML]{329A9D} 
		{\color[HTML]{FFFFFF} \textbf{Semana 2}} & {\color[HTML]{FFFFFF} \textbf{Día 1}} & {\color[HTML]{FFFFFF} \textbf{Día 2}} & {\color[HTML]{FFFFFF} \textbf{Día 3}} & {\color[HTML]{FFFFFF} \textbf{Día 4}}  & {\color[HTML]{FFFFFF} \textbf{Día 5}} \\ \hline
		Álvaro Fernández-Alonso Araluce & 12.2 & 12.3 & 12.3 & 12.4 & 13.1 13.2 \\ \hline
	\end{tabular}
\end{table}

\begin{table}[h]
	\centering
	\begin{tabular}{| p{2cm} | p{2cm} | p{2cm} | p{2cm} | p{2cm} | p{2cm} |}
		\rowcolor[HTML]{329A9D} 
		{\color[HTML]{FFFFFF} \textbf{Semana 3}} & {\color[HTML]{FFFFFF} \textbf{Día 1}} & {\color[HTML]{FFFFFF} \textbf{Día 2}} & {\color[HTML]{FFFFFF} \textbf{Día 3}} & {\color[HTML]{FFFFFF} \textbf{Día 4}}  & {\color[HTML]{FFFFFF} \textbf{Día 5}} \\ \hline
		Álvaro Fernández-Alonso Araluce & Testing & Testing & Testing & Testing & Testing \\ \hline
	\end{tabular}
\end{table}


\newpage
\section{Plan de entrega 4}

\subsection{Objetivos de la entrega}

El objetivo principal de esta entrega consiste en crear las primeras secciones para los usuarios Padawans para gestionar sus perfiles y que puedan comenzar a interactuar con los retos propuestos por el Maestro Guidoogway.

\subsection{Listado de HU a desarrollar}

\begin{table}[h]
	\centering
	\begin{tabular}{| p{2.3cm} | p{6.7cm} | p{2cm} |}
		\rowcolor[HTML]{329A9D} 
		{\color[HTML]{FFFFFF} \textbf{Identificador}} & {\color[HTML]{FFFFFF} \textbf{Historias de Usuario}} & {\color[HTML]{FFFFFF} \textbf{Estimación}}  \\ \hline
		HU.14 & Permitir el registro a la plataforma & 8 \\ \hline
		HU.15 & Permitir al usuario Padawan recuperar su contraseña si la ha olvidado & 4 \\ \hline
		HU.16 & Permitir al usuario Padawan modificar sus datos de perfil & 4 \\ \hline
		HU.17 & Permitir al usuario Padawan ver su historial de cargos e ingresos de su cuenta de TdV & 4 \\ \hline
		HU.19 & Permitir al usuario Padawan realizar retos de tipo Pruebas de Nivel & 20 \\ \hline
	\end{tabular}
\end{table}

\newpage

\subsection{Descomposición en tareas de desarrollo}

A continuación incluimos un desglose de las HU en tareas de desarrollo junto con la estimación realizada de su duración.\\

\begin{table}[h]
	\centering
	\begin{tabular}{| p{2.3cm} | p{6.7cm} | p{2cm} |}
		\rowcolor[HTML]{329A9D} 
		{\color[HTML]{FFFFFF} \textbf{HU.14}} & {\color[HTML]{FFFFFF} \textbf{Permitir el registro a la plataforma}} & {\color[HTML]{FFFFFF} \textbf{8}}  \\ \hline
		Tarea 14.1 & Permitir al Maestro Guidoogway gestionar una lista de DNIs que pueden registrarse & 4 \\ \hline
		Tarea 14.2 & Permitir el registro a la plataforma de un usuario con un DNI validado por el Maestro Guidoogway & 4 \\ \hline
	\end{tabular}
\end{table}

\begin{table}[t]
	\centering
	\begin{tabular}{| p{2.3cm} | p{6.7cm} | p{2cm} |}
		\rowcolor[HTML]{329A9D} 
		{\color[HTML]{FFFFFF} \textbf{HU.15}} & {\color[HTML]{FFFFFF} \textbf{Permitir al usuario Padawan recuperar su contraseña si la ha olvidado}} & {\color[HTML]{FFFFFF} \textbf{4}}  \\ \hline
		Tarea 15.1 & Permitir al usuario Padawan recuperar su contraseña & 2 \\ \hline
		Tarea 15.2 & Permitir al sistema asegurar la acción mediante un email de confirmación & 2 \\ \hline
	\end{tabular}
\end{table}

\begin{table}[t]
	\centering
	\begin{tabular}{| p{2.3cm} | p{6.7cm} | p{2cm} |}
		\rowcolor[HTML]{329A9D} 
		{\color[HTML]{FFFFFF} \textbf{HU.16}} & {\color[HTML]{FFFFFF} \textbf{Permitir al usuario Padawan modificar sus datos de perfil}} & {\color[HTML]{FFFFFF} \textbf{4}}  \\ \hline
		Tarea 16.1 & Crear una sección donde el usuario pueda gestionar sus datos de perfil & 4 \\ \hline
	\end{tabular}
\end{table}

\begin{table}[t]
	\centering
	\begin{tabular}{| p{2.3cm} | p{6.7cm} | p{2cm} |}
		\rowcolor[HTML]{329A9D} 
		{\color[HTML]{FFFFFF} \textbf{HU.17}} & {\color[HTML]{FFFFFF} \textbf{Permitir al usuario Padawan ver su historial de cargos e ingresos de su cuenta de TdV}} & {\color[HTML]{FFFFFF} \textbf{4}}  \\ \hline
		Tarea 17.1 & Permitir al usuario Padawan ver su historial de cargos e ingresos de su cuenta de TdV & 4 \\ \hline
	\end{tabular}
\end{table}

\begin{table}[t]
	\centering
	\begin{tabular}{| p{2.3cm} | p{6.7cm} | p{2cm} |}
		\rowcolor[HTML]{329A9D} 
		{\color[HTML]{FFFFFF} \textbf{HU.19}} & {\color[HTML]{FFFFFF} \textbf{Permitir al usuario Padawan realizar retos de tipo Pruebas de Nivel}} & {\color[HTML]{FFFFFF} \textbf{20}}  \\ \hline
		Tarea 19.1 & Permitir al usuario Padawan ver los retos de este tipo que el Maestro Guidoogway ha creado & 4 \\ \hline
		Tarea 19.2 & Permitir al usuario la compra del reto & 2 \\ \hline
		Tarea 19.3 & Permitir al usuario la entrega del reto & 2 \\ \hline
		Tarea 19.4 & Permitir al sistema controlar el número de usuarios Padawan que compran el reto y no permitir más su compra en caso de que exceda el máximo permitido por el Maestro Guidoogway & 2 \\ \hline
		Tarea 19.5 & Permitir al Maestro Guidoogway calificar las entregas & 6 \\ \hline
		Tarea 19.6 & Permitir al Sistema abonar al usuario Padawan la cantidad de TdV correspondiente a la calificación dada & 8 \\ \hline
	\end{tabular}
\end{table}

\newpage
\FloatBarrier

\subsection{Carga prevista en los desarrolladores}

A continuación mostramos la carga de trabajo prevista para cada uno de los desarrolladores en relación a las tareas definidas anteriormente:

\begin{table}[h]
	\centering
	\begin{tabular}{| p{3cm} | p{2cm} | p{2cm} | p{2cm} | p{2cm} |}
		\rowcolor[HTML]{329A9D} 
		{\color[HTML]{FFFFFF} \textbf{Desarrollador}} & {\color[HTML]{FFFFFF} \textbf{Velocidad inicial (días ideales)}} & {\color[HTML]{FFFFFF} \textbf{Dedicación (\% del tiempo)}} & {\color[HTML]{FFFFFF} \textbf{Carga de trabajo (días ideales)}} & {\color[HTML]{FFFFFF} \textbf{\#Tareas aceptadas}}  \\ \hline
		Álvaro Fernández-Alonso Araluce & 15 & 50\% & 15 & 10 \\ \hline
	\end{tabular}
\end{table}

\subsection{Planificación temporal de las tareas}

\begin{table}[h]
	\centering
	\begin{tabular}{| p{2cm} | p{2cm} | p{2cm} | p{2cm} | p{2cm} | p{2cm} |}
		\rowcolor[HTML]{329A9D} 
		 {\color[HTML]{FFFFFF} \textbf{Semana 1}} & {\color[HTML]{FFFFFF} \textbf{Día 1}} & {\color[HTML]{FFFFFF} \textbf{Día 2}} & {\color[HTML]{FFFFFF} \textbf{Día 3}} & {\color[HTML]{FFFFFF} \textbf{Día 4}}  & {\color[HTML]{FFFFFF} \textbf{Día 5}} \\ \hline
		Álvaro Fernández-Alonso Araluce & 14.1 & 14.2 & 15.1 15.2 & 16.1 & 17.1 \\ \hline
	\end{tabular}
\end{table}

\begin{table}[h]
	\centering
	\begin{tabular}{| p{2cm} | p{2cm} | p{2cm} | p{2cm} | p{2cm} | p{2cm} |}
		\rowcolor[HTML]{329A9D} 
		{\color[HTML]{FFFFFF} \textbf{Semana 2}} & {\color[HTML]{FFFFFF} \textbf{Día 1}} & {\color[HTML]{FFFFFF} \textbf{Día 2}} & {\color[HTML]{FFFFFF} \textbf{Día 3}} & {\color[HTML]{FFFFFF} \textbf{Día 4}}  & {\color[HTML]{FFFFFF} \textbf{Día 5}} \\ \hline
		Álvaro Fernández-Alonso Araluce & 19.1 & 19.2 19.3 & 19.4 19.5 & 19.5 & 19.6 \\ \hline
	\end{tabular}
\end{table}

\begin{table}[h]
	\centering
	\begin{tabular}{| p{2cm} | p{2cm} | p{2cm} | p{2cm} | p{2cm} | p{2cm} |}
		\rowcolor[HTML]{329A9D} 
		{\color[HTML]{FFFFFF} \textbf{Semana 3}} & {\color[HTML]{FFFFFF} \textbf{Día 1}} & {\color[HTML]{FFFFFF} \textbf{Día 2}} & {\color[HTML]{FFFFFF} \textbf{Día 3}} & {\color[HTML]{FFFFFF} \textbf{Día 4}}  & {\color[HTML]{FFFFFF} \textbf{Día 5}} \\ \hline
		Álvaro Fernández-Alonso Araluce & 19.6 & Testing & Testing & Testing & Testing \\ \hline
	\end{tabular}
\end{table}


\newpage
\section{Plan de entrega 5}

\subsection{Objetivos de la entrega}

El objetivo principal de esta entrega se va a centrar en permitir al usuario Padawan poder realizar retos de tipo Seguimiento del Maestro Jedi, integrar en nuestro sistema el servicio Twitter, permitir que el usuario almacene Tweets y que el sistema lo recompense automáticamente en caso de haber cumplido el mínimo.

\subsection{Listado de HU a desarrollar}

\begin{table}[h]
	\centering
	\begin{tabular}{| p{2.3cm} | p{6.7cm} | p{2cm} |}
		\rowcolor[HTML]{329A9D} 
		{\color[HTML]{FFFFFF} \textbf{Identificador}} & {\color[HTML]{FFFFFF} \textbf{Historias de Usuario}} & {\color[HTML]{FFFFFF} \textbf{Estimación}}  \\ \hline
		HU.18 & Permitir al usuario Padawan realizar retos de tipo Seguimiento del Maestro Jedi & 40 \\ \hline
	\end{tabular}
\end{table}

\newpage

\subsection{Descomposición en tareas de desarrollo}

A continuación incluimos un desglose de las HU en tareas de desarrollo junto con la estimación realizada de su duración.\\

\begin{table}[h]
	\centering
	\begin{tabular}{| p{2.3cm} | p{6.7cm} | p{2cm} |}
		\rowcolor[HTML]{329A9D} 
		{\color[HTML]{FFFFFF} \textbf{HU.18}} & {\color[HTML]{FFFFFF} \textbf{Permitir al usuario Padawan realizar retos de tipo Seguimiento del Maestro Jedi}} & {\color[HTML]{FFFFFF} \textbf{40}}  \\ \hline
		Tarea 18.1 & Estudiar posibles gemas para gestión de Twitter & 8 \\ \hline
		Tarea 18.2 & Integración de Twitter en el sistema & 12 \\ \hline
		Tarea 18.3 & Permitir al usuario Padawan almacenar Tweets & 8 \\ \hline
		Tarea 18.4 & Permitir al sistema comprobar diariamente el número de Tweets que ha almacenado cada usuario e ingresarle la recompensa en su cuenta de TdV & 12 \\ \hline
	\end{tabular}
\end{table}

\subsection{Carga prevista en los desarrolladores}

A continuación mostramos la carga de trabajo prevista para cada uno de los desarrolladores en relación a las tareas definidas anteriormente:

\begin{table}[h]
	\centering
	\begin{tabular}{| p{3cm} | p{2cm} | p{2cm} | p{2cm} | p{2cm} |}
		\rowcolor[HTML]{329A9D} 
		{\color[HTML]{FFFFFF} \textbf{Desarrollador}} & {\color[HTML]{FFFFFF} \textbf{Velocidad inicial (días ideales)}} & {\color[HTML]{FFFFFF} \textbf{Dedicación (\% del tiempo)}} & {\color[HTML]{FFFFFF} \textbf{Carga de trabajo (días ideales)}} & {\color[HTML]{FFFFFF} \textbf{\#Tareas aceptadas}}  \\ \hline
		Álvaro Fernández-Alonso Araluce & 15 & 50\% & 15 & 4 \\ \hline
	\end{tabular}
\end{table}

\newpage

\subsection{Planificación temporal de las tareas}

\begin{table}[h]
	\centering
	\begin{tabular}{| p{2cm} | p{2cm} | p{2cm} | p{2cm} | p{2cm} | p{2cm} |}
		\rowcolor[HTML]{329A9D} 
		 {\color[HTML]{FFFFFF} \textbf{Semana 1}} & {\color[HTML]{FFFFFF} \textbf{Día 1}} & {\color[HTML]{FFFFFF} \textbf{Día 2}} & {\color[HTML]{FFFFFF} \textbf{Día 3}} & {\color[HTML]{FFFFFF} \textbf{Día 4}}  & {\color[HTML]{FFFFFF} \textbf{Día 5}} \\ \hline
		Álvaro Fernández-Alonso Araluce & 18.1 & 18.1 & 18.2 & 18.2 & 18.2 \\ \hline
	\end{tabular}
\end{table}

\begin{table}[h]
	\centering
	\begin{tabular}{| p{2cm} | p{2cm} | p{2cm} | p{2cm} | p{2cm} | p{2cm} |}
		\rowcolor[HTML]{329A9D} 
		{\color[HTML]{FFFFFF} \textbf{Semana 2}} & {\color[HTML]{FFFFFF} \textbf{Día 1}} & {\color[HTML]{FFFFFF} \textbf{Día 2}} & {\color[HTML]{FFFFFF} \textbf{Día 3}} & {\color[HTML]{FFFFFF} \textbf{Día 4}}  & {\color[HTML]{FFFFFF} \textbf{Día 5}} \\ \hline
		Álvaro Fernández-Alonso Araluce & 18.3 & 18.3 & 18.4 & 18.4 & 18.4 \\ \hline
	\end{tabular}
\end{table}

\begin{table}[h]
	\centering
	\begin{tabular}{| p{2cm} | p{2cm} | p{2cm} | p{2cm} | p{2cm} | p{2cm} |}
		\rowcolor[HTML]{329A9D} 
		{\color[HTML]{FFFFFF} \textbf{Semana 3}} & {\color[HTML]{FFFFFF} \textbf{Día 1}} & {\color[HTML]{FFFFFF} \textbf{Día 2}} & {\color[HTML]{FFFFFF} \textbf{Día 3}} & {\color[HTML]{FFFFFF} \textbf{Día 4}}  & {\color[HTML]{FFFFFF} \textbf{Día 5}} \\ \hline
		Álvaro Fernández-Alonso Araluce & Testing & Testing & Testing & Testing & Testing \\ \hline
	\end{tabular}
\end{table}


\newpage
\section{Plan de entrega 6}

\subsection{Objetivos de la entrega}

El objetivo principal de esta entrega se va a centrar en permitir al usuario Padawan poder realizar retos de tipo Dominio de la Fuerza y Audiencia ante el Senado Galáctico.

\subsection{Listado de HU a desarrollar}

\begin{table}[h]
	\centering
	\begin{tabular}{| p{2.3cm} | p{6.7cm} | p{2cm} |}
		\rowcolor[HTML]{329A9D} 
		{\color[HTML]{FFFFFF} \textbf{Identificador}} & {\color[HTML]{FFFFFF} \textbf{Historias de Usuario}} & {\color[HTML]{FFFFFF} \textbf{Estimación}}  \\ \hline
		HU.20 & Permitir al usuario Padawan realizar retos de tipo Dominio de la Fuerza & 20 \\ \hline
		HU.21 & Permitir al usuario Padawan realizar retos de tipo Audiencia ante el Senado Galáctico & 20 \\ \hline
	\end{tabular}
\end{table}


\subsection{Descomposición en tareas de desarrollo}

A continuación incluimos un desglose de las HU en tareas de desarrollo junto con la estimación realizada de su duración.\\

\begin{table}[h]
	\centering
	\begin{tabular}{| p{2.3cm} | p{6.7cm} | p{2cm} |}
		\rowcolor[HTML]{329A9D} 
		{\color[HTML]{FFFFFF} \textbf{HU.20}} & {\color[HTML]{FFFFFF} \textbf{Permitir al usuario Padawan realizar retos de tipo Dominio de la Fuerza}} & {\color[HTML]{FFFFFF} \textbf{20}}  \\ \hline
		Tarea 20.1 & Permitir al sistema cobrar 1 día de vida a todos los usuarios Padawans cuando el Maestro Guidoogway publica un reto de este tipo & 4 \\ \hline
		Tarea 20.2 & Permitir a los usuarios Padawans realizar el reto tipo cuestionario & 6 \\ \hline
		Tarea 20.3 & Permitir al sistema dar una respuesta positiva o negativa dependiendo del resultado del test & 5 \\ \hline
		Tarea 20.4 & Permitir al sistema reingresar al usuario Padawan 1 TdV si realiza correctamente el reto & 5 \\ \hline
	\end{tabular}
\end{table}

\begin{table}[h]
	\centering
	\begin{tabular}{| p{2.3cm} | p{6.7cm} | p{2cm} |}
		\rowcolor[HTML]{329A9D} 
		{\color[HTML]{FFFFFF} \textbf{HU.21}} & {\color[HTML]{FFFFFF} \textbf{Permitir al usuario Padawan realizar retos de tipo Audiencia ante el Senado Galáctico}} & {\color[HTML]{FFFFFF} \textbf{20}}  \\ \hline
		Tarea 21.1 & Permitir al usuario visualizar los retos de Senado Galáctico que el Maestro Guidoogway publique. & 4 \\ \hline
		Tarea 21.2 & Permitir al Maestro Guidoogway recibir las entregas de los usuarios Padawan & 6 \\ \hline
		Tarea 21.3 & Permitir al Maestro Guidoogway corregir las entregas de este tipo & 8 \\ \hline
		Tarea 21.4 & Permitir al sistema ingresar el TdV equivalente al badge obtenido & 2 \\ \hline
	\end{tabular}
\end{table}

\subsection{Carga prevista en los desarrolladores}

A continuación mostramos la carga de trabajo prevista para cada uno de los desarrolladores en relación a las tareas definidas anteriormente:

\begin{table}[h]
	\centering
	\begin{tabular}{| p{3cm} | p{2cm} | p{2cm} | p{2cm} | p{2cm} |}
		\rowcolor[HTML]{329A9D} 
		{\color[HTML]{FFFFFF} \textbf{Desarrollador}} & {\color[HTML]{FFFFFF} \textbf{Velocidad inicial (días ideales)}} & {\color[HTML]{FFFFFF} \textbf{Dedicación (\% del tiempo)}} & {\color[HTML]{FFFFFF} \textbf{Carga de trabajo (días ideales)}} & {\color[HTML]{FFFFFF} \textbf{\#Tareas aceptadas}}  \\ \hline
		Álvaro Fernández-Alonso Araluce & 15 & 50\% & 15 & 8 \\ \hline
	\end{tabular}
\end{table}

\subsection{Planificación temporal de las tareas}

\begin{table}[h]
	\centering
	\begin{tabular}{| p{2cm} | p{2cm} | p{2cm} | p{2cm} | p{2cm} | p{2cm} |}
		\rowcolor[HTML]{329A9D} 
		 {\color[HTML]{FFFFFF} \textbf{Semana 1}} & {\color[HTML]{FFFFFF} \textbf{Día 1}} & {\color[HTML]{FFFFFF} \textbf{Día 2}} & {\color[HTML]{FFFFFF} \textbf{Día 3}} & {\color[HTML]{FFFFFF} \textbf{Día 4}}  & {\color[HTML]{FFFFFF} \textbf{Día 5}} \\ \hline
		Álvaro Fernández-Alonso Araluce & 20.1 & 20.2 & 20.2 20.3 & 20.3 20.4 & 20.4 \\ \hline
	\end{tabular}
\end{table}

\begin{table}[h]
	\centering
	\begin{tabular}{| p{2cm} | p{2cm} | p{2cm} | p{2cm} | p{2cm} | p{2cm} |}
		\rowcolor[HTML]{329A9D} 
		{\color[HTML]{FFFFFF} \textbf{Semana 2}} & {\color[HTML]{FFFFFF} \textbf{Día 1}} & {\color[HTML]{FFFFFF} \textbf{Día 2}} & {\color[HTML]{FFFFFF} \textbf{Día 3}} & {\color[HTML]{FFFFFF} \textbf{Día 4}}  & {\color[HTML]{FFFFFF} \textbf{Día 5}} \\ \hline
		Álvaro Fernández-Alonso Araluce & 21.1 & 21.2 & 21.2 21.3 & 21.6 & 21.6 21.4 \\ \hline
	\end{tabular}
\end{table}

\begin{table}[h]
	\centering
	\begin{tabular}{| p{2cm} | p{2cm} | p{2cm} | p{2cm} | p{2cm} | p{2cm} |}
		\rowcolor[HTML]{329A9D} 
		{\color[HTML]{FFFFFF} \textbf{Semana 3}} & {\color[HTML]{FFFFFF} \textbf{Día 1}} & {\color[HTML]{FFFFFF} \textbf{Día 2}} & {\color[HTML]{FFFFFF} \textbf{Día 3}} & {\color[HTML]{FFFFFF} \textbf{Día 4}}  & {\color[HTML]{FFFFFF} \textbf{Día 5}} \\ \hline
		Álvaro Fernández-Alonso Araluce & Testing & Testing & Testing & Testing & Testing \\ \hline
	\end{tabular}
\end{table}


\newpage
\section{Plan de entrega 7}

\subsection{Objetivos de la entrega}

El objetivo principal de esta entrega se va a centrar en permitir al usuario Padawan poder realizar retos de tipo Comida.

\subsection{Listado de HU a desarrollar}

\begin{table}[h]
	\centering
	\begin{tabular}{| p{2.3cm} | p{6.7cm} | p{2cm} |}
		\rowcolor[HTML]{329A9D} 
		{\color[HTML]{FFFFFF} \textbf{Identificador}} & {\color[HTML]{FFFFFF} \textbf{Historias de Usuario}} & {\color[HTML]{FFFFFF} \textbf{Estimación}}  \\ \hline
		HU.22 & Permitir al usuario Padawan realizar retos de tipo Comida & 40 \\ \hline
	\end{tabular}
\end{table}

\newpage

\subsection{Descomposición en tareas de desarrollo}

A continuación incluimos un desglose de las HU en tareas de desarrollo junto con la estimación realizada de su duración.\\

\begin{table}[h]
	\centering
	\begin{tabular}{| p{2.3cm} | p{6.7cm} | p{2cm} |}
		\rowcolor[HTML]{329A9D} 
		{\color[HTML]{FFFFFF} \textbf{HU.22}} & {\color[HTML]{FFFFFF} \textbf{Permitir al usuario Padawan realizar retos de tipo Comida}} & {\color[HTML]{FFFFFF} \textbf{40}}  \\ \hline
		Tarea 22.1 & Permitir al usuario Padawan visualizar los retos de este tipo publicados por el Guardián del Tiempo & 4 \\ \hline
		Tarea 22.2 & Permitir a los usuarios Padawans comprar un reto individual & 4 \\ \hline
		Tarea 22.3 & Permitir a los usuarios Padawans comprar retos por clanes & 4 \\ \hline
		Tarea 22.4 & Permitir al sistema cobrar las compras del usuario & 4 \\ \hline
		Tarea 22.5 & Permitir al usuario Padawan entregar el reto a través de la plataforma & 4 \\ \hline
		Tarea 22.6 & Permitir al sistema mostrar una barra de progreso a modo de estómago & 6 \\ \hline
		Tarea 22.7 & Permitir usuario Padawan ver el estado de su entrega en cada momento & 4 \\ \hline
		Tarea 22.8 & Permitir al usuario Padawan acceder a su entrega & 2 \\ \hline
		Tarea 22.9 & Permitir al usuario Padawan ver el Feedback del Maestro Guidoogway & 4 \\ \hline
		Tarea 22.10 & Permitir al sistema ingresar el TdV asociado al Feedback del usuario Padawan & 4 \\
		 \hline
		Tarea 22.11 & Permitir al sistema ingresar el TdV asociado al Feedback del clan Padawan & 4 \\ \hline
	\end{tabular}
\end{table}

\newpage

\subsection{Carga prevista en los desarrolladores}

A continuación mostramos la carga de trabajo prevista para cada uno de los desarrolladores en relación a las tareas definidas anteriormente:

\begin{table}[h]
	\centering
	\begin{tabular}{| p{3cm} | p{2cm} | p{2cm} | p{2cm} | p{2cm} |}
		\rowcolor[HTML]{329A9D} 
		{\color[HTML]{FFFFFF} \textbf{Desarrollador}} & {\color[HTML]{FFFFFF} \textbf{Velocidad inicial (días ideales)}} & {\color[HTML]{FFFFFF} \textbf{Dedicación (\% del tiempo)}} & {\color[HTML]{FFFFFF} \textbf{Carga de trabajo (días ideales)}} & {\color[HTML]{FFFFFF} \textbf{\#Tareas aceptadas}}  \\ \hline
		Álvaro Fernández-Alonso Araluce & 15 & 50\% & 15 & 11 \\ \hline
	\end{tabular}
\end{table}


\subsection{Planificación temporal de las tareas}

\begin{table}[h]
	\centering
	\begin{tabular}{| p{2cm} | p{2cm} | p{2cm} | p{2cm} | p{2cm} | p{2cm} |}
		\rowcolor[HTML]{329A9D} 
		 {\color[HTML]{FFFFFF} \textbf{Semana 1}} & {\color[HTML]{FFFFFF} \textbf{Día 1}} & {\color[HTML]{FFFFFF} \textbf{Día 2}} & {\color[HTML]{FFFFFF} \textbf{Día 3}} & {\color[HTML]{FFFFFF} \textbf{Día 4}}  & {\color[HTML]{FFFFFF} \textbf{Día 5}} \\ \hline
		Álvaro Fernández-Alonso Araluce & 22.1 & 22.2 & 22.3 & 22.4 & 22.5 \\ \hline
	\end{tabular}
\end{table}

\begin{table}[h]
	\centering
	\begin{tabular}{| p{2cm} | p{2cm} | p{2cm} | p{2cm} | p{2cm} | p{2cm} |}
		\rowcolor[HTML]{329A9D} 
		{\color[HTML]{FFFFFF} \textbf{Semana 2}} & {\color[HTML]{FFFFFF} \textbf{Día 1}} & {\color[HTML]{FFFFFF} \textbf{Día 2}} & {\color[HTML]{FFFFFF} \textbf{Día 3}} & {\color[HTML]{FFFFFF} \textbf{Día 4}}  & {\color[HTML]{FFFFFF} \textbf{Día 5}} \\ \hline
		Álvaro Fernández-Alonso Araluce & 22.6 & 22.6 22.7 & 22.7 22.8 & 22.9 & 22.10 \\ \hline
	\end{tabular}
\end{table}

\begin{table}[h]
	\centering
	\begin{tabular}{| p{2cm} | p{2cm} | p{2cm} | p{2cm} | p{2cm} | p{2cm} |}
		\rowcolor[HTML]{329A9D} 
		{\color[HTML]{FFFFFF} \textbf{Semana 3}} & {\color[HTML]{FFFFFF} \textbf{Día 1}} & {\color[HTML]{FFFFFF} \textbf{Día 2}} & {\color[HTML]{FFFFFF} \textbf{Día 3}} & {\color[HTML]{FFFFFF} \textbf{Día 4}}  & {\color[HTML]{FFFFFF} \textbf{Día 5}} \\ \hline
		Álvaro Fernández-Alonso Araluce & 22.11 & Testing & Testing & Testing & Testing \\ \hline
	\end{tabular}
\end{table}


\newpage
\input{capitulos/plan_de_entregas/08}
\newpage
\section{Plan de entrega 9}

\subsection{Objetivos de la entrega}

El objetivo principal de esta entrega se va a centrar en permitir al usuario Padawan poder realizar retos de tipo Deporte.

\subsection{Listado de HU a desarrollar}

\begin{table}[h]
	\centering
	\begin{tabular}{| p{2.3cm} | p{6.7cm} | p{2cm} |}
		\rowcolor[HTML]{329A9D} 
		{\color[HTML]{FFFFFF} \textbf{Identificador}} & {\color[HTML]{FFFFFF} \textbf{Historias de Usuario}} & {\color[HTML]{FFFFFF} \textbf{Estimación}}  \\ \hline
		HU.24 & Permitir al usuario Padawan realizar retos de tipo Deporte & 20 \\ \hline
	\end{tabular}
\end{table}

\newpage

\subsection{Descomposición en tareas de desarrollo}

A continuación incluimos un desglose de las HU en tareas de desarrollo junto con la estimación realizada de su duración.\\

\begin{table}[h]
	\centering
	\begin{tabular}{| p{2.3cm} | p{6.7cm} | p{2cm} |}
		\rowcolor[HTML]{329A9D} 
		{\color[HTML]{FFFFFF} \textbf{HU.24}} & {\color[HTML]{FFFFFF} \textbf{Permitir al usuario Padawan realizar retos de tipo Deporte}} & {\color[HTML]{FFFFFF} \textbf{40}}  \\ \hline
		Tarea 24.1 & Permitir a los usuarios Padawans visualizar el reto deportivo correspondiente a la semana actual & 4 \\ \hline
		Tarea 24.2 & Integrar Runtastic con el sistema para que el usuario pueda hacer Login desde la aplicación y el sistema pueda descargar la información relativa a sus sesiones deportivas de Runtastic & 20 \\ \hline
		Tarea 24.3 & Permitir al usuario introducir en el sistema sus credenciales de Runtastic & 4 \\ \hline
		Tarea 24.4 & Permitir al sistema comprobar al final de cada semana si ha superado los parámetros mínimos planteados por el Maestro Guidoogway & 16 \\ \hline
		Tarea 24.5 & Permitir al sistema enviar un correo electrónico al usuario Padawan con el resumen semanal de sus sesiones deportivas y con el resultado de si ha superado o no el reto & 12 \\ \hline
		Tarea 24.6 & Permitir al sistema ingresar el TdV correspondiente a la superación del reto si se ha superado & 4 \\ \hline
	\end{tabular}
\end{table}

\newpage

\subsection{Carga prevista en los desarrolladores}

A continuación mostramos la carga de trabajo prevista para cada uno de los desarrolladores en relación a las tareas definidas anteriormente:

\begin{table}[h]
	\centering
	\begin{tabular}{| p{3cm} | p{2cm} | p{2cm} | p{2cm} | p{2cm} |}
		\rowcolor[HTML]{329A9D} 
		{\color[HTML]{FFFFFF} \textbf{Desarrollador}} & {\color[HTML]{FFFFFF} \textbf{Velocidad inicial (días ideales)}} & {\color[HTML]{FFFFFF} \textbf{Dedicación (\% del tiempo)}} & {\color[HTML]{FFFFFF} \textbf{Carga de trabajo (días ideales)}} & {\color[HTML]{FFFFFF} \textbf{\#Tareas aceptadas}}  \\ \hline
		Álvaro Fernández-Alonso Araluce & 15 & 50\% & 15 & 6 \\ \hline
	\end{tabular}
\end{table}


\subsection{Planificación temporal de las tareas}

\begin{table}[h]
	\centering
	\begin{tabular}{| p{2cm} | p{2cm} | p{2cm} | p{2cm} | p{2cm} | p{2cm} |}
		\rowcolor[HTML]{329A9D} 
		 {\color[HTML]{FFFFFF} \textbf{Semana 1}} & {\color[HTML]{FFFFFF} \textbf{Día 1}} & {\color[HTML]{FFFFFF} \textbf{Día 2}} & {\color[HTML]{FFFFFF} \textbf{Día 3}} & {\color[HTML]{FFFFFF} \textbf{Día 4}}  & {\color[HTML]{FFFFFF} \textbf{Día 5}} \\ \hline
		Álvaro Fernández-Alonso Araluce & 24.1 & 24.2 & 24.2 & 24.2 & 24.2 \\ \hline
	\end{tabular}
\end{table}

\begin{table}[h]
	\centering
	\begin{tabular}{| p{2cm} | p{2cm} | p{2cm} | p{2cm} | p{2cm} | p{2cm} |}
		\rowcolor[HTML]{329A9D} 
		{\color[HTML]{FFFFFF} \textbf{Semana 2}} & {\color[HTML]{FFFFFF} \textbf{Día 1}} & {\color[HTML]{FFFFFF} \textbf{Día 2}} & {\color[HTML]{FFFFFF} \textbf{Día 3}} & {\color[HTML]{FFFFFF} \textbf{Día 4}}  & {\color[HTML]{FFFFFF} \textbf{Día 5}} \\ \hline
		Álvaro Fernández-Alonso Araluce & 24.2 & 24.3 & 24.4 & 24.4 & 24.4 \\ \hline
	\end{tabular}
\end{table}

\begin{table}[h]
	\centering
	\begin{tabular}{| p{2cm} | p{2cm} | p{2cm} | p{2cm} | p{2cm} | p{2cm} |}
		\rowcolor[HTML]{329A9D} 
		{\color[HTML]{FFFFFF} \textbf{Semana 3}} & {\color[HTML]{FFFFFF} \textbf{Día 1}} & {\color[HTML]{FFFFFF} \textbf{Día 2}} & {\color[HTML]{FFFFFF} \textbf{Día 3}} & {\color[HTML]{FFFFFF} \textbf{Día 4}}  & {\color[HTML]{FFFFFF} \textbf{Día 5}} \\ \hline
		Álvaro Fernández-Alonso Araluce & 24.4 & 24.5 & 24.5 & 24.5 & 24.6 \\ \hline
	\end{tabular}
\end{table}


\newpage
\section{Plan de entrega 10}

\subsection{Objetivos de la entrega}

El objetivo principal de esta entrega se va a centrar en permitir al usuario Padawan poder participar en las apuestas creadas por el Maestro Guidoogway.

\subsection{Listado de HU a desarrollar}

\begin{table}[h]
	\centering
	\begin{tabular}{| p{2.3cm} | p{6.7cm} | p{2cm} |}
		\rowcolor[HTML]{329A9D} 
		{\color[HTML]{FFFFFF} \textbf{Identificador}} & {\color[HTML]{FFFFFF} \textbf{Historias de Usuario}} & {\color[HTML]{FFFFFF} \textbf{Estimación}}  \\ \hline
		HU.25 & Permitir Maestro Guidoogway gestionar las apuestas & 20 \\ \hline
		HU.26 & Permitir al usuario Padawan participar en las apuestas publicadas por el Maestro Guidoogway & 20 \\ \hline
	\end{tabular}
\end{table}

\newpage

\subsection{Descomposición en tareas de desarrollo}

A continuación incluimos un desglose de las HU en tareas de desarrollo junto con la estimación realizada de su duración.\\

\begin{table}[h]
	\centering
	\begin{tabular}{| p{2.3cm} | p{6.7cm} | p{2cm} |}
		\rowcolor[HTML]{329A9D} 
		{\color[HTML]{FFFFFF} \textbf{HU.25}} & {\color[HTML]{FFFFFF} \textbf{Permitir al usuario Padawan participar en las apuestas publicadas por el Maestro Guidoogway}} & {\color[HTML]{FFFFFF} \textbf{28}}  \\ \hline
		Tarea 25.1 & Permitir al Maestro Guidoogway crear/editar apuestas & 4 \\ \hline
		Tarea 25.2 & Permitir al Maestro Guidoogway publicar la apuesta & 4 \\ \hline
		Tarea 25.3 & Permitir al Maestro Guidoogway cerrar y pagar la apuesta & 8 \\ \hline
		Tarea 25.4 & Permitir al Maestro Guidoogway solo parar la apuesta & 4 \\ \hline
		Tarea 25.5 & Permitir al Maestro Guidoogway ocultar la apuesta & 4 \\ \hline
		Tarea 25.6 & Permitir al Maestro Guidoogway añadir opciones a la apuesta y marcar una de ellas como correcta & 4 \\ \hline
	\end{tabular}
\end{table}

\begin{table}[h]
	\centering
	\begin{tabular}{| p{2.3cm} | p{6.7cm} | p{2cm} |}
		\rowcolor[HTML]{329A9D} 
		{\color[HTML]{FFFFFF} \textbf{HU.26}} & {\color[HTML]{FFFFFF} \textbf{Permitir al usuario Padawan participar en las apuestas publicadas por el Maestro Guidoogway}} & {\color[HTML]{FFFFFF} \textbf{12}}  \\ \hline
		Tarea 26.1 & Permitir a los usuarios Padawans visualizar las apuestas publicadas por el Maestro Guidoogway & 4 \\ \hline
		Tarea 26.2 & Permitir al usuario Padawan apostar una cantidad de TdV a una opción de la apuesta & 8 \\ \hline
	\end{tabular}
\end{table}

\newpage

\subsection{Carga prevista en los desarrolladores}

A continuación mostramos la carga de trabajo prevista para cada uno de los desarrolladores en relación a las tareas definidas anteriormente:

\begin{table}[h]
	\centering
	\begin{tabular}{| p{3cm} | p{2cm} | p{2cm} | p{2cm} | p{2cm} |}
		\rowcolor[HTML]{329A9D} 
		{\color[HTML]{FFFFFF} \textbf{Desarrollador}} & {\color[HTML]{FFFFFF} \textbf{Velocidad inicial (días ideales)}} & {\color[HTML]{FFFFFF} \textbf{Dedicación (\% del tiempo)}} & {\color[HTML]{FFFFFF} \textbf{Carga de trabajo (días ideales)}} & {\color[HTML]{FFFFFF} \textbf{\#Tareas aceptadas}}  \\ \hline
		Álvaro Fernández-Alonso Araluce & 15 & 50\% & 15 & 8 \\ \hline
	\end{tabular}
\end{table}


\subsection{Planificación temporal de las tareas}

\begin{table}[h]
	\centering
	\begin{tabular}{| p{2cm} | p{2cm} | p{2cm} | p{2cm} | p{2cm} | p{2cm} |}
		\rowcolor[HTML]{329A9D} 
		 {\color[HTML]{FFFFFF} \textbf{Semana 1}} & {\color[HTML]{FFFFFF} \textbf{Día 1}} & {\color[HTML]{FFFFFF} \textbf{Día 2}} & {\color[HTML]{FFFFFF} \textbf{Día 3}} & {\color[HTML]{FFFFFF} \textbf{Día 4}}  & {\color[HTML]{FFFFFF} \textbf{Día 5}} \\ \hline
		Álvaro Fernández-Alonso Araluce & 25.1 & 25.2 & 25.3 & 25.3 & 25.4 \\ \hline
	\end{tabular}
\end{table}

\begin{table}[h]
	\centering
	\begin{tabular}{| p{2cm} | p{2cm} | p{2cm} | p{2cm} | p{2cm} | p{2cm} |}
		\rowcolor[HTML]{329A9D} 
		{\color[HTML]{FFFFFF} \textbf{Semana 2}} & {\color[HTML]{FFFFFF} \textbf{Día 1}} & {\color[HTML]{FFFFFF} \textbf{Día 2}} & {\color[HTML]{FFFFFF} \textbf{Día 3}} & {\color[HTML]{FFFFFF} \textbf{Día 4}}  & {\color[HTML]{FFFFFF} \textbf{Día 5}} \\ \hline
		Álvaro Fernández-Alonso Araluce & 25.5 & 25.6 & 26.1 & 26.2 & 26.3 \\ \hline
	\end{tabular}
\end{table}

\begin{table}[h]
	\centering
	\begin{tabular}{| p{2cm} | p{2cm} | p{2cm} | p{2cm} | p{2cm} | p{2cm} |}
		\rowcolor[HTML]{329A9D} 
		{\color[HTML]{FFFFFF} \textbf{Semana 3}} & {\color[HTML]{FFFFFF} \textbf{Día 1}} & {\color[HTML]{FFFFFF} \textbf{Día 2}} & {\color[HTML]{FFFFFF} \textbf{Día 3}} & {\color[HTML]{FFFFFF} \textbf{Día 4}}  & {\color[HTML]{FFFFFF} \textbf{Día 5}} \\ \hline
		Álvaro Fernández-Alonso Araluce & Testing & Testing & Testing &Testing & Testing \\ \hline
	\end{tabular}
\end{table}


\newpage
\section{Plan de entrega 11}

\subsection{Objetivos de la entrega}

El objetivo principal de esta entrega se va a centrar en permitir al usuario Padawan poder realizar donaciones y crear las Cartas de Privilegios.

\subsection{Listado de HU a desarrollar}

\begin{table}[h]
	\centering
	\begin{tabular}{| p{2.3cm} | p{6.7cm} | p{2cm} |}
		\rowcolor[HTML]{329A9D} 
		{\color[HTML]{FFFFFF} \textbf{Identificador}} & {\color[HTML]{FFFFFF} \textbf{Historias de Usuario}} & {\color[HTML]{FFFFFF} \textbf{Estimación}}  \\ \hline
		HU.27 & Permitir al usuario Padawan realizar donaciones & 12 \\ \hline
		HU.28 & Crear cartas de privilegios con un comportamiento específico por cada una de ellas & 28 \\ \hline
	\end{tabular}
\end{table}

\newpage

\subsection{Descomposición en tareas de desarrollo}

A continuación incluimos un desglose de las HU en tareas de desarrollo junto con la estimación realizada de su duración.\\

\begin{table}[h]
	\centering
	\begin{tabular}{| p{2.3cm} | p{6.7cm} | p{2cm} |}
		\rowcolor[HTML]{329A9D} 
		{\color[HTML]{FFFFFF} \textbf{HU.27}} & {\color[HTML]{FFFFFF} \textbf{Permitir al usuario Padawan realizar donaciones}} & {\color[HTML]{FFFFFF} \textbf{12}}  \\ \hline
		Tarea 27.1 & Permitir al Maestro Guidoogway seleccionar un usuario distinto al suyo para donar una cantidad de TdV & 8 \\ \hline
		Tarea 27.2 & No permitir que un usuario Padawan pueda donar a otro usuario Padawan que tenga más de 7 días de vida & 2 \\ \hline
		Tarea 27.3 & No permitir que un usuario Padawan pueda donarse a sí mismo & 1 \\ \hline
		Tarea 27.4 & No permitir que un usuario Padawan donar más de una vez a otro usuario Padawan & 1 \\ \hline
	\end{tabular}
\end{table}

\begin{table}[h]
	\centering
	\begin{tabular}{| p{2.3cm} | p{6.7cm} | p{2cm} |}
		\rowcolor[HTML]{329A9D} 
		{\color[HTML]{FFFFFF} \textbf{HU.28}} & {\color[HTML]{FFFFFF} \textbf{Crear Cartas de Privilegios con un comportamiento específico por cada una de ellas}} & {\color[HTML]{FFFFFF} \textbf{28}}  \\ \hline
		Tarea 28.1 & Permitir al Maestro Guidoogway gestionar las Cartas de Privilegios & 4 \\ \hline
		Tarea 28.2 & Crear una sección donde cualquier usuario pueda visualizar esas cartas & 4 \\ \hline
		Tarea 28.3 & Permitir al usuario Padawan comprar Cartas de Privilegios & 4 \\ \hline
		Tarea 28.4 & Permitir al sistema comprobar cada una de las cartas para aplicar diferentes acciones en diferentes espacios de tiempo & 16 \\ \hline
	\end{tabular}
\end{table}

\newpage

\subsection{Carga prevista en los desarrolladores}

A continuación mostramos la carga de trabajo prevista para cada uno de los desarrolladores en relación a las tareas definidas anteriormente:

\begin{table}[h]
	\centering
	\begin{tabular}{| p{3cm} | p{2cm} | p{2cm} | p{2cm} | p{2cm} |}
		\rowcolor[HTML]{329A9D} 
		{\color[HTML]{FFFFFF} \textbf{Desarrollador}} & {\color[HTML]{FFFFFF} \textbf{Velocidad inicial (días ideales)}} & {\color[HTML]{FFFFFF} \textbf{Dedicación (\% del tiempo)}} & {\color[HTML]{FFFFFF} \textbf{Carga de trabajo (días ideales)}} & {\color[HTML]{FFFFFF} \textbf{\#Tareas aceptadas}}  \\ \hline
		Álvaro Fernández-Alonso Araluce & 15 & 50\% & 15 & 8 \\ \hline
	\end{tabular}
\end{table}


\subsection{Planificación temporal de las tareas}

\begin{table}[h]
	\centering
	\begin{tabular}{| p{2cm} | p{2cm} | p{2cm} | p{2cm} | p{2cm} | p{2cm} |}
		\rowcolor[HTML]{329A9D} 
		{\color[HTML]{FFFFFF} \textbf{Semana 1}} & {\color[HTML]{FFFFFF} \textbf{Día 1}} & {\color[HTML]{FFFFFF} \textbf{Día 2}} & {\color[HTML]{FFFFFF} \textbf{Día 3}} & {\color[HTML]{FFFFFF} \textbf{Día 4}}  & {\color[HTML]{FFFFFF} \textbf{Día 5}} \\ \hline
		Álvaro Fernández-Alonso Araluce & 27.1 & 27.1 & 27.2 27.3 27.4 & 28.1 & 28.2 \\ \hline
	\end{tabular}
\end{table}

\begin{table}[h]
	\centering
	\begin{tabular}{| p{2cm} | p{2cm} | p{2cm} | p{2cm} | p{2cm} | p{2cm} |}
		\rowcolor[HTML]{329A9D} 
		{\color[HTML]{FFFFFF} \textbf{Semana 2}} & {\color[HTML]{FFFFFF} \textbf{Día 1}} & {\color[HTML]{FFFFFF} \textbf{Día 2}} & {\color[HTML]{FFFFFF} \textbf{Día 3}} & {\color[HTML]{FFFFFF} \textbf{Día 4}}  & {\color[HTML]{FFFFFF} \textbf{Día 5}} \\ \hline
		Álvaro Fernández-Alonso Araluce & 28.3 & 28.4 & 28.4 & 28.4 & 28.4 \\ \hline
	\end{tabular}
\end{table}

\begin{table}[h]
	\centering
	\begin{tabular}{| p{2cm} | p{2cm} | p{2cm} | p{2cm} | p{2cm} | p{2cm} |}
		\rowcolor[HTML]{329A9D} 
		{\color[HTML]{FFFFFF} \textbf{Semana 3}} & {\color[HTML]{FFFFFF} \textbf{Día 1}} & {\color[HTML]{FFFFFF} \textbf{Día 2}} & {\color[HTML]{FFFFFF} \textbf{Día 3}} & {\color[HTML]{FFFFFF} \textbf{Día 4}}  & {\color[HTML]{FFFFFF} \textbf{Día 5}} \\ \hline
		Álvaro Fernández-Alonso Araluce & Testing & Testing & Testing &Testing & Testing \\ \hline
	\end{tabular}
\end{table}


\newpage
\section{Plan de entrega 12}

\subsection{Objetivos de la entrega}

El objetivo principal de esta entrega se va a centrar en permitir la comunicación entre usuarios por medio de un chat interno.

\subsection{Listado de HU a desarrollar}

\begin{table}[h]
	\centering
	\begin{tabular}{| p{2.3cm} | p{6.7cm} | p{2cm} |}
		\rowcolor[HTML]{329A9D} 
		{\color[HTML]{FFFFFF} \textbf{Identificador}} & {\color[HTML]{FFFFFF} \textbf{Historias de Usuario}} & {\color[HTML]{FFFFFF} \textbf{Estimación}}  \\ \hline
		HU.30 & Permitir a los diferentes usuarios chatear entre ellos & 48 \\ \hline
	\end{tabular}
\end{table}

\newpage

\subsection{Descomposición en tareas de desarrollo}

A continuación incluimos un desglose de las HU en tareas de desarrollo junto con la estimación realizada de su duración.\\

\begin{table}[h]
	\centering
	\begin{tabular}{| p{2.3cm} | p{6.7cm} | p{2cm} |}
		\rowcolor[HTML]{329A9D} 
		{\color[HTML]{FFFFFF} \textbf{HU.30}} & {\color[HTML]{FFFFFF} \textbf{Permitir a los diferentes usuarios chatear entre ellos}} & {\color[HTML]{FFFFFF} \textbf{48}}  \\ \hline
		Tarea 30.1 & Permitir a los usuarios visualizar una lista de todos los contactos del sistema & 8 \\ \hline
		Tarea 30.2 & Permitir a los usuarios visualizar un chat grupal común & 8 \\ \hline
		Tarea 30.3 & Crear salas de chat de usuario a usuario & 12 \\ \hline
		Tarea 30.4 & Crear salas de chat grupales & 12 \\ \hline
		Tarea 30.5 & En una sala de chat, recibir los mensajes dinámicamente sin tener que recargar & 4 \\ \hline
		Tarea 30.6 & Marcar los mensajes como vistos & 2 \\ \hline
		Tarea 30.7 & Marcar visualmente las salas que contengan mensajes pendientes & 2 \\ \hline
	\end{tabular}
\end{table}

\subsection{Carga prevista en los desarrolladores}

A continuación mostramos la carga de trabajo prevista para cada uno de los desarrolladores en relación a las tareas definidas anteriormente:

\begin{table}[h]
	\centering
	\begin{tabular}{| p{3cm} | p{2cm} | p{2cm} | p{2cm} | p{2cm} |}
		\rowcolor[HTML]{329A9D} 
		{\color[HTML]{FFFFFF} \textbf{Desarrollador}} & {\color[HTML]{FFFFFF} \textbf{Velocidad inicial (días ideales)}} & {\color[HTML]{FFFFFF} \textbf{Dedicación (\% del tiempo)}} & {\color[HTML]{FFFFFF} \textbf{Carga de trabajo (días ideales)}} & {\color[HTML]{FFFFFF} \textbf{\#Tareas aceptadas}}  \\ \hline
		Álvaro Fernández-Alonso Araluce & 15 & 50\% & 15 & 7 \\ \hline
	\end{tabular}
\end{table}

\newpage

\subsection{Planificación temporal de las tareas}

\begin{table}[h]
	\centering
	\begin{tabular}{| p{2cm} | p{2cm} | p{2cm} | p{2cm} | p{2cm} | p{2cm} |}
		\rowcolor[HTML]{329A9D} 
		{\color[HTML]{FFFFFF} \textbf{Semana 1}} & {\color[HTML]{FFFFFF} \textbf{Día 1}} & {\color[HTML]{FFFFFF} \textbf{Día 2}} & {\color[HTML]{FFFFFF} \textbf{Día 3}} & {\color[HTML]{FFFFFF} \textbf{Día 4}}  & {\color[HTML]{FFFFFF} \textbf{Día 5}} \\ \hline
		Álvaro Fernández-Alonso Araluce & 30.1 & 30.1 & 30.2 & 30.2 & 30.3 \\ \hline
	\end{tabular}
\end{table}

\begin{table}[h]
	\centering
	\begin{tabular}{| p{2cm} | p{2cm} | p{2cm} | p{2cm} | p{2cm} | p{2cm} |}
		\rowcolor[HTML]{329A9D} 
		{\color[HTML]{FFFFFF} \textbf{Semana 2}} & {\color[HTML]{FFFFFF} \textbf{Día 1}} & {\color[HTML]{FFFFFF} \textbf{Día 2}} & {\color[HTML]{FFFFFF} \textbf{Día 3}} & {\color[HTML]{FFFFFF} \textbf{Día 4}}  & {\color[HTML]{FFFFFF} \textbf{Día 5}} \\ \hline
		Álvaro Fernández-Alonso Araluce & 30.3 & 30.3 & 30.4 & 30.4 & 30.4 \\ \hline
	\end{tabular}
\end{table}

\begin{table}[h]
	\centering
	\begin{tabular}{| p{2cm} | p{2cm} | p{2cm} | p{2cm} | p{2cm} | p{2cm} |}
		\rowcolor[HTML]{329A9D} 
		{\color[HTML]{FFFFFF} \textbf{Semana 3}} & {\color[HTML]{FFFFFF} \textbf{Día 1}} & {\color[HTML]{FFFFFF} \textbf{Día 2}} & {\color[HTML]{FFFFFF} \textbf{Día 3}} & {\color[HTML]{FFFFFF} \textbf{Día 4}}  & {\color[HTML]{FFFFFF} \textbf{Día 5}} \\ \hline
		Álvaro Fernández-Alonso Araluce & 30.5 & 30.6 30.7 & Testing &Testing & Testing \\ \hline
	\end{tabular}
\end{table}


\newpage
\section{Plan de entrega 13}

\subsection{Objetivos de la entrega}

El objetivo principal de esta entrega se va a centrar en permitir al usuario Padawan poder participar en misiones de desactivación de minas.

\subsection{Listado de HU a desarrollar}

\begin{table}[h]
	\centering
	\begin{tabular}{| p{2.3cm} | p{6.7cm} | p{2cm} |}
		\rowcolor[HTML]{329A9D} 
		{\color[HTML]{FFFFFF} \textbf{Identificador}} & {\color[HTML]{FFFFFF} \textbf{Historias de Usuario}} & {\color[HTML]{FFFFFF} \textbf{Estimación}}  \\ \hline
		HU.29 & Permitir al usuario Padawan participar en misiones de desactivación de minas & 11 \\ \hline
		HU.31 & Permitir al usuario Padawan solicitar vacaciones & 16 \\ \hline
		HU.32 & Permitir al usuario Padawan solicitar un préstamo & 13 \\ \hline
	\end{tabular}
\end{table}

\subsection{Descomposición en tareas de desarrollo}

A continuación incluimos un desglose de las HU en tareas de desarrollo junto con la estimación realizada de su duración.\\

\begin{table}[h]
	\centering
	\begin{tabular}{| p{2.3cm} | p{6.7cm} | p{2cm} |}
		\rowcolor[HTML]{329A9D} 
		{\color[HTML]{FFFFFF} \textbf{HU.29}} & {\color[HTML]{FFFFFF} \textbf{Permitir al usuario Padawan participar en misiones de desactivación de minas}} & {\color[HTML]{FFFFFF} \textbf{10}}  \\ \hline
		Tarea 29.1 & Permitir al usuario Padawan visualizar las misiones de mina creadas por el Maestro Guidoogway & 4 \\ \hline
		Tarea 29.2 & Permitir al usuario Padawan la compra de pistas por TdV & 2 \\ \hline
		Tarea 29.3 & Permitir al usuario Padawan la compra de pistas por Cartas de Privilegios & 3 \\ \hline
		Tarea 29.4 & Permitir al usuario Padawan introducir claves  & 2 \\ \hline
	\end{tabular}
\end{table}

\newpage

\begin{table}[h]
	\centering
	\begin{tabular}{| p{2.3cm} | p{6.7cm} | p{2cm} |}
		\rowcolor[HTML]{329A9D} 
		{\color[HTML]{FFFFFF} \textbf{HU.31}} & {\color[HTML]{FFFFFF} \textbf{Permitir al usuario Padawan solicitar vacaciones}} & {\color[HTML]{FFFFFF} \textbf{16}}  \\ \hline
		Tarea 31.1 & Permitir al usuario Padawan comprar vacaciones con TdV & 4 \\ \hline
		Tarea 31.2 & Permitir al usuario Padawan comprar vacaciones con Cartas de Privilegios & 4 \\ \hline
		Tarea 31.3 & No permitir al usuario Padawan comprar vacaciones más de una vez & 4 \\ \hline
		Tarea 31.4 & Permitir al sistema llenar las barras de comida y agua del usuario que esté de vacaciones & 4 \\ \hline
		Tarea 31.5 & Permitir al sistema congelar el reloj regresivo del usuario hasta que se le agoten las vacaciones & 4 \\ \hline
	\end{tabular}
\end{table}

\newpage

\begin{table}[h]
	\centering
	\begin{tabular}{| p{2.3cm} | p{6.7cm} | p{2cm} |}
		\rowcolor[HTML]{329A9D} 
		{\color[HTML]{FFFFFF} \textbf{HU.32}} & {\color[HTML]{FFFFFF} \textbf{Permitir al usuario Padawan solicitar un préstamo}} & {\color[HTML]{FFFFFF} \textbf{13}}  \\ \hline
		Tarea 32.1 & Permitir al Maestro Guidoogway establecer la comisión que se llevará el sistema por cada préstamo & 1 \\ \hline
		Tarea 32.2 & Permitir al usuario Padawan solicitar un TdV al sistema & 4 \\ \hline
		Tarea 32.3 & No permitir al usuario Padawan solicitar TdV si éste tiene más de 7 días de vida & 1 \\ \hline
		Tarea 32.4 & Permitir al sistema comprobar semanalmente cada deuda & 4 \\ \hline
		Tarea 32.5 & Permitir al usuario Padawan liquidar el total de la deuda & 1 \\ \hline
		Tarea 32.6 & Permitir al usuario Padawan liquidar la deuda por cuotas & 1 \\ \hline
		Tarea 32.5 & Impedir al usuario Padawan solicitar un préstamo teniendo otro en curso & 1 \\ \hline
	\end{tabular}
\end{table}

\subsection{Carga prevista en los desarrolladores}

A continuación mostramos la carga de trabajo prevista para cada uno de los desarrolladores en relación a las tareas definidas anteriormente:

\begin{table}[h]
	\centering
	\begin{tabular}{| p{3cm} | p{2cm} | p{2cm} | p{2cm} | p{2cm} |}
		\rowcolor[HTML]{329A9D} 
		{\color[HTML]{FFFFFF} \textbf{Desarrollador}} & {\color[HTML]{FFFFFF} \textbf{Velocidad inicial (días ideales)}} & {\color[HTML]{FFFFFF} \textbf{Dedicación (\% del tiempo)}} & {\color[HTML]{FFFFFF} \textbf{Carga de trabajo (días ideales)}} & {\color[HTML]{FFFFFF} \textbf{\#Tareas aceptadas}}  \\ \hline
		Álvaro Fernández-Alonso Araluce & 15 & 50\% & 15 & 14 \\ \hline
	\end{tabular}
\end{table}

\newpage

\subsection{Planificación temporal de las tareas}

\begin{table}[h]
	\centering
	\begin{tabular}{| p{2cm} | p{2cm} | p{2cm} | p{2cm} | p{2cm} | p{2cm} |}
		\rowcolor[HTML]{329A9D} 
		{\color[HTML]{FFFFFF} \textbf{Semana 1}} & {\color[HTML]{FFFFFF} \textbf{Día 1}} & {\color[HTML]{FFFFFF} \textbf{Día 2}} & {\color[HTML]{FFFFFF} \textbf{Día 3}} & {\color[HTML]{FFFFFF} \textbf{Día 4}}  & {\color[HTML]{FFFFFF} \textbf{Día 5}} \\ \hline
		Álvaro Fernández-Alonso Araluce & 29.1 & 29.2 & 29.3 29.4 & 29.4 31.1 & 31.1 31.2 \\ \hline
	\end{tabular}
\end{table}

\begin{table}[h]
	\centering
	\begin{tabular}{| p{2cm} | p{2cm} | p{2cm} | p{2cm} | p{2cm} | p{2cm} |}
		\rowcolor[HTML]{329A9D} 
		{\color[HTML]{FFFFFF} \textbf{Semana 2}} & {\color[HTML]{FFFFFF} \textbf{Día 1}} & {\color[HTML]{FFFFFF} \textbf{Día 2}} & {\color[HTML]{FFFFFF} \textbf{Día 3}} & {\color[HTML]{FFFFFF} \textbf{Día 4}}  & {\color[HTML]{FFFFFF} \textbf{Día 5}} \\ \hline
		Álvaro Fernández-Alonso Araluce & 31.2 & 31.3 & 31.3 & 31.4 & 31.4 31.5 \\ \hline
	\end{tabular}
\end{table}

\begin{table}[h]
	\centering
	\begin{tabular}{| p{2cm} | p{2cm} | p{2cm} | p{2cm} | p{2cm} | p{2cm} |}
		\rowcolor[HTML]{329A9D} 
		{\color[HTML]{FFFFFF} \textbf{Semana 3}} & {\color[HTML]{FFFFFF} \textbf{Día 1}} & {\color[HTML]{FFFFFF} \textbf{Día 2}} & {\color[HTML]{FFFFFF} \textbf{Día 3}} & {\color[HTML]{FFFFFF} \textbf{Día 4}}  & {\color[HTML]{FFFFFF} \textbf{Día 5}} \\ \hline
		Álvaro Fernández-Alonso Araluce & 31.5 32.1 32.2 & 32.2 32.3 32.4 & 32.4 32.5 & 32.6 32.7 & Testing \\ \hline
	\end{tabular}
\end{table}


\newpage
\section{Plan de entrega 14}

\subsection{Objetivos de la entrega}

El objetivo principal de esta entrega se va a centrar en permitir al usuario Padawan poder entregar retos de tipo Felicidad y en la gestión de tareas en segundo plano.

\subsection{Listado de HU a desarrollar}

\begin{table}[h]
	\centering
	\begin{tabular}{| p{2.3cm} | p{6.7cm} | p{2cm} |}
		\rowcolor[HTML]{329A9D} 
		{\color[HTML]{FFFFFF} \textbf{Identificador}} & {\color[HTML]{FFFFFF} \textbf{Historias de Usuario}} & {\color[HTML]{FFFFFF} \textbf{Estimación}}  \\ \hline
		HU.33 & Permitir al usuario Padawan interactuar con los proyectos de felicidad & 10 \\ \hline
		HU.34 & Permitir al sistema realizar tareas automáticas de gestión en segundo plano en un momento dado & 30 \\ \hline
	\end{tabular}
\end{table}

\subsection{Descomposición en tareas de desarrollo}

A continuación incluimos un desglose de las HU en tareas de desarrollo junto con la estimación realizada de su duración.\\

\begin{table}[h]
	\centering
	\begin{tabular}{| p{2.3cm} | p{6.7cm} | p{2cm} |}
		\rowcolor[HTML]{329A9D} 
		{\color[HTML]{FFFFFF} \textbf{HU.33}} & {\color[HTML]{FFFFFF} \textbf{Permitir al usuario Padawan interactuar con los proyectos de felicidad}} & {\color[HTML]{FFFFFF} \textbf{10}}  \\ \hline
		Tarea 33.1 & Permitir al usuario Padawan visualizar los retos publicados de este tipo por el Maestro Guidoogway & 4 \\ \hline
		Tarea 33.2 & Permitir al usuario Padawan entregar un fichero de propuesta de felicidad & 2 \\ \hline
		Tarea 33.3 & Permitir al usuario Padawan entregar un fichero de evidencias & 2 \\ \hline
		Tarea 33.4 & Imperdir al usuario Padawan entregue el fichero de evidencias si no han transcurrido los días especificados por el Maestro Guidoogway y la fecha de entrega de la propuesta de felicidad & 2 \\ \hline
	\end{tabular}
\end{table}

\newpage

\begin{table}[h]
	\centering
	\begin{tabular}{| p{2.3cm} | p{6.7cm} | p{2cm} |}
		\rowcolor[HTML]{329A9D} 
		{\color[HTML]{FFFFFF} \textbf{HU.34}} & {\color[HTML]{FFFFFF} \textbf{Permitir al sistema realizar tareas automáticas de gestión en segundo plano en un momento dado}} & {\color[HTML]{FFFFFF} \textbf{30}}  \\ \hline
		Tarea 34.1 & Permitir al sistema comprobar si un usuario ha subido de nivel & 4 \\ \hline
		Tarea 34.2 & Permitir al sistema enviar un email a cada usuario Padawan con el resultado de la operación automática de subida de nivel & 8 \\ \hline
		Tarea 34.3 & Permitir al sistema diariamente comprobar el número de Tweets y realizar el ingreso para los usuarios Padawans si se ha suerado el reto& 6 \\ \hline
		Tarea 34.4 & Permitir al sistema semanalmente comprobar automáticamente las sesiones deportivas de Runtastic de los usuarios Padawans e ingreaser el TdV estipulado en caso de superación  & 8 \\ \hline
		Tarea 34.5 & Permitir al sistema comprobar el estado de los préstamos y cobrar las cuotas semanalmente & 4 \\ \hline
	\end{tabular}
\end{table}

\newpage

\subsection{Carga prevista en los desarrolladores}

A continuación mostramos la carga de trabajo prevista para cada uno de los desarrolladores en relación a las tareas definidas anteriormente:

\begin{table}[h]
	\centering
	\begin{tabular}{| p{3cm} | p{2cm} | p{2cm} | p{2cm} | p{2cm} |}
		\rowcolor[HTML]{329A9D} 
		{\color[HTML]{FFFFFF} \textbf{Desarrollador}} & {\color[HTML]{FFFFFF} \textbf{Velocidad inicial (días ideales)}} & {\color[HTML]{FFFFFF} \textbf{Dedicación (\% del tiempo)}} & {\color[HTML]{FFFFFF} \textbf{Carga de trabajo (días ideales)}} & {\color[HTML]{FFFFFF} \textbf{\#Tareas aceptadas}}  \\ \hline
		Álvaro Fernández-Alonso Araluce & 15 & 50\% & 15 & 14 \\ \hline
	\end{tabular}
\end{table}

\subsection{Planificación temporal de las tareas}

\begin{table}[h]
	\centering
	\begin{tabular}{| p{2cm} | p{2cm} | p{2cm} | p{2cm} | p{2cm} | p{2cm} |}
		\rowcolor[HTML]{329A9D} 
		{\color[HTML]{FFFFFF} \textbf{Semana 1}} & {\color[HTML]{FFFFFF} \textbf{Día 1}} & {\color[HTML]{FFFFFF} \textbf{Día 2}} & {\color[HTML]{FFFFFF} \textbf{Día 3}} & {\color[HTML]{FFFFFF} \textbf{Día 4}}  & {\color[HTML]{FFFFFF} \textbf{Día 5}} \\ \hline
		Álvaro Fernández-Alonso Araluce & 33.1 & 33.2 33.3 & 33.4 34.1 & 34.1 34.2 & 34.2 \\ \hline
	\end{tabular}
\end{table}

\begin{table}[h]
	\centering
	\begin{tabular}{| p{2cm} | p{2cm} | p{2cm} | p{2cm} | p{2cm} | p{2cm} |}
		\rowcolor[HTML]{329A9D} 
		{\color[HTML]{FFFFFF} \textbf{Semana 2}} & {\color[HTML]{FFFFFF} \textbf{Día 1}} & {\color[HTML]{FFFFFF} \textbf{Día 2}} & {\color[HTML]{FFFFFF} \textbf{Día 3}} & {\color[HTML]{FFFFFF} \textbf{Día 4}}  & {\color[HTML]{FFFFFF} \textbf{Día 5}} \\ \hline
		Álvaro Fernández-Alonso Araluce & 34.2 & 34.3 & 34.3 & 34.4 & 34.5 \\ \hline
	\end{tabular}
\end{table}

\begin{table}[h]
	\centering
	\begin{tabular}{| p{2cm} | p{2cm} | p{2cm} | p{2cm} | p{2cm} | p{2cm} |}
		\rowcolor[HTML]{329A9D} 
		{\color[HTML]{FFFFFF} \textbf{Semana 3}} & {\color[HTML]{FFFFFF} \textbf{Día 1}} & {\color[HTML]{FFFFFF} \textbf{Día 2}} & {\color[HTML]{FFFFFF} \textbf{Día 3}} & {\color[HTML]{FFFFFF} \textbf{Día 4}}  & {\color[HTML]{FFFFFF} \textbf{Día 5}} \\ \hline
		Álvaro Fernández-Alonso Araluce & Testing & Testing & Testing & Testing & Testing \\ \hline
	\end{tabular}
\end{table}

