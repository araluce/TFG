\setcounter{chapter}{4}
\setcounter{section}{0}
\setcounter{subsection}{0}
\chapter{Plan de entregas}

\section{Breve descripción del alcance del sistema}
El desarrollo del proyecto \textbf{SinTime} consiste en la implementación de una aplicación web cuya implantación se realizará en el marco universitario con el fin de gamificar la experiencia del alumnado mediante la inmersión de ellos en un universo ficticio.\\

Los objetivos más importantes que debe cumplir la aplicación serán:

\begin{itemize}
	\item Debe permitir la administración de todos los usuarios con independencia de su rol.
	\item Debe facilitar al docente la creación de todo tipo de retos para el alumnado.
	\item Debe facilitar al docente la información de los movimientos de un alumno dentro de la app.
	\item Debe permitir al alumno ser evaluado por el docente.
	\item Debe permitir al alumno ser evaluado por el sistema.
	\item El alumno debe poder obtener su calificación.
\end{itemize}

\section{Listado inicial de Historias de Usuario}

A continuación se muestran las Historias de Usuario obtenidas durante las reuniones de planificación y entregas del producto realizadas entre el cliente y el equipo de desarrollo. La lista se divide en 4 partes: un identificador, una descripción de la historia, una estimación en días ideales y una prioridad. La prioridad se medirá por el cliente en un rango de 0 a 100, siendo 100 la prioridad más alta.\\

\begin{table}[h]
	\centering
	\begin{tabular}{| p{2.3cm} | p{5.1cm} | p{2cm} | p{1.6cm} |}
		\rowcolor[HTML]{329A9D} 
		{\color[HTML]{FFFFFF} \textbf{Identificador}} & {\color[HTML]{FFFFFF} \textbf{Historias de Usuario}} & {\color[HTML]{FFFFFF} \textbf{Estimación}} & {\color[HTML]{FFFFFF} \textbf{Prioridad}} \\ \hline
		HU.1 & Ejemplo & 1 & 100 \\         
		\hline              
	\end{tabular}
\end{table}

\section{Cálculo de la velocidad del equipo}

El equipo inicial está formado por un programador con una dedicación al proyecto del 50\%. La duración de cada Sprint será, como ya se ha comentado, de 3 semanas de las cuales dos semanas se destinarán a desarrollo y una a testing.\\

La estimación está basada en \textit{días ideales}. Un día ideal se compone de 4 horas de trabajo por lo que el resultado en horas de cada  sprint será:\\

\[
1 \textrm{Sprint} = 3 \textrm{Semanas} \quad \textrm{y} \quad 1 \textrm{Semana} = 5 \textrm{días ideales}
\]

\[
1 \textrm{día ideal} = 4 \textrm{horas reales} \quad \textrm{entonces} \quad 1 \textrm{Spint} = 60 \textrm{horas reales}
\]

\section{Descripción de las entregas}

\textbf{Esfuerzo total del proyecto} = X PH
\textbf{Duración del proyecto} = X meses
\textbf{Velocidad del equipo} = X a Y PH

En base a las estimaciones de velocidad del equipo y al esfuerzo necesario para el desarrollo se ha decidido realizar X entregas. La semana laboral dedicada al proyecto se dividirá en 5 días (Lunes a Viernes) con una duración de 4 horas diarias que suman un total de 20 horas semanales. El desarrollo del proyecto comenzará el día \textbf{1 de Octubre de 2017}.\\


El plan de entregas es el siguiente:

\begin{table}[h]
	\centering
	\begin{tabular}{| p{1.4cm} | p{6.7cm} | p{3.2cm} |}
		\rowcolor[HTML]{329A9D} 
		{\color[HTML]{FFFFFF} \textbf{Entrega}} & {\color[HTML]{FFFFFF} \textbf{Objetivo}} & {\color[HTML]{FFFFFF} \textbf{Fecha de entrega}} \\    
		Sprint 1 & Ejemplo largo Ejemplo largo Ejemplo largo Ejemplo largo  & 20 Octubre 2017 \\ \hline
		Sprint 2 & Ejemplo largo Ejemplo largo Ejemplo largo Ejemplo largoEjemplo largo Ejemplo largo Ejemplo largo Ejemplo largo  & 10 Noviembre 2017 \\ 
		\hline              
	\end{tabular}
\end{table}

\section{Lista Inicial del Producto}

La lista del producto, con las historias que se usaran en el inicio del desarrollo es la siguiente:\\

\begin{table}[h]
	\centering
	\begin{tabular}{| p{2.3cm} | p{5.1cm} | p{2cm} | p{1.6cm} |}
		\rowcolor[HTML]{329A9D} 
		{\color[HTML]{FFFFFF} \textbf{Identificador}} & {\color[HTML]{FFFFFF} \textbf{Historias de Usuario}} & {\color[HTML]{FFFFFF} \textbf{Estimación}} & {\color[HTML]{FFFFFF} \textbf{Prioridad}} \\ \hline
		HU.1 & Ejemplo & 1 & 100 \\         
		\hline              
	\end{tabular}
\end{table}