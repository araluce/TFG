\setcounter{chapter}{4}
\setcounter{section}{0}
\setcounter{subsection}{0}
\chapter{Plan de entregas}

\section{Breve descripción del alcance del sistema}
El desarrollo del proyecto \textbf{SinTime} consiste en la implementación de una aplicación web cuya implantación se realizará en el marco universitario con el fin de gamificar la experiencia del alumnado mediante la inmersión de ellos en un universo ficticio.\\

Los objetivos más importantes que debe cumplir la aplicación serán:

\begin{itemize}
	\item Debe permitir la administración de todos los usuarios con independencia de su rol.
	\item Debe facilitar al docente la creación de todo tipo de retos para el alumnado.
	\item Debe facilitar al docente la información de los movimientos de un alumno dentro de la app.
	\item Debe permitir al alumno ser evaluado por el docente.
	\item Debe permitir al alumno ser evaluado por el sistema.
	\item El alumno debe poder obtener su calificación.
\end{itemize}

\section{Listado inicial de Historias de Usuario}

A continuación se muestran las Historias de Usuario obtenidas durante las reuniones de planificación y entregas del producto realizadas entre el cliente y el equipo de desarrollo. La lista se divide en 4 partes: un identificador, una descripción de la historia, una estimación en días ideales y una prioridad. La prioridad se medirá por el cliente en un rango de 0 a 100, siendo 100 la prioridad más alta.\\

\begin{table}[h]
	\begin{tabular}{llll}
		\rowcolor[HTML]{329A9D} 
		{\color[HTML]{FFFFFF} \textbf{Identificador}} & {\color[HTML]{FFFFFF} \textbf{Historias de Usuario}} & {\color[HTML]{FFFFFF} \textbf{Estimación}} & {\color[HTML]{FFFFFF} \textbf{Prioridad}} \\
		HU.1 & Ejemplo & 1 & 100 \\                                      
	\end{tabular}
\end{table}