\section{Plan de entrega 13}

\subsection{Objetivos de la entrega}

El objetivo principal de esta entrega se va a centrar en permitir al usuario Padawan poder participar en misiones de desactivación de minas.

\subsection{Listado de HU a desarrollar}

\begin{table}[h]
	\centering
	\begin{tabular}{| p{2.3cm} | p{6.7cm} | p{2cm} |}
		\rowcolor[HTML]{329A9D} 
		{\color[HTML]{FFFFFF} \textbf{Identificador}} & {\color[HTML]{FFFFFF} \textbf{Historias de Usuario}} & {\color[HTML]{FFFFFF} \textbf{Estimación}}  \\ \hline
		HU.29 & Permitir al usuario Padawan participar en misiones de desactivación de minas & 11 \\ \hline
		HU.31 & Permitir al usuario Padawan solicitar vacaciones & 16 \\ \hline
		HU.32 & Permitir al usuario Padawan solicitar un préstamo & 13 \\ \hline
	\end{tabular}
\end{table}

\subsection{Descomposición en tareas de desarrollo}

A continuación incluimos un desglose de las HU en tareas de desarrollo junto con la estimación realizada de su duración.\\

\begin{table}[h]
	\centering
	\begin{tabular}{| p{2.3cm} | p{6.7cm} | p{2cm} |}
		\rowcolor[HTML]{329A9D} 
		{\color[HTML]{FFFFFF} \textbf{HU.29}} & {\color[HTML]{FFFFFF} \textbf{Permitir al usuario Padawan participar en misiones de desactivación de minas}} & {\color[HTML]{FFFFFF} \textbf{10}}  \\ \hline
		Tarea 29.1 & Permitir al usuario Padawan visualizar las misiones de mina creadas por el Maestro Guidoogway & 4 \\ \hline
		Tarea 29.2 & Permitir al usuario Padawan la compra de pistas por TdV & 2 \\ \hline
		Tarea 29.3 & Permitir al usuario Padawan la compra de pistas por Cartas de Privilegios & 3 \\ \hline
		Tarea 29.4 & Permitir al usuario Padawan introducir claves  & 2 \\ \hline
	\end{tabular}
\end{table}

\newpage

\begin{table}[h]
	\centering
	\begin{tabular}{| p{2.3cm} | p{6.7cm} | p{2cm} |}
		\rowcolor[HTML]{329A9D} 
		{\color[HTML]{FFFFFF} \textbf{HU.31}} & {\color[HTML]{FFFFFF} \textbf{Permitir al usuario Padawan solicitar vacaciones}} & {\color[HTML]{FFFFFF} \textbf{16}}  \\ \hline
		Tarea 31.1 & Permitir al usuario Padawan comprar vacaciones con TdV & 4 \\ \hline
		Tarea 31.2 & Permitir al usuario Padawan comprar vacaciones con Cartas de Privilegios & 4 \\ \hline
		Tarea 31.3 & No permitir al usuario Padawan comprar vacaciones más de una vez & 4 \\ \hline
		Tarea 31.4 & Permitir al sistema llenar las barras de comida y agua del usuario que esté de vacaciones & 4 \\ \hline
		Tarea 31.5 & Permitir al sistema congelar el reloj regresivo del usuario hasta que se le agoten las vacaciones & 4 \\ \hline
	\end{tabular}
\end{table}

\newpage

\begin{table}[h]
	\centering
	\begin{tabular}{| p{2.3cm} | p{6.7cm} | p{2cm} |}
		\rowcolor[HTML]{329A9D} 
		{\color[HTML]{FFFFFF} \textbf{HU.32}} & {\color[HTML]{FFFFFF} \textbf{Permitir al usuario Padawan solicitar un préstamo}} & {\color[HTML]{FFFFFF} \textbf{13}}  \\ \hline
		Tarea 32.1 & Permitir al Maestro Guidoogway establecer la comisión que se llevará el sistema por cada préstamo & 1 \\ \hline
		Tarea 32.2 & Permitir al usuario Padawan solicitar un TdV al sistema & 4 \\ \hline
		Tarea 32.3 & No permitir al usuario Padawan solicitar TdV si éste tiene más de 7 días de vida & 1 \\ \hline
		Tarea 32.4 & Permitir al sistema comprobar semanalmente cada deuda & 4 \\ \hline
		Tarea 32.5 & Permitir al usuario Padawan liquidar el total de la deuda & 1 \\ \hline
		Tarea 32.6 & Permitir al usuario Padawan liquidar la deuda por cuotas & 1 \\ \hline
		Tarea 32.5 & Impedir al usuario Padawan solicitar un préstamo teniendo otro en curso & 1 \\ \hline
	\end{tabular}
\end{table}

\subsection{Carga prevista en los desarrolladores}

A continuación mostramos la carga de trabajo prevista para cada uno de los desarrolladores en relación a las tareas definidas anteriormente:

\begin{table}[h]
	\centering
	\begin{tabular}{| p{3cm} | p{2cm} | p{2cm} | p{2cm} | p{2cm} |}
		\rowcolor[HTML]{329A9D} 
		{\color[HTML]{FFFFFF} \textbf{Desarrollador}} & {\color[HTML]{FFFFFF} \textbf{Velocidad inicial (días ideales)}} & {\color[HTML]{FFFFFF} \textbf{Dedicación (\% del tiempo)}} & {\color[HTML]{FFFFFF} \textbf{Carga de trabajo (días ideales)}} & {\color[HTML]{FFFFFF} \textbf{\#Tareas aceptadas}}  \\ \hline
		Álvaro Fernández-Alonso Araluce & 15 & 50\% & 15 & 14 \\ \hline
	\end{tabular}
\end{table}

\newpage

\subsection{Planificación temporal de las tareas}

\begin{table}[h]
	\centering
	\begin{tabular}{| p{2cm} | p{2cm} | p{2cm} | p{2cm} | p{2cm} | p{2cm} |}
		\rowcolor[HTML]{329A9D} 
		{\color[HTML]{FFFFFF} \textbf{Semana 1}} & {\color[HTML]{FFFFFF} \textbf{Día 1}} & {\color[HTML]{FFFFFF} \textbf{Día 2}} & {\color[HTML]{FFFFFF} \textbf{Día 3}} & {\color[HTML]{FFFFFF} \textbf{Día 4}}  & {\color[HTML]{FFFFFF} \textbf{Día 5}} \\ \hline
		Álvaro Fernández-Alonso Araluce & 29.1 & 29.2 & 29.3 29.4 & 29.4 31.1 & 31.1 31.2 \\ \hline
	\end{tabular}
\end{table}

\begin{table}[h]
	\centering
	\begin{tabular}{| p{2cm} | p{2cm} | p{2cm} | p{2cm} | p{2cm} | p{2cm} |}
		\rowcolor[HTML]{329A9D} 
		{\color[HTML]{FFFFFF} \textbf{Semana 2}} & {\color[HTML]{FFFFFF} \textbf{Día 1}} & {\color[HTML]{FFFFFF} \textbf{Día 2}} & {\color[HTML]{FFFFFF} \textbf{Día 3}} & {\color[HTML]{FFFFFF} \textbf{Día 4}}  & {\color[HTML]{FFFFFF} \textbf{Día 5}} \\ \hline
		Álvaro Fernández-Alonso Araluce & 31.2 & 31.3 & 31.3 & 31.4 & 31.4 31.5 \\ \hline
	\end{tabular}
\end{table}

\begin{table}[h]
	\centering
	\begin{tabular}{| p{2cm} | p{2cm} | p{2cm} | p{2cm} | p{2cm} | p{2cm} |}
		\rowcolor[HTML]{329A9D} 
		{\color[HTML]{FFFFFF} \textbf{Semana 3}} & {\color[HTML]{FFFFFF} \textbf{Día 1}} & {\color[HTML]{FFFFFF} \textbf{Día 2}} & {\color[HTML]{FFFFFF} \textbf{Día 3}} & {\color[HTML]{FFFFFF} \textbf{Día 4}}  & {\color[HTML]{FFFFFF} \textbf{Día 5}} \\ \hline
		Álvaro Fernández-Alonso Araluce & 31.5 32.1 32.2 & 32.2 32.3 32.4 & 32.4 32.5 & 32.6 32.7 & Testing \\ \hline
	\end{tabular}
\end{table}

