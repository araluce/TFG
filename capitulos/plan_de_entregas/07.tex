\section{Plan de entrega 7}

\subsection{Objetivos de la entrega}

El objetivo principal de esta entrega se va a centrar en permitir al usuario Padawan poder realizar retos de tipo Comida.

\subsection{Listado de HU a desarrollar}

\begin{table}[h]
	\centering
	\begin{tabular}{| p{2.3cm} | p{6.7cm} | p{2cm} |}
		\rowcolor[HTML]{329A9D} 
		{\color[HTML]{FFFFFF} \textbf{Identificador}} & {\color[HTML]{FFFFFF} \textbf{Historias de Usuario}} & {\color[HTML]{FFFFFF} \textbf{Estimación}}  \\ \hline
		HU.22 & Permitir al usuario Padawan realizar retos de tipo Comida & 40 \\ \hline
	\end{tabular}
\end{table}

\newpage

\subsection{Descomposición en tareas de desarrollo}

A continuación incluimos un desglose de las HU en tareas de desarrollo junto con la estimación realizada de su duración.\\

\begin{table}[h]
	\centering
	\begin{tabular}{| p{2.3cm} | p{6.7cm} | p{2cm} |}
		\rowcolor[HTML]{329A9D} 
		{\color[HTML]{FFFFFF} \textbf{HU.22}} & {\color[HTML]{FFFFFF} \textbf{Permitir al usuario Padawan realizar retos de tipo Comida}} & {\color[HTML]{FFFFFF} \textbf{40}}  \\ \hline
		Tarea 22.1 & Permitir al usuario Padawan visualizar los retos de este tipo publicados por el Guardián del Tiempo & 4 \\ \hline
		Tarea 22.2 & Permitir a los usuarios Padawans comprar un reto individual & 4 \\ \hline
		Tarea 22.3 & Permitir a los usuarios Padawans comprar retos por clanes & 4 \\ \hline
		Tarea 22.4 & Permitir al sistema cobrar las compras del usuario & 4 \\ \hline
		Tarea 22.5 & Permitir al usuario Padawan entregar el reto a través de la plataforma & 4 \\ \hline
		Tarea 22.6 & Permitir al sistema mostrar una barra de progreso a modo de estómago & 6 \\ \hline
		Tarea 22.7 & Permitir usuario Padawan ver el estado de su entrega en cada momento & 4 \\ \hline
		Tarea 22.8 & Permitir al usuario Padawan acceder a su entrega & 2 \\ \hline
		Tarea 22.9 & Permitir al usuario Padawan ver el Feedback del Maestro Guidoogway & 4 \\ \hline
		Tarea 22.10 & Permitir al sistema ingresar el TdV asociado al Feedback del usuario Padawan & 4 \\
		 \hline
		Tarea 22.11 & Permitir al sistema ingresar el TdV asociado al Feedback del clan Padawan & 4 \\ \hline
	\end{tabular}
\end{table}

\newpage

\subsection{Carga prevista en los desarrolladores}

A continuación mostramos la carga de trabajo prevista para cada uno de los desarrolladores en relación a las tareas definidas anteriormente:

\begin{table}[h]
	\centering
	\begin{tabular}{| p{3cm} | p{2cm} | p{2cm} | p{2cm} | p{2cm} |}
		\rowcolor[HTML]{329A9D} 
		{\color[HTML]{FFFFFF} \textbf{Desarrollador}} & {\color[HTML]{FFFFFF} \textbf{Velocidad inicial (días ideales)}} & {\color[HTML]{FFFFFF} \textbf{Dedicación (\% del tiempo)}} & {\color[HTML]{FFFFFF} \textbf{Carga de trabajo (días ideales)}} & {\color[HTML]{FFFFFF} \textbf{\#Tareas aceptadas}}  \\ \hline
		Álvaro Fernández-Alonso Araluce & 15 & 50\% & 15 & 11 \\ \hline
	\end{tabular}
\end{table}


\subsection{Planificación temporal de las tareas}

\begin{table}[h]
	\centering
	\begin{tabular}{| p{2cm} | p{2cm} | p{2cm} | p{2cm} | p{2cm} | p{2cm} |}
		\rowcolor[HTML]{329A9D} 
		 {\color[HTML]{FFFFFF} \textbf{Semana 1}} & {\color[HTML]{FFFFFF} \textbf{Día 1}} & {\color[HTML]{FFFFFF} \textbf{Día 2}} & {\color[HTML]{FFFFFF} \textbf{Día 3}} & {\color[HTML]{FFFFFF} \textbf{Día 4}}  & {\color[HTML]{FFFFFF} \textbf{Día 5}} \\ \hline
		Álvaro Fernández-Alonso Araluce & 22.1 & 22.2 & 22.3 & 22.4 & 22.5 \\ \hline
	\end{tabular}
\end{table}

\begin{table}[h]
	\centering
	\begin{tabular}{| p{2cm} | p{2cm} | p{2cm} | p{2cm} | p{2cm} | p{2cm} |}
		\rowcolor[HTML]{329A9D} 
		{\color[HTML]{FFFFFF} \textbf{Semana 2}} & {\color[HTML]{FFFFFF} \textbf{Día 1}} & {\color[HTML]{FFFFFF} \textbf{Día 2}} & {\color[HTML]{FFFFFF} \textbf{Día 3}} & {\color[HTML]{FFFFFF} \textbf{Día 4}}  & {\color[HTML]{FFFFFF} \textbf{Día 5}} \\ \hline
		Álvaro Fernández-Alonso Araluce & 22.6 & 22.6 22.7 & 22.7 22.8 & 22.9 & 22.10 \\ \hline
	\end{tabular}
\end{table}

\begin{table}[h]
	\centering
	\begin{tabular}{| p{2cm} | p{2cm} | p{2cm} | p{2cm} | p{2cm} | p{2cm} |}
		\rowcolor[HTML]{329A9D} 
		{\color[HTML]{FFFFFF} \textbf{Semana 3}} & {\color[HTML]{FFFFFF} \textbf{Día 1}} & {\color[HTML]{FFFFFF} \textbf{Día 2}} & {\color[HTML]{FFFFFF} \textbf{Día 3}} & {\color[HTML]{FFFFFF} \textbf{Día 4}}  & {\color[HTML]{FFFFFF} \textbf{Día 5}} \\ \hline
		Álvaro Fernández-Alonso Araluce & 22.11 & Testing & Testing & Testing & Testing \\ \hline
	\end{tabular}
\end{table}

