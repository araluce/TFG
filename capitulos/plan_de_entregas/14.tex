\section{Plan de entrega 14}

\subsection{Objetivos de la entrega}

El objetivo principal de esta entrega se va a centrar en permitir al usuario Padawan poder entregar retos de tipo Felicidad y en la gestión de tareas en segundo plano.

\subsection{Listado de HU a desarrollar}

\begin{table}[h]
	\centering
	\begin{tabular}{| p{2.3cm} | p{6.7cm} | p{2cm} |}
		\rowcolor[HTML]{329A9D} 
		{\color[HTML]{FFFFFF} \textbf{Identificador}} & {\color[HTML]{FFFFFF} \textbf{Historias de Usuario}} & {\color[HTML]{FFFFFF} \textbf{Estimación}}  \\ \hline
		HU.33 & Permitir al usuario Padawan interactuar con los proyectos de felicidad & 10 \\ \hline
		HU.34 & Permitir al sistema realizar tareas automáticas de gestión en segundo plano en un momento dado & 30 \\ \hline
	\end{tabular}
\end{table}

\subsection{Descomposición en tareas de desarrollo}

A continuación incluimos un desglose de las HU en tareas de desarrollo junto con la estimación realizada de su duración.\\

\begin{table}[h]
	\centering
	\begin{tabular}{| p{2.3cm} | p{6.7cm} | p{2cm} |}
		\rowcolor[HTML]{329A9D} 
		{\color[HTML]{FFFFFF} \textbf{HU.33}} & {\color[HTML]{FFFFFF} \textbf{Permitir al usuario Padawan interactuar con los proyectos de felicidad}} & {\color[HTML]{FFFFFF} \textbf{10}}  \\ \hline
		Tarea 33.1 & Permitir al usuario Padawan visualizar los retos publicados de este tipo por el Maestro Guidoogway & 4 \\ \hline
		Tarea 33.2 & Permitir al usuario Padawan entregar un fichero de propuesta de felicidad & 2 \\ \hline
		Tarea 33.3 & Permitir al usuario Padawan entregar un fichero de evidencias & 2 \\ \hline
		Tarea 33.4 & Imperdir al usuario Padawan entregue el fichero de evidencias si no han transcurrido los días especificados por el Maestro Guidoogway y la fecha de entrega de la propuesta de felicidad & 2 \\ \hline
	\end{tabular}
\end{table}

\newpage

\begin{table}[h]
	\centering
	\begin{tabular}{| p{2.3cm} | p{6.7cm} | p{2cm} |}
		\rowcolor[HTML]{329A9D} 
		{\color[HTML]{FFFFFF} \textbf{HU.34}} & {\color[HTML]{FFFFFF} \textbf{Permitir al sistema realizar tareas automáticas de gestión en segundo plano en un momento dado}} & {\color[HTML]{FFFFFF} \textbf{30}}  \\ \hline
		Tarea 34.1 & Permitir al sistema comprobar si un usuario ha subido de nivel & 4 \\ \hline
		Tarea 34.2 & Permitir al sistema enviar un email a cada usuario Padawan con el resultado de la operación automática de subida de nivel & 8 \\ \hline
		Tarea 34.3 & Permitir al sistema diariamente comprobar el número de Tweets y realizar el ingreso para los usuarios Padawans si se ha suerado el reto& 6 \\ \hline
		Tarea 34.4 & Permitir al sistema semanalmente comprobar automáticamente las sesiones deportivas de Runtastic de los usuarios Padawans e ingreaser el TdV estipulado en caso de superación  & 8 \\ \hline
		Tarea 34.5 & Permitir al sistema comprobar el estado de los préstamos y cobrar las cuotas semanalmente & 4 \\ \hline
	\end{tabular}
\end{table}

\newpage

\subsection{Carga prevista en los desarrolladores}

A continuación mostramos la carga de trabajo prevista para cada uno de los desarrolladores en relación a las tareas definidas anteriormente:

\begin{table}[h]
	\centering
	\begin{tabular}{| p{3cm} | p{2cm} | p{2cm} | p{2cm} | p{2cm} |}
		\rowcolor[HTML]{329A9D} 
		{\color[HTML]{FFFFFF} \textbf{Desarrollador}} & {\color[HTML]{FFFFFF} \textbf{Velocidad inicial (días ideales)}} & {\color[HTML]{FFFFFF} \textbf{Dedicación (\% del tiempo)}} & {\color[HTML]{FFFFFF} \textbf{Carga de trabajo (días ideales)}} & {\color[HTML]{FFFFFF} \textbf{\#Tareas aceptadas}}  \\ \hline
		Álvaro Fernández-Alonso Araluce & 15 & 50\% & 15 & 14 \\ \hline
	\end{tabular}
\end{table}

\subsection{Planificación temporal de las tareas}

\begin{table}[h]
	\centering
	\begin{tabular}{| p{2cm} | p{2cm} | p{2cm} | p{2cm} | p{2cm} | p{2cm} |}
		\rowcolor[HTML]{329A9D} 
		{\color[HTML]{FFFFFF} \textbf{Semana 1}} & {\color[HTML]{FFFFFF} \textbf{Día 1}} & {\color[HTML]{FFFFFF} \textbf{Día 2}} & {\color[HTML]{FFFFFF} \textbf{Día 3}} & {\color[HTML]{FFFFFF} \textbf{Día 4}}  & {\color[HTML]{FFFFFF} \textbf{Día 5}} \\ \hline
		Álvaro Fernández-Alonso Araluce & 33.1 & 33.2 33.3 & 33.4 34.1 & 34.1 34.2 & 34.2 \\ \hline
	\end{tabular}
\end{table}

\begin{table}[h]
	\centering
	\begin{tabular}{| p{2cm} | p{2cm} | p{2cm} | p{2cm} | p{2cm} | p{2cm} |}
		\rowcolor[HTML]{329A9D} 
		{\color[HTML]{FFFFFF} \textbf{Semana 2}} & {\color[HTML]{FFFFFF} \textbf{Día 1}} & {\color[HTML]{FFFFFF} \textbf{Día 2}} & {\color[HTML]{FFFFFF} \textbf{Día 3}} & {\color[HTML]{FFFFFF} \textbf{Día 4}}  & {\color[HTML]{FFFFFF} \textbf{Día 5}} \\ \hline
		Álvaro Fernández-Alonso Araluce & 34.2 & 34.3 & 34.3 & 34.4 & 34.5 \\ \hline
	\end{tabular}
\end{table}

\begin{table}[h]
	\centering
	\begin{tabular}{| p{2cm} | p{2cm} | p{2cm} | p{2cm} | p{2cm} | p{2cm} |}
		\rowcolor[HTML]{329A9D} 
		{\color[HTML]{FFFFFF} \textbf{Semana 3}} & {\color[HTML]{FFFFFF} \textbf{Día 1}} & {\color[HTML]{FFFFFF} \textbf{Día 2}} & {\color[HTML]{FFFFFF} \textbf{Día 3}} & {\color[HTML]{FFFFFF} \textbf{Día 4}}  & {\color[HTML]{FFFFFF} \textbf{Día 5}} \\ \hline
		Álvaro Fernández-Alonso Araluce & Testing & Testing & Testing & Testing & Testing \\ \hline
	\end{tabular}
\end{table}

