\section{Plan de entrega 11}

\subsection{Objetivos de la entrega}

El objetivo principal de esta entrega se va a centrar en permitir al usuario Padawan poder realizar donaciones y crear las Cartas de Privilegios.

\subsection{Listado de HU a desarrollar}

\begin{table}[h]
	\centering
	\begin{tabular}{| p{2.3cm} | p{6.7cm} | p{2cm} |}
		\rowcolor[HTML]{329A9D} 
		{\color[HTML]{FFFFFF} \textbf{Identificador}} & {\color[HTML]{FFFFFF} \textbf{Historias de Usuario}} & {\color[HTML]{FFFFFF} \textbf{Estimación}}  \\ \hline
		HU.27 & Permitir al usuario Padawan realizar donaciones & 12 \\ \hline
		HU.28 & Crear cartas de privilegios con un comportamiento específico por cada una de ellas & 28 \\ \hline
	\end{tabular}
\end{table}

\newpage

\subsection{Descomposición en tareas de desarrollo}

A continuación incluimos un desglose de las HU en tareas de desarrollo junto con la estimación realizada de su duración.\\

\begin{table}[h]
	\centering
	\begin{tabular}{| p{2.3cm} | p{6.7cm} | p{2cm} |}
		\rowcolor[HTML]{329A9D} 
		{\color[HTML]{FFFFFF} \textbf{HU.27}} & {\color[HTML]{FFFFFF} \textbf{Permitir al usuario Padawan realizar donaciones}} & {\color[HTML]{FFFFFF} \textbf{12}}  \\ \hline
		Tarea 27.1 & Permitir al Maestro Guidoogway seleccionar un usuario distinto al suyo para donar una cantidad de TdV & 8 \\ \hline
		Tarea 27.2 & No permitir que un usuario Padawan pueda donar a otro usuario Padawan que tenga más de 7 días de vida & 2 \\ \hline
		Tarea 27.3 & No permitir que un usuario Padawan pueda donarse a sí mismo & 1 \\ \hline
		Tarea 27.4 & No permitir que un usuario Padawan donar más de una vez a otro usuario Padawan & 1 \\ \hline
	\end{tabular}
\end{table}

\begin{table}[h]
	\centering
	\begin{tabular}{| p{2.3cm} | p{6.7cm} | p{2cm} |}
		\rowcolor[HTML]{329A9D} 
		{\color[HTML]{FFFFFF} \textbf{HU.28}} & {\color[HTML]{FFFFFF} \textbf{Crear Cartas de Privilegios con un comportamiento específico por cada una de ellas}} & {\color[HTML]{FFFFFF} \textbf{28}}  \\ \hline
		Tarea 28.1 & Permitir al Maestro Guidoogway gestionar las Cartas de Privilegios & 4 \\ \hline
		Tarea 28.2 & Crear una sección donde cualquier usuario pueda visualizar esas cartas & 4 \\ \hline
		Tarea 28.3 & Permitir al usuario Padawan comprar Cartas de Privilegios & 4 \\ \hline
		Tarea 28.4 & Permitir al sistema comprobar cada una de las cartas para aplicar diferentes acciones en diferentes espacios de tiempo & 16 \\ \hline
	\end{tabular}
\end{table}

\newpage

\subsection{Carga prevista en los desarrolladores}

A continuación mostramos la carga de trabajo prevista para cada uno de los desarrolladores en relación a las tareas definidas anteriormente:

\begin{table}[h]
	\centering
	\begin{tabular}{| p{3cm} | p{2cm} | p{2cm} | p{2cm} | p{2cm} |}
		\rowcolor[HTML]{329A9D} 
		{\color[HTML]{FFFFFF} \textbf{Desarrollador}} & {\color[HTML]{FFFFFF} \textbf{Velocidad inicial (días ideales)}} & {\color[HTML]{FFFFFF} \textbf{Dedicación (\% del tiempo)}} & {\color[HTML]{FFFFFF} \textbf{Carga de trabajo (días ideales)}} & {\color[HTML]{FFFFFF} \textbf{\#Tareas aceptadas}}  \\ \hline
		Álvaro Fernández-Alonso Araluce & 15 & 50\% & 15 & 8 \\ \hline
	\end{tabular}
\end{table}


\subsection{Planificación temporal de las tareas}

\begin{table}[h]
	\centering
	\begin{tabular}{| p{2cm} | p{2cm} | p{2cm} | p{2cm} | p{2cm} | p{2cm} |}
		\rowcolor[HTML]{329A9D} 
		{\color[HTML]{FFFFFF} \textbf{Semana 1}} & {\color[HTML]{FFFFFF} \textbf{Día 1}} & {\color[HTML]{FFFFFF} \textbf{Día 2}} & {\color[HTML]{FFFFFF} \textbf{Día 3}} & {\color[HTML]{FFFFFF} \textbf{Día 4}}  & {\color[HTML]{FFFFFF} \textbf{Día 5}} \\ \hline
		Álvaro Fernández-Alonso Araluce & 27.1 & 27.1 & 27.2 27.3 27.4 & 28.1 & 28.2 \\ \hline
	\end{tabular}
\end{table}

\begin{table}[h]
	\centering
	\begin{tabular}{| p{2cm} | p{2cm} | p{2cm} | p{2cm} | p{2cm} | p{2cm} |}
		\rowcolor[HTML]{329A9D} 
		{\color[HTML]{FFFFFF} \textbf{Semana 2}} & {\color[HTML]{FFFFFF} \textbf{Día 1}} & {\color[HTML]{FFFFFF} \textbf{Día 2}} & {\color[HTML]{FFFFFF} \textbf{Día 3}} & {\color[HTML]{FFFFFF} \textbf{Día 4}}  & {\color[HTML]{FFFFFF} \textbf{Día 5}} \\ \hline
		Álvaro Fernández-Alonso Araluce & 28.3 & 28.4 & 28.4 & 28.4 & 28.4 \\ \hline
	\end{tabular}
\end{table}

\begin{table}[h]
	\centering
	\begin{tabular}{| p{2cm} | p{2cm} | p{2cm} | p{2cm} | p{2cm} | p{2cm} |}
		\rowcolor[HTML]{329A9D} 
		{\color[HTML]{FFFFFF} \textbf{Semana 3}} & {\color[HTML]{FFFFFF} \textbf{Día 1}} & {\color[HTML]{FFFFFF} \textbf{Día 2}} & {\color[HTML]{FFFFFF} \textbf{Día 3}} & {\color[HTML]{FFFFFF} \textbf{Día 4}}  & {\color[HTML]{FFFFFF} \textbf{Día 5}} \\ \hline
		Álvaro Fernández-Alonso Araluce & Testing & Testing & Testing &Testing & Testing \\ \hline
	\end{tabular}
\end{table}

