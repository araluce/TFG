\section{Plan de entrega 12}

\subsection{Objetivos de la entrega}

El objetivo principal de esta entrega se va a centrar en permitir la comunicación entre usuarios por medio de un chat interno.

\subsection{Listado de HU a desarrollar}

\begin{table}[h]
	\centering
	\begin{tabular}{| p{2.3cm} | p{6.7cm} | p{2cm} |}
		\rowcolor[HTML]{329A9D} 
		{\color[HTML]{FFFFFF} \textbf{Identificador}} & {\color[HTML]{FFFFFF} \textbf{Historias de Usuario}} & {\color[HTML]{FFFFFF} \textbf{Estimación}}  \\ \hline
		HU.30 & Permitir a los diferentes usuarios chatear entre ellos & 48 \\ \hline
	\end{tabular}
\end{table}

\newpage

\subsection{Descomposición en tareas de desarrollo}

A continuación incluimos un desglose de las HU en tareas de desarrollo junto con la estimación realizada de su duración.\\

\begin{table}[h]
	\centering
	\begin{tabular}{| p{2.3cm} | p{6.7cm} | p{2cm} |}
		\rowcolor[HTML]{329A9D} 
		{\color[HTML]{FFFFFF} \textbf{HU.30}} & {\color[HTML]{FFFFFF} \textbf{Permitir a los diferentes usuarios chatear entre ellos}} & {\color[HTML]{FFFFFF} \textbf{48}}  \\ \hline
		Tarea 30.1 & Permitir a los usuarios visualizar una lista de todos los contactos del sistema & 8 \\ \hline
		Tarea 30.2 & Permitir a los usuarios visualizar un chat grupal común & 8 \\ \hline
		Tarea 30.3 & Crear salas de chat de usuario a usuario & 12 \\ \hline
		Tarea 30.4 & Crear salas de chat grupales & 12 \\ \hline
		Tarea 30.5 & En una sala de chat, recibir los mensajes dinámicamente sin tener que recargar & 4 \\ \hline
		Tarea 30.6 & Marcar los mensajes como vistos & 2 \\ \hline
		Tarea 30.7 & Marcar visualmente las salas que contengan mensajes pendientes & 2 \\ \hline
	\end{tabular}
\end{table}

\subsection{Carga prevista en los desarrolladores}

A continuación mostramos la carga de trabajo prevista para cada uno de los desarrolladores en relación a las tareas definidas anteriormente:

\begin{table}[h]
	\centering
	\begin{tabular}{| p{3cm} | p{2cm} | p{2cm} | p{2cm} | p{2cm} |}
		\rowcolor[HTML]{329A9D} 
		{\color[HTML]{FFFFFF} \textbf{Desarrollador}} & {\color[HTML]{FFFFFF} \textbf{Velocidad inicial (días ideales)}} & {\color[HTML]{FFFFFF} \textbf{Dedicación (\% del tiempo)}} & {\color[HTML]{FFFFFF} \textbf{Carga de trabajo (días ideales)}} & {\color[HTML]{FFFFFF} \textbf{\#Tareas aceptadas}}  \\ \hline
		Álvaro Fernández-Alonso Araluce & 15 & 50\% & 15 & 7 \\ \hline
	\end{tabular}
\end{table}

\newpage

\subsection{Planificación temporal de las tareas}

\begin{table}[h]
	\centering
	\begin{tabular}{| p{2cm} | p{2cm} | p{2cm} | p{2cm} | p{2cm} | p{2cm} |}
		\rowcolor[HTML]{329A9D} 
		{\color[HTML]{FFFFFF} \textbf{Semana 1}} & {\color[HTML]{FFFFFF} \textbf{Día 1}} & {\color[HTML]{FFFFFF} \textbf{Día 2}} & {\color[HTML]{FFFFFF} \textbf{Día 3}} & {\color[HTML]{FFFFFF} \textbf{Día 4}}  & {\color[HTML]{FFFFFF} \textbf{Día 5}} \\ \hline
		Álvaro Fernández-Alonso Araluce & 30.1 & 30.1 & 30.2 & 30.2 & 30.3 \\ \hline
	\end{tabular}
\end{table}

\begin{table}[h]
	\centering
	\begin{tabular}{| p{2cm} | p{2cm} | p{2cm} | p{2cm} | p{2cm} | p{2cm} |}
		\rowcolor[HTML]{329A9D} 
		{\color[HTML]{FFFFFF} \textbf{Semana 2}} & {\color[HTML]{FFFFFF} \textbf{Día 1}} & {\color[HTML]{FFFFFF} \textbf{Día 2}} & {\color[HTML]{FFFFFF} \textbf{Día 3}} & {\color[HTML]{FFFFFF} \textbf{Día 4}}  & {\color[HTML]{FFFFFF} \textbf{Día 5}} \\ \hline
		Álvaro Fernández-Alonso Araluce & 30.3 & 30.3 & 30.4 & 30.4 & 30.4 \\ \hline
	\end{tabular}
\end{table}

\begin{table}[h]
	\centering
	\begin{tabular}{| p{2cm} | p{2cm} | p{2cm} | p{2cm} | p{2cm} | p{2cm} |}
		\rowcolor[HTML]{329A9D} 
		{\color[HTML]{FFFFFF} \textbf{Semana 3}} & {\color[HTML]{FFFFFF} \textbf{Día 1}} & {\color[HTML]{FFFFFF} \textbf{Día 2}} & {\color[HTML]{FFFFFF} \textbf{Día 3}} & {\color[HTML]{FFFFFF} \textbf{Día 4}}  & {\color[HTML]{FFFFFF} \textbf{Día 5}} \\ \hline
		Álvaro Fernández-Alonso Araluce & 30.5 & 30.6 30.7 & Testing &Testing & Testing \\ \hline
	\end{tabular}
\end{table}

