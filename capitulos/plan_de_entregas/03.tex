\section{Plan de entrega 3}

\subsection{Objetivos de la entrega}

El objetivo principal de esta entrega consiste en crear las primeras secciones para los usuarios Padawans y terminar de desarrollar las funcionalidades que permiten al usuario Maestro Guidoogway crear y editar retos y misiones desde el BackOffice.

\subsection{Listado de HU a desarrollar}

\begin{table}[h]
	\centering
	\begin{tabular}{| p{2.3cm} | p{6.7cm} | p{2cm} |}
		\rowcolor[HTML]{329A9D} 
		{\color[HTML]{FFFFFF} \textbf{Identificador}} & {\color[HTML]{FFFFFF} \textbf{Historias de Usuario}} & {\color[HTML]{FFFFFF} \textbf{Estimación}}  \\ \hline
		HU.10 & Permitir al Maestro Guidoogway gestionar las misiones de desactivación de minas & 12 \\ \hline
		HU.11 & Permitir al Maestro Guidoogway gestionar los badges & 4 \\ \hline
		HU.12 & Preparar la sección de usuarios Padawans & 20 \\ \hline
		HU.13 & Control de accesos & 4 \\ \hline
	\end{tabular}
\end{table}

\subsection{Descomposición en tareas de desarrollo}

A continuación incluimos un desglose de las HU en tareas de desarrollo junto con la estimación realizada de su duración.\\

\begin{table}[h]
	\centering
	\begin{tabular}{| p{2.3cm} | p{6.7cm} | p{2cm} |}
		\rowcolor[HTML]{329A9D} 
		{\color[HTML]{FFFFFF} \textbf{HU.11}} & {\color[HTML]{FFFFFF} \textbf{Permitir al Maestro Guidoogway gestionar los badges}} & {\color[HTML]{FFFFFF} \textbf{4}}  \\ \hline
		Tarea 11.1 & Permitir al Maestro Guidoogway gestionar los badges & 3 \\ \hline
		Tarea 11.2 & Permitir al Maestro Guidoogway asignar un logo a cada badge  & 1 \\ \hline
	\end{tabular}
\end{table}

\newpage

\begin{table}[t]
	\centering
	\begin{tabular}{| p{2.3cm} | p{6.7cm} | p{2cm} |}
		\rowcolor[HTML]{329A9D} 
		{\color[HTML]{FFFFFF} \textbf{HU.10}} & {\color[HTML]{FFFFFF} \textbf{Permitir al Maestro Guidoogway gestionar las misiones de desactivación de minas}} & {\color[HTML]{FFFFFF} \textbf{12}}  \\ \hline
		Tarea 10.1 & Permitir al Maestro Guidoogway gestionar misiones de minas& 4 \\ \hline
		Tarea 10.2 & Permitir al Maestro Guidoogway asignar un código secreto para desactivar una mina & 2 \\ \hline
		Tarea 10.3 & Permitir al Maestro Guidoogway añadir pistas secretas para que los usuarios Padawans consigan la clave secreta & 2 \\ \hline
		Tarea 10.4 & Permitir al Maestro Guidoogway añadir un beneficio en TdV para los usuarios Padawans que consigan desactivar la mina & 2 \\ \hline
		Tarea 10.5 & Permitir al Maestro Guidoogway establecer la fecha y la hora de la explosión de la mina & 2 \\ \hline
	\end{tabular}
\end{table}

\begin{table}[t]
	\centering
	\begin{tabular}{| p{2.3cm} | p{6.7cm} | p{2cm} |}
		\rowcolor[HTML]{329A9D} 
		{\color[HTML]{FFFFFF} \textbf{HU.12}} & {\color[HTML]{FFFFFF} \textbf{Preparar la sección de usuarios Padawans}} & {\color[HTML]{FFFFFF} \textbf{20}}  \\ \hline
		Tarea 12.1 & Creación en la plataforma Web del apartado para usuarios Padawan & 4 \\ \hline
		Tarea 12.2 & Colocación de las imágenes facilitadas por el cliente para cada sección & 4 \\ \hline
		Tarea 12.3 & Preparación de estilos, fondos y fuentes estipulados por el cliente & 8 \\ \hline
		Tarea 12.4 & Colocación del contador regresivo en cada sección & 4 \\ \hline
	\end{tabular}
\end{table}

\begin{table}[t]
	\centering
	\begin{tabular}{| p{2.3cm} | p{6.7cm} | p{2cm} |}
		\rowcolor[HTML]{329A9D} 
		{\color[HTML]{FFFFFF} \textbf{HU.13}} & {\color[HTML]{FFFFFF} \textbf{Control de accesos}} & {\color[HTML]{FFFFFF} \textbf{4}}  \\ \hline
		Tarea 13.1 & Restringir el acceso a la sección de gestión solo para que puedan acceder usuarios Administradores  & 2 \\ \hline
		Tarea 13.2 & Restringir el acceso a la sección de usuario solo para que puedan acceder usuarios Padawans  & 2 \\ \hline
	\end{tabular}
\end{table}

\newpage
\FloatBarrier

\subsection{Carga prevista en los desarrolladores}

A continuación mostramos la carga de trabajo prevista para cada uno de los desarrolladores en relación a las tareas definidas anteriormente:

\begin{table}[h]
	\centering
	\begin{tabular}{| p{3cm} | p{2cm} | p{2cm} | p{2cm} | p{2cm} |}
		\rowcolor[HTML]{329A9D} 
		{\color[HTML]{FFFFFF} \textbf{Desarrollador}} & {\color[HTML]{FFFFFF} \textbf{Velocidad inicial (días ideales)}} & {\color[HTML]{FFFFFF} \textbf{Dedicación (\% del tiempo)}} & {\color[HTML]{FFFFFF} \textbf{Carga de trabajo (días ideales)}} & {\color[HTML]{FFFFFF} \textbf{\#Tareas aceptadas}}  \\ \hline
		Álvaro Fernández-Alonso Araluce & 15 & 50\% & 15 & 13 \\ \hline
	\end{tabular}
\end{table}

\subsection{Planificación temporal de las tareas}

\begin{table}[h]
	\centering
	\begin{tabular}{| p{2cm} | p{2cm} | p{2cm} | p{2cm} | p{2cm} | p{2cm} |}
		\rowcolor[HTML]{329A9D} 
		 {\color[HTML]{FFFFFF} \textbf{Semana 1}} & {\color[HTML]{FFFFFF} \textbf{Día 1}} & {\color[HTML]{FFFFFF} \textbf{Día 2}} & {\color[HTML]{FFFFFF} \textbf{Día 3}} & {\color[HTML]{FFFFFF} \textbf{Día 4}}  & {\color[HTML]{FFFFFF} \textbf{Día 5}} \\ \hline
		Álvaro Fernández-Alonso Araluce & 10.1 & 10.2 10.3 & 10.4 10.5 & 11.1 11.2 & 12.1 \\ \hline
	\end{tabular}
\end{table}

\begin{table}[h]
	\centering
	\begin{tabular}{| p{2cm} | p{2cm} | p{2cm} | p{2cm} | p{2cm} | p{2cm} |}
		\rowcolor[HTML]{329A9D} 
		{\color[HTML]{FFFFFF} \textbf{Semana 2}} & {\color[HTML]{FFFFFF} \textbf{Día 1}} & {\color[HTML]{FFFFFF} \textbf{Día 2}} & {\color[HTML]{FFFFFF} \textbf{Día 3}} & {\color[HTML]{FFFFFF} \textbf{Día 4}}  & {\color[HTML]{FFFFFF} \textbf{Día 5}} \\ \hline
		Álvaro Fernández-Alonso Araluce & 12.2 & 12.3 & 12.3 & 12.4 & 13.1 13.2 \\ \hline
	\end{tabular}
\end{table}

\begin{table}[h]
	\centering
	\begin{tabular}{| p{2cm} | p{2cm} | p{2cm} | p{2cm} | p{2cm} | p{2cm} |}
		\rowcolor[HTML]{329A9D} 
		{\color[HTML]{FFFFFF} \textbf{Semana 3}} & {\color[HTML]{FFFFFF} \textbf{Día 1}} & {\color[HTML]{FFFFFF} \textbf{Día 2}} & {\color[HTML]{FFFFFF} \textbf{Día 3}} & {\color[HTML]{FFFFFF} \textbf{Día 4}}  & {\color[HTML]{FFFFFF} \textbf{Día 5}} \\ \hline
		Álvaro Fernández-Alonso Araluce & Testing & Testing & Testing & Testing & Testing \\ \hline
	\end{tabular}
\end{table}

