\section{Plan de entrega 6}

\subsection{Objetivos de la entrega}

El objetivo principal de esta entrega se va a centrar en permitir al usuario Padawan poder realizar retos de tipo Dominio de la Fuerza y Audiencia ante el Senado Galáctico.

\subsection{Listado de HU a desarrollar}

\begin{table}[h]
	\centering
	\begin{tabular}{| p{2.3cm} | p{6.7cm} | p{2cm} |}
		\rowcolor[HTML]{329A9D} 
		{\color[HTML]{FFFFFF} \textbf{Identificador}} & {\color[HTML]{FFFFFF} \textbf{Historias de Usuario}} & {\color[HTML]{FFFFFF} \textbf{Estimación}}  \\ \hline
		HU.20 & Permitir al usuario Padawan realizar retos de tipo Dominio de la Fuerza & 20 \\ \hline
		HU.21 & Permitir al usuario Padawan realizar retos de tipo Audiencia ante el Senado Galáctico & 20 \\ \hline
	\end{tabular}
\end{table}


\subsection{Descomposición en tareas de desarrollo}

A continuación incluimos un desglose de las HU en tareas de desarrollo junto con la estimación realizada de su duración.\\

\begin{table}[h]
	\centering
	\begin{tabular}{| p{2.3cm} | p{6.7cm} | p{2cm} |}
		\rowcolor[HTML]{329A9D} 
		{\color[HTML]{FFFFFF} \textbf{HU.20}} & {\color[HTML]{FFFFFF} \textbf{Permitir al usuario Padawan realizar retos de tipo Dominio de la Fuerza}} & {\color[HTML]{FFFFFF} \textbf{20}}  \\ \hline
		Tarea 20.1 & Permitir al sistema cobrar 1 día de vida a todos los usuarios Padawans cuando el Maestro Guidoogway publica un reto de este tipo & 4 \\ \hline
		Tarea 20.2 & Permitir a los usuarios Padawans realizar el reto tipo cuestionario & 6 \\ \hline
		Tarea 20.3 & Permitir al sistema dar una respuesta positiva o negativa dependiendo del resultado del test & 5 \\ \hline
		Tarea 20.4 & Permitir al sistema reingresar al usuario Padawan 1 TdV si realiza correctamente el reto & 5 \\ \hline
	\end{tabular}
\end{table}

\begin{table}[h]
	\centering
	\begin{tabular}{| p{2.3cm} | p{6.7cm} | p{2cm} |}
		\rowcolor[HTML]{329A9D} 
		{\color[HTML]{FFFFFF} \textbf{HU.21}} & {\color[HTML]{FFFFFF} \textbf{Permitir al usuario Padawan realizar retos de tipo Audiencia ante el Senado Galáctico}} & {\color[HTML]{FFFFFF} \textbf{20}}  \\ \hline
		Tarea 21.1 & Permitir al usuario visualizar los retos de Senado Galáctico que el Maestro Guidoogway publique. & 4 \\ \hline
		Tarea 21.2 & Permitir al Maestro Guidoogway recibir las entregas de los usuarios Padawan & 6 \\ \hline
		Tarea 21.3 & Permitir al Maestro Guidoogway corregir las entregas de este tipo & 8 \\ \hline
		Tarea 21.4 & Permitir al sistema ingresar el TdV equivalente al badge obtenido & 2 \\ \hline
	\end{tabular}
\end{table}

\subsection{Carga prevista en los desarrolladores}

A continuación mostramos la carga de trabajo prevista para cada uno de los desarrolladores en relación a las tareas definidas anteriormente:

\begin{table}[h]
	\centering
	\begin{tabular}{| p{3cm} | p{2cm} | p{2cm} | p{2cm} | p{2cm} |}
		\rowcolor[HTML]{329A9D} 
		{\color[HTML]{FFFFFF} \textbf{Desarrollador}} & {\color[HTML]{FFFFFF} \textbf{Velocidad inicial (días ideales)}} & {\color[HTML]{FFFFFF} \textbf{Dedicación (\% del tiempo)}} & {\color[HTML]{FFFFFF} \textbf{Carga de trabajo (días ideales)}} & {\color[HTML]{FFFFFF} \textbf{\#Tareas aceptadas}}  \\ \hline
		Álvaro Fernández-Alonso Araluce & 15 & 50\% & 15 & 8 \\ \hline
	\end{tabular}
\end{table}

\subsection{Planificación temporal de las tareas}

\begin{table}[h]
	\centering
	\begin{tabular}{| p{2cm} | p{2cm} | p{2cm} | p{2cm} | p{2cm} | p{2cm} |}
		\rowcolor[HTML]{329A9D} 
		 {\color[HTML]{FFFFFF} \textbf{Semana 1}} & {\color[HTML]{FFFFFF} \textbf{Día 1}} & {\color[HTML]{FFFFFF} \textbf{Día 2}} & {\color[HTML]{FFFFFF} \textbf{Día 3}} & {\color[HTML]{FFFFFF} \textbf{Día 4}}  & {\color[HTML]{FFFFFF} \textbf{Día 5}} \\ \hline
		Álvaro Fernández-Alonso Araluce & 20.1 & 20.2 & 20.2 20.3 & 20.3 20.4 & 20.4 \\ \hline
	\end{tabular}
\end{table}

\begin{table}[h]
	\centering
	\begin{tabular}{| p{2cm} | p{2cm} | p{2cm} | p{2cm} | p{2cm} | p{2cm} |}
		\rowcolor[HTML]{329A9D} 
		{\color[HTML]{FFFFFF} \textbf{Semana 2}} & {\color[HTML]{FFFFFF} \textbf{Día 1}} & {\color[HTML]{FFFFFF} \textbf{Día 2}} & {\color[HTML]{FFFFFF} \textbf{Día 3}} & {\color[HTML]{FFFFFF} \textbf{Día 4}}  & {\color[HTML]{FFFFFF} \textbf{Día 5}} \\ \hline
		Álvaro Fernández-Alonso Araluce & 21.1 & 21.2 & 21.2 21.3 & 21.6 & 21.6 21.4 \\ \hline
	\end{tabular}
\end{table}

\begin{table}[h]
	\centering
	\begin{tabular}{| p{2cm} | p{2cm} | p{2cm} | p{2cm} | p{2cm} | p{2cm} |}
		\rowcolor[HTML]{329A9D} 
		{\color[HTML]{FFFFFF} \textbf{Semana 3}} & {\color[HTML]{FFFFFF} \textbf{Día 1}} & {\color[HTML]{FFFFFF} \textbf{Día 2}} & {\color[HTML]{FFFFFF} \textbf{Día 3}} & {\color[HTML]{FFFFFF} \textbf{Día 4}}  & {\color[HTML]{FFFFFF} \textbf{Día 5}} \\ \hline
		Álvaro Fernández-Alonso Araluce & Testing & Testing & Testing & Testing & Testing \\ \hline
	\end{tabular}
\end{table}

