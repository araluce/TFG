\setcounter{chapter}{9}
\setcounter{section}{0}
\setcounter{subsection}{0}
\chapter{Conclusiones}

Tras la realización del trabajo y posterior prueba en aulas reales durante dos cursos consecutivos puedo afirmar que la gamificación de una asignatura estimula al alumnado y despierta en éstos un interés por participar. Durante el desarrollo de los cursos se aprecia una creciente interacción entre los alumnos y los retos propuestos por el docente.\\

Por otro lado, durante dos años consecutivos se ha abordado un problema real en las instituciones educativas de manera que se demuestra que el uso de las tecnologías se adaptan completamente al ámbito educativo.

\section{Líneas futuras}

Como próximos desarrollos que me gustaría implementar en el proyecto \textbf{SinTime} destacaría:

\begin{enumerate}
	\item Desarrollar una versión móvil
	\item Apuestas
	\item Tutorías
	\item Romper el sistema
\end{enumerate}

\subsubsection{Desarrollar una versión móvil}

Todos conocemos la creciente demanda de aplicaciones móviles por lo que sería una evolución natural desarrollar una versión para dispositivos móviles.\\

Para llegar a ese punto se debe realizar un nuevo desarrollo de una API en el entorno del servidor. De esta forma, cualquier desarrollo podría consumir de este servicio.\\

Por otro lado el desarrollo móvil podría realizarse en Ionic para dar soporte a diferentes plataformas como Android y Iphone.

\subsubsection{Apuestas}

No sigue una línea educativa pero realmente expandería la gama de toma de decisiones dentro del entorno SinTime. Cada semana el sistema calcularía un número al azar del 1 al 20. El usuario Padawan que lo desee podría comprar un numero durante la semana por una cantidad de TdV y, al final de la semana, el sistema abonaría el bote recaudado a los ganadores.\\

\subsubsection{Tutorías}

Una sección bastante útil para el docente. La gestión de tutorías desde la misma plataforma sería bastante útil ya que, si la asignatura se basa en la aplicación, podría tener toda la información del alumno en una vista de todo el desarrollo del alumno en la aplicación.\\

El docente podría marcar los huecos libres que tendría durante la semana para ofrecer tutorías. \\
\\
El alumno podría visualizar el horario con esos huecos y distinguir cuáles de ellos están disponibles o están ya ocupados por otro alumno. Podría a su vez solicitar un slot de tiempo para una cita que el tutor tendría que aceptar posteriormente y ese hueco de tiempo tendría un coste simbólico en TdV.\\

Por otro lado el docente podría aceptar esa cita y devolver esa cantidad de TdV pagada por el alumno si la tutoría ha merecido la pena.\\

\subsubsection{Romper el sistema}

Como punto final, el docente podría marcar el final de una etapa docente con un circuito de retos que el alumno podrá completar opcionalmente. Es más bien un reto global para todos los alumnos.\\

El sistema iría acumulando a lo largo del curso todo el TdV cobrado a los alumnos en un \textbf{Banco del Tiempo}. Una vez abierto el circuito, el que lo complete podrá acceder a un apartado donde se mostrará todo el TdV acumulado por el sistema a lo largo del curso. Este TdV del Banco del Tiempo estará dividido en slots de TdV equitativos y el usuario que acceda podrá repartir cada slot con usuarios diferentes incluyéndose a sí mismo.