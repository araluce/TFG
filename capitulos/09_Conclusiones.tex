\setcounter{chapter}{9}
\setcounter{section}{0}
\setcounter{subsection}{0}
\chapter{Conclusiones}

Tras la realización del trabajo y posterior prueba en aulas reales durante dos cursos consecutivos puedo afirmar que la gamificación de una asignatura estimula al alumnado y despierta en éstos un interés 







Después de realizar este trabajo, se puede afirmar que la localización en
interiores tiene un amplio abanico de posibilidades y tecnologías que ofrecen
muy buenos resultados, cada una de ellas tiene sus ventajas y desventajas. Se
puede llegar a decir que una combinación de tecnologías de localización en
interiores mejora la precisión del sistema de localización.
Se ha comprobado el potencial de los pequeños dispositivos de Bluetooth
BLE, y explorado las numerosas opciones que se pueden escoger para la
aplicación en el mundo real.
Se ha aplicado a un problema real, con un futuro prometedor en el ámbito
de la educación, de forma que se demuestra que estas tecnologías se adaptan
a cualquier ámbito.
Se ha profundizado en la aplicación de un sistema gamificado usando
localización en interiores de manera efectiva y con buenos resultados.
También se ha demostrado que el uso de estas tecnologías y técnicas de
localización son útiles para resolver una infinidad de problemas en la vida
cotidiana de la gente.