\setcounter{chapter}{2}
\setcounter{section}{0}
\setcounter{subsection}{0}
\chapter{Planificación}

Para este proyecto hemos seguido una metodología de desarrollo XP (eXtreme Programming). En el destacamos simplicidad en el desarrollo, un feedback constante con el usuario (incluso con los usuarios finales) y un desarrollo iterativo e incremental.\\

\section{Trello}

Para el proceso de planificación hemos decidido usar la herramienta online \textbf{trello}, que es un software de administración de proyectos con interfaz web basada en \textit{cartas kanban}. Con esta herramienta dividimos tareas en columnas para crear un flujo de trabajo organizado, la organización de las tareas las dividiremos en las siguientes columnas:

\begin{itemize}
	\item \textbf{Backlog:} En esta categoría agruparemos las tareas o mejoras que no sean urgentes pero estaría bien desarrollar. Estas tareas tienen una prioridad mayormente baja o una fecha de desarrollo estipulada en un futuro próximo.
	\item \textbf{ToDo:} En esta categoría agruparemos las tareas que tengamos pendiente de desarrollo ordenadas por nivel de prioridad de alta a baja.
	\item \textbf{Doing:} En esta categoría agruparemos las tareas que estemos desarrollando. Cada desarrollador tendrá asignada una y solo una tarea bajo esta categoría siempre. Cuando termine esa tarea la pasará a la siguiente categoría y escogerá otra tarea de la categoría \textbf{ToDo}. En este caso, como es un desarrollo de una sola persona siempre habrá una sola tarea en esta categoría.
	\item \textbf{QA:} En esta categoría agruparemos las tareas ya terminadas. La función de esta es la de revisar que el nuevo desarrollo se adapta bien al código de la rama master. Si supera los test se hará un PR y se pasará a la siguiente categoría. En caso contrario esta tarea volverá a la categoría \textbf{Doing}.
	\item \textbf{Done: } En esta categoría agruparemos las tareas terminadas que hayan pasado los test. Se acepta el Pull Request y se hace un merge con la rama master. Esta rama no se desplegará hasta que llegue la fecha final del Sprint.
\end{itemize}

\begin{figure}[ht]
	\centering
	\includegraphics[width=0.8\textwidth]{imagenes/trello.png}
	\caption{Trello}
	\label{trello}
\end{figure}

\section{Sprints}
Los sprint son \textit{periodos de desarrollo} pactados con el cliente en el que definimos un paquete de nuevas funcionalidades. En este proyecto vamos a definir el Sprint tipo 2 + 1, esto es, dos semanas de desarrollo más una semana de testing.

\section{Técnica Pomodoro (Productividad) \cite{Pomodoro}}

En relación a la productividad del desarrollo del proyecto \textbf{SinTime}, se ha usado la técnica \textbf{Pomodoro} que es un método para gestionar el tiempo. Esta técnica se usa para combatir la ansiedad producida por la cantidad de tareas a desarrollar en un corto periodo de tiempo.\\

En lugar de trabajar con prisas a fin de conseguir terminar todo el trabajo en la fecha estipulada, esta técnica usa el tiempo como aliado para alcanzar el objetivo de un modo correcto y nos permite mejorar continuamente la manera en que trabajamos.

\subsection{Los objetivos}

La Técnica \textbf{Pomodoro} nos permite mejorar la productividad mediante un proceso para conseguir:\\

\begin{itemize}
	\item Aliviar la ansiedad a comenzar las tareas.
	\item Aumentar la concentración disminuyendo las interrupciones.
	\item Impulsar la motivación y nos mantiene constantes.
	\item Refina el proceso de estimación.
	\item Mejora el proceso de trabajo.
\end{itemize}

\subsection{Los principios}

Cada día se seleccionan las tareas a completar y las colocamos en la lista \textbf{ToDo} de la que hemos hablado anteriormente. Podemos usar el cronómetro de nuestro reloj o un timer de \textbf{Pomodoro} para, a continuación:

\begin{enumerate}
	\item Comenzar a trabajar: Iniciamos el timer con una duración de 25 minutos, que es el equivalente a una \textbf{sesión Pomodoro}. Una sesión Pomodoro es ininterrumpible por lo que no se debe pausar y reanudar, si se hace una pausa en mitad debe reiniciarse la sesión.
	\item Cuando el timer llegue a cero comenzamos un periodo de descanso llamado \textbf{Short break} de 3 a 5 minutos, después de este descanso se comienza otra sesión Pomodoro.
	\item Cada 4 sesiones Pomodoro completas se realiza un \textbf{Long Break}. Un tiempo entre 15-30 minutos.
	\item Seguir trabajando hasta que la tarea haya terminado.
\end{enumerate}


