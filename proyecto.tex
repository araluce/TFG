\documentclass[a4paper,11pt]{book}
%\documentclass[a4paper,twoside,11pt,titlepage]{book}
\usepackage{listings}
\usepackage[utf8]{inputenc}
\usepackage[spanish]{babel}

\usepackage[final]{pdfpages}

% \usepackage[style=list, number=none]{glossary} %
%\usepackage{titlesec}
%\usepackage{pailatino}

\decimalpoint
\usepackage{dcolumn}
\newcolumntype{.}{D{.}{\esperiod}{-1}}
\makeatletter
\addto\shorthandsspanish{\let\esperiod\es@period@code}
\makeatother


%\usepackage[chapter]{algorithm}
\RequirePackage{verbatim}
%\RequirePackage[Glenn]{fncychap}
\usepackage{fancyhdr}
\usepackage{graphicx}
\usepackage{afterpage}
\usepackage{wrapfig}
\usepackage{longtable}
\usepackage[pdfborder={000}]{hyperref} %referencia

\usepackage[T1]{fontenc}
\usepackage{charter}
\usepackage{environ}
\usepackage{tikz}
\usetikzlibrary{calc,matrix}

% ********************************************************************
% Re-usable information
% ********************************************************************
\newcommand{\myTitle}{Proyecto SinTime\xspace}
\newcommand{\mySubtitle}{Aplicacion Web Gamificada de Docendia}
\newcommand{\myDegree}{Grado en Ingeniería Informática\xspace}
\newcommand{\myName}{Álvaro Fernández-Alonso Araluce (alumno)\xspace}
\newcommand{\myProf}{Juan Manuel Fernández Luna (tutor1)\xspace}
\newcommand{\myOtherProf}{Isaac José Pérez López (tutor2)\xspace}
%\newcommand{\mySupervisor}{Put name here\xspace}
\newcommand{\myFaculty}{Escuela Técnica Superior de Ingenierías Informática y de
Telecomunicación\xspace}
\newcommand{\myFacultyShort}{E.T.S. de Ingenierías Informática y de
Telecomunicación\xspace}
\newcommand{\myDepartment}{Departamento de ...\xspace}
\newcommand{\myUni}{\protect{Universidad de Granada}\xspace}
\newcommand{\myLocation}{Granada\xspace}
\newcommand{\myTime}{\today\xspace}
\newcommand{\myVersion}{Version 0.1\xspace}
\newcommand{\myKeyWords}{SinTime, SinTime\_Wars, docencia, gamificación}


\hypersetup{
pdfauthor = {\myName (araluce (en) correo (punto) ugr (punto) es)},
pdftitle = {\myTitle},
pdfsubject = {},
pdfkeywords = {SinTime, SinTime\_Wars, docencia, gamificación, ...},
pdfcreator = {LaTeX con el paquete ....},
pdfproducer = {pdflatex}
}

%\hyphenation{}


%\usepackage{doxygen/doxygen}
%\usepackage{pdfpages}
\usepackage{url}
\usepackage{colortbl,longtable}
\usepackage[stable]{footmisc}
\usepackage{dirtree}
\usepackage{booktabs}
\usepackage{tabularx}
\usepackage{acro}


%Acronimos
\DeclareAcronym{TdV}{
short = TdV ,
long = Tiempo de Vida,
class = abbrev
}
\DeclareAcronym{Guidoogway}{
short = Maestro Guidoogway ,
long = Profesor en el papel de ``Guardián del Tiempo'',
class = abbrev
}
\DeclareAcronym{Padawan}{
short = Padawan ,
long = Alumno en el papel de ``Ciudadano SinTime'',
class = abbrev
}
\DeclareAcronym{Metronomista}{
short = Metronomista ,
long = Sistema en el papel de ``Metronomista'',
class = abbrev
}

\DeclareAcronym{Reto}{
	short = Reto ,
	long = Ejercicio,
	class = abbrev
}

\DeclareAcronym{Badge}{
	short = Badge ,
	long = Calificación,
	class = abbrev
}

%\usepackage{index}

%\makeindex
%\usepackage[style=long, cols=2,border=plain,toc=true,number=none]{glossary}
% \makeglossary

% Definición de comandos que me son tiles:
%\renewcommand{\indexname}{Índice alfabético}
%\renewcommand{\glossaryname}{Glosario}

\pagestyle{fancy}
\fancyhf{}
\fancyhead[LO]{\leftmark}
\fancyhead[RE]{\rightmark}
\fancyhead[RO,LE]{\textbf{\thepage}}
\renewcommand{\chaptermark}[1]{\markboth{\textbf{#1}}{}}
\renewcommand{\sectionmark}[1]{\markright{\textbf{\thesection. #1}}}

\setlength{\headheight}{1.5\headheight}

\newcommand{\HRule}{\rule{\linewidth}{0.5mm}}
%Definimos los tipos teorema, ejemplo y definición podremos usar estos tipos
%simplemente poniendo \begin{teorema} \end{teorema} ...
\newtheorem{teorema}{Teorema}[chapter]
\newtheorem{ejemplo}{Ejemplo}[chapter]
\newtheorem{definicion}{Definición}[chapter]

\definecolor{gray97}{gray}{.97}
\definecolor{gray75}{gray}{.75}
\definecolor{gray45}{gray}{.45}
\definecolor{gray30}{gray}{.94}

\lstset{ frame=Ltb,
     framerule=0.5pt,
     aboveskip=0.5cm,
     framextopmargin=3pt,
     framexbottommargin=3pt,
     framexleftmargin=0.1cm,
     framesep=0pt,
     rulesep=.4pt,
     backgroundcolor=\color{gray97},
     rulesepcolor=\color{black},
     %
     stringstyle=\ttfamily,
     showstringspaces = false,
     basicstyle=\scriptsize\ttfamily,
     commentstyle=\color{gray45},
     keywordstyle=\bfseries,
     %
     numbers=left,
     numbersep=6pt,
     numberstyle=\tiny,
     numberfirstline = false,
     breaklines=true,
   }
 
% minimizar fragmentado de listados
\lstnewenvironment{listing}[1][]
   {\lstset{#1}\pagebreak[0]}{\pagebreak[0]}

\lstdefinestyle{CodigoC}
   {
	basicstyle=\scriptsize,
	frame=single,
	language=C,
	numbers=left
   }
\lstdefinestyle{CodigoC++}
   {
	basicstyle=\small,
	frame=single,
	backgroundcolor=\color{gray30},
	language=C++,
	numbers=left
   }

 
\lstdefinestyle{Consola}
   {basicstyle=\scriptsize\bf\ttfamily,
    backgroundcolor=\color{gray30},
    frame=single,
    numbers=none
   }


\newcommand{\bigrule}{\titlerule[0.5mm]}


%Para conseguir que en las páginas en blanco no ponga cabecerass
\makeatletter
\def\clearpage{%
  \ifvmode
    \ifnum \@dbltopnum =\m@ne
      \ifdim \pagetotal <\topskip
        \hbox{}
      \fi
    \fi
  \fi
  \newpage
  \thispagestyle{empty}
  \write\m@ne{}
  \vbox{}
  \penalty -\@Mi
}
\makeatother

% code by Andrew:
% http://tex.stackexchange.com/a/28452/13304
\makeatletter
\let\matamp=&
\catcode`\&=13
\makeatletter
\def&{\iftikz@is@matrix
	\pgfmatrixnextcell
	\else
	\matamp
	\fi}
\makeatother

\newcounter{lines}
\def\endlr{\stepcounter{lines}\\}

\newcounter{vtml}
\setcounter{vtml}{0}

\newif\ifvtimelinetitle
\newif\ifvtimebottomline
\tikzset{description/.style={
		column 2/.append style={#1}
	},
	timeline color/.store in=\vtmlcolor,
	timeline color=red!80!black,
	timeline color st/.style={fill=\vtmlcolor,draw=\vtmlcolor},
	use timeline header/.is if=vtimelinetitle,
	use timeline header=false,
	add bottom line/.is if=vtimebottomline,
	add bottom line=false,
	timeline title/.store in=\vtimelinetitle,
	timeline pretitle/.store in=\vtimelinepretitle,
	timeline title={},
	line offset/.store in=\lineoffset,
	line offset=4pt,
}

\NewEnviron{vtimeline}[1][]{%
	\setcounter{lines}{1}%
	\stepcounter{vtml}%
	\begin{tikzpicture}[column 1/.style={anchor=east},
	column 2/.style={anchor=west},
	text depth=0pt,text height=1ex,
	row sep=1ex,
	column sep=1em,
	#1
	]
	\matrix(vtimeline\thevtml)[matrix of nodes]{\BODY};
	\pgfmathtruncatemacro\endmtx{\thelines-1}
	\path[timeline color st] 
	($(vtimeline\thevtml-1-1.north east)!0.5!(vtimeline\thevtml-1-2.north west)$)--
	($(vtimeline\thevtml-\endmtx-1.south east)!0.5!(vtimeline\thevtml-\endmtx-2.south west)$);
	\foreach \x in {1,...,\endmtx}{
		\node[circle,timeline color st, inner sep=0.15pt, draw=white, thick] 
		(vtimeline\thevtml-c-\x) at 
		($(vtimeline\thevtml-\x-1.east)!0.5!(vtimeline\thevtml-\x-2.west)$){};
		\draw[timeline color st](vtimeline\thevtml-c-\x.west)--++(-3pt,0);
	}
	\ifvtimelinetitle%
	\draw[timeline color st]([yshift=\lineoffset]vtimeline\thevtml.north west)--
	([yshift=\lineoffset]vtimeline\thevtml.north east);
	\node[anchor=west,yshift=16pt,font=\large]
	at (vtimeline\thevtml-1-1.north west) 
	{\textsc{\vtimelinepretitle}: \textit{\vtimelinetitle}};
	\else%
	\relax%
	\fi%
	\ifvtimebottomline%
	\draw[timeline color st]([yshift=-\lineoffset]vtimeline\thevtml.south west)--
	([yshift=-\lineoffset]vtimeline\thevtml.south east);
	\else%
	\relax%
	\fi%
	\end{tikzpicture}
}

\usepackage{pdfpages}

\lstloadlanguages{Ruby}
\lstset{%
	basicstyle=\ttfamily\color{black},
	commentstyle = \ttfamily\color{red},
	keywordstyle=\ttfamily\color{blue},
	stringstyle=\color{orange}}

\begin{document}
\input{portada/portada}
\chapter*{}
%\thispagestyle{empty}
%\cleardoublepage

%\thispagestyle{empty}

\input{portada/portada_2}



\cleardoublepage
\thispagestyle{empty}

\begin{center}
{\large\bfseries Proyecto \$inTime: Aplicación Web Gamificada de Docencia }\\
\end{center}
\begin{center}
Álvaro Fernández-Alonso Araluce (alumno)\\
\end{center}

%\vspace{0.7cm}
\noindent{\textbf{Palabras clave}: \$inTime, docencia, gamificación }\\

\vspace{0.7cm}
\noindent{\textbf{Resumen}}\\

En este documento se desarrollará una idea dirigida a incrementar la atención, participación, dedicación y esfuerzo por parte del alumnado en sus obligaciones como estudiante.\\

Las actividades lúdicas provocan que focalicemos toda nuestra atención en superar retos de forma constante provocando incluso, por medio de historias; escenarios y experiencias, la inmersión total de la persona que las realice. La idea consiste en extraer todo ese mecanismo e implantarlo en la educación haciendo que los retos a superar sean retos educativos.\\

En este trabajo se presenta un escenario inmersivo en el que científicos han conseguido detener el gen de crecimiento. Dado que ya nadie fallece por causas naturales se implanta en cada persona un reloj regresivo con 10 años de vida. A partir de ese momento el trabajo se cobra en tiempo, los alimentos cuestan tiempo, toda la economía se basa en el tiempo. Si un reloj llega a cero provocará un infarto instantáneo.\\

El escenario anteriormente descrito es una breve sinopsis de la película \textbf{InTime} del director neozelandés \textbf{Andrew Niccol} y es en esta trama donde van a verse inmersos \textbf{los estudiantes de Fundamentos de la Educación Física} del curso 2016/17 por medio de una \textbf{Aplicación Web} que va a simular este escenario.\\

En este caso cada estudiante (ciudadano de \$inTime a partir de ahora) va a comenzar su experiencia con 15 días de vida en distritos independientes, cada segundo que pasa es un segundo menos que tienen y tendrán que realizar diversos retos para obtener más tiempo.

\cleardoublepage


\thispagestyle{empty}


\begin{center}
{\large\bfseries \$inTime Project: Gamified Web Application for Teaching}\\
\end{center}
\begin{center}
Álvaro Fernández-Alonso Araluce (alumno)\\
\end{center}

%\vspace{0.7cm}
\noindent{\textbf{Keywords}: \$inTime, Gamification, Teaching}\\

\vspace{0.7cm}
\noindent{\textbf{Abstract}}\\

Write here the abstract in English.

\chapter*{}
\thispagestyle{empty}

\noindent\rule[-1ex]{\textwidth}{2pt}\\[4.5ex]

Yo, \textbf{Álvaro Fernández-Alonso Araluce (alumno)}, alumno de la titulación TITULACIÓN de la \textbf{Escuela Técnica Superior
de Ingenierías Informática y de Telecomunicación de la Universidad de Granada}, con DNI 75167394J, autorizo la
ubicación de la siguiente copia de mi Trabajo Fin de Grado en la biblioteca del centro para que pueda ser
consultada por las personas que lo deseen.

\vspace{6cm}

\noindent Fdo: Álvaro Fernández-Alonso Araluce (alumno)

\vspace{2cm}

\begin{flushright}
Granada a Mayo de mes de 2017.
\end{flushright}


\chapter*{}
\thispagestyle{empty}

\noindent\rule[-1ex]{\textwidth}{2pt}\\[4.5ex]

D. \textbf{Nombre Apellido1 Apellido2 (tutor1)}, Profesor del Área de XXXX del Departamento YYYY de la Universidad de Granada.

\vspace{0.5cm}

D. \textbf{Nombre Apellido1 Apellido2 (tutor2)}, Profesor del Área de XXXX del Departamento YYYY de la Universidad de Granada.


\vspace{0.5cm}

\textbf{Informan:}

\vspace{0.5cm}

Que el presente trabajo, titulado \textit{\textbf{Título del proyecto, Subtítulo del proyecto}},
ha sido realizado bajo su supervisión por \textbf{Nombre Apellido1 Apellido2 (alumno)}, y autorizamos la defensa de dicho trabajo ante el tribunal
que corresponda.

\vspace{0.5cm}

Y para que conste, expiden y firman el presente informe en Granada a X de mes de 201 .

\vspace{1cm}

\textbf{Los directores:}

\vspace{5cm}

\noindent \textbf{Nombre Apellido1 Apellido2 (tutor1) \ \ \ \ \ Nombre Apellido1 Apellido2 (tutor2)}

\chapter*{Agradecimientos}
\thispagestyle{empty}

       \vspace{1cm}


A \textbf{Soraya} por su apoyo incondicional.\\

A \textbf{Juan Manuel Fernández Luna} por su paciencia y consejos.\\

A \textbf{Isaac José Pérez López} por contar conmigo para desarrollar una idea tan original y a \textbf{sus alumnos de Fundamentos de la Educación Física} por la experiencia vivida.


\frontmatter
\tableofcontents
\listoffigures
\listoftables

%\mainmatter
%\setlength{\parskip}{5pt}
\setcounter{chapter}{1}
\setcounter{section}{0}
\chapter{Introducción}

El objetivo de este proyecto es favorecer la participación del alumnado tanto en las sesiones presenciales como en casa. Para ello se ha desarrollado una aplicación Web gamificada en Ruby on Rails a modo de plataforma de docencia donde alumnos y profesores interactúan en todo momento.\\

La aplicación va a simular un escenario donde los alumnos tendrán en cada una de las pantalla un reloj regresivo, todos los alumnos comenzarán con el mismo tiempo de vida (TdV en adelante) y no llegar a cero será el objetivo principal de cada uno de ellos. El TdV, por tanto, se convertirá en una moneda sobre la que emergerá toda la economía de la aplicación, si además añadimos que cada ejercicio realizado reportará una cantidad de TdV sobre el contador del que lo realice en función de la resolución de éste tenemos como resultado una competición entre alumnos que provocará que ellos mismos pidan realizar todos los ejercicios/entregas posibles. Las formas de obtener TdV se detallarán más adelante en la sección \textbf{Objetivos de un padawan Sintime}.\\

\section{Motivación\cite{Motivacion1, Motivacion2, Motivacion3, Motivacion4, Motivacion5, Motivacion6, Motivacion7, Motivacion8, Motivacion9}}

Uno de los principales problemas que sufre la Universidad en la actualidad es la falta de motivación y compromiso de los estudiantes a la hora de participar activamente en su aprendizaje. Esta falta de interés y motivación puede estar determinada, en gran medida, por el rol pasivo que desempeñan los estudiantes en las metodologías de enseñanza tradicionales (Martí-Parreño, 2015), especialmente las generaciones más jóvenes, los nativos digitales, esto es, estudiantes que hacen un uso intensivo de la tecnología y de la interactividad digital (Prensky, 2001). Por tanto, la necesidad de mejorar el atractivo de la docencia y adoptar nuevas metodologías de enseñanza-aprendizaje activas, que favorezcan la motivación e implicación del alumnado, se ha convertido en todo un reto para la Universidad que su profesorado deberá afrontar cuanto antes (Contreras y Eguia, 2016; De Jorge et al., 2011; Kiryakova, Angelova, y Yordanova, 2014; Martínez González, 2011). \\

En este sentido, una poderosa estrategia para motivar y favorecer el aprendizaje del alumnado es la gamificación, entendida como el uso de elementos de diseño del juego en contextos no lúdicos con la finalidad de motivar a los participantes (Deterding et al., 2011). Para ello incorpora elementos de los videojuegos como el contexto, los desafíos o las recompensas, aumentando la interacción del alumnado con el entorno de aprendizaje y acabar modificando su conducta (González González y Mora, 2015; Prieto et al., 2014).\\


La gamificación nos ofrece ciertas ventajas frente a las metodologías de enseñanza tradicionales:\\

\begin{itemize}
\item Motivación del alumnado.
\item Desarrollo de habilidades mediante el crecimiento gradual de la dificultad para realizar una tarea.
\item Fomenta la competencia y ofrece un reconocimiento (Rankings).
\item Fomenta la conexión social, ya que los estudiantes viven juntos una experiencia estimulante.
\end{itemize}

\section{Diccionario de acrónimos}
\begin{itemize}
\item \ac{TdV}
\item \ac{Guidoogway}
\item \ac{Padawan}
\item \ac{Metronomista}
\item \ac{Reto}
\item \ac{Badge}
\end{itemize}

\section{La economía en SinTime}
Anteriormente se ha hablado de forma general sobre la economía en SinTime. Podría decirse que ésta es la pieza central de la plataforma. Para gestionar esta moneda, cada alumno (Padawan en adelante) tendrá una ``cuenta corriente'' en la que se verán reflejadas todas sus operaciones bancarias. Para cada operación tendrán la información de cuándo se ha realizado, el motivo del ingreso/gasto y la cantidad de TdV implicada.\\

\textbf{Historial de transacciones}
\begin{figure}[ht]
	\centering
	\includegraphics[width=0.5\textwidth]{imagenes/secciones/introduccion/historial.png}
	\caption{Historial de transacciones de un usuario SinTime}
	\label{historial}
\end{figure}

\section{Los retos en SinTime}
Los retos en SinTime se podrían reducir a ejercicios/entregas en su definición más simple aunque se le da un significado totalmente diferente. Son retos porque se desafía al alumno a realizarlos, y se le da la oportunidad de ir mejorándolo hasta alcanzar la máxima ``calificación".\\

Como en un juego, al realizar una entrega de un reto el alumno consigue un score/puntuación y, por medio de la reentrega conseguimos que el alumno quiera superarse, reintentarlo hasta conseguir el máximo feddback posible. \\


La relación entre un reto y un alumno tendrá uno de los siguiente estados:

\begin{itemize}
	\item \textbf{Comprado}: Un alumno que ha comprado ese reto con su TDV
	\item \textbf{Entregado}: Después de la compra podrá entregar el reto.
	\item \textbf{Calificado}: El docente ha calificado la entrega y asignado un badge.
\end{itemize}

\section{Objetivos de un padawan SinTime}

El objetivo principal de un padawan SinTime será obtener el máximo TdV que pueda para no caer en el lado oscuro y obtener así una buena calificación en su nota final. Son muchas y variadas las formas por las que podrán obtener/perder TdV. Algunas de ellas son:

\begin{itemize}
	\item Seguimiento de maestro Jedi
	\item Pruebas de nivel
	\item Dominio de la fuerza
	\item Víveres
	\item Audiencia ante el Senado Galáctico
	\item Viajes Interplanetarios
	\item Felicidad
\end{itemize}

\subsection{Seguimiento de maestro Jedi}
Para este tipo de reto se ha desarrollado una integración de la plataforma con Twitter. El reto consiste en guardar un mínimo de Tweets diarios que tengan cierta relevancia en el mundo de la Educación Física.\\

Si al final del día el alumno ha guardado un número mínimo de Tweets (especificado por el docente), se le ingresará en su cuenta bancaria un TDV (especificado por el docente).

\subsection{Pruebas de nivel}
Consiste en un \textbf{reto gratuito} compuesto por un enunciado en el que el alumno realizará una entrega de un fichero con una presentación que desarrollará en clase frente a los compañeros. La presentación que hará estará marcada por el enunciado del reto.\\

El profesor marcará un número máximo de alumnos que puedan ``pedirse'' el reto, sólo podrán realizarlo los X alumnos primeros en marcar el reto como "Pedido" (comprado internamente).

\subsection{Dominio de la fuerza}
El docente podrá lanzar a sus alumnos retos sorpresas, este tipo de reto resta 1 día de vida a todos los alumnos al ser lanzado recuperable únicamente si contestan correctamente al cuestionario.

\subsection{Víveres}
Para hacer la experiencia más inmersiva creamos a los alumnos la necesidad de alimentarse e hidratarse. Para ello lanzamos dos tipos de retos; \textbf{Comida} y \textbf{Agua}. La necesidad de realizar estos tipos de retos surge al crear una barra de energía para la sección de comida y otra barra de energía para la sección de agua. Estas barras se irán vaciando conforme pasen días sin entregas realizadas.\\

Estos retos son comprados por el alumno (tienen un coste en tiempo), al ser corregidos por el docente se le realizará un ingreso correspondiente al badge recibido.\\

Además, estos retos podrán ser grupales (realizados por todo el clan) o individuales. Añadir que para la sección de Agua, los retos individuales no podrán ser elegidos directamente, el alumno solicitará un reto individual y el sistema le asignará un reto aleatoriamente.

\subsection{Audiencia ante el Senado Galáctico}
Este reto se lanza la última semana de la asignatura. En esta sección, el alumno realizará la entrega de su proyecto final, en el que reflejará todo lo aprendido durante toda su experiencia.

\subsection{Viajes Interplanetarios}
Es un tipo de reto deportivo. Durante cada semana se especifican unos requisitos para cubrir durante la semana bien corriendo o bien en bicicleta.\\

En este tipo de retos el docente especifica una velocidad media y duración míminas (en caso de bicicleta) y un ritmo medio y duración minimas (en caso de running). Los alumnos por su lado intentarán cubrir esa duración durante la semana en la modalidad que prefieran y en diferentes sesiones (mínimo 3 sesiones distintas). Si el alumno cumple los requisitos de la semana en la modalidad seleccionada se le hará un ingreso automático al final de la semana determinado por el docente.

\subsection{Retos de Felicidad}
Estos retos son más una labor humanitaria por parte de los alumnos, es un reto totalmente opcional sin coste alguno en el que un alumno propone una labor social a realizar durante 3 meses en una primera entrega. La segunda entrega que realice será una evidencia de la labor realizada.\\

Este reto tendrá como respuesta un badge con el ingreso de tiempo correspondiente al mismo.

\newpage

\section{Secciones}

\subsection{Organización de secciones [Rol=Guidoogway]}
\dirtree{%
	.1 manager/home[Rol=Guidoogway].
	.2 Dashboard.
	.3 Distritos.
	.3 Padawans.
	.4 Entregas.
	.4 Movimientos bancarios.
	.4 Subir de nivel.
	.4 Datos deportivos.
	.3 Vacaciones.
	.2 Ejercicios (Retos).
	.3 Entregas.
	.3 Cuestionarios (Dominio de la fuerza).
	.3 Comida.
	.3 Agua.
	.3 Retos deportivos.
	.3 Retos de felicidad.
	.3 Pruebas de nivel.
	.3 Pruebas de audiencia.
	.2 General.
	.3 Calificaciones (Badges).
	.3 Constantes (Constantes de la app).
	.3 Rangos.
	.3 Crear Padawans (Por lotes de DNI).
	.3 Chat.
	.3 Cartas de privilegio.
	.3 Apuestas.
	.3 Minas.
}

\newpage


\subsection{Funciones de cada sección [Rol=Guidoogway]}
\subsubsection{Dashboard}
Esta vista mostrará un resumen de las demás secciones, esto es; Usuarios, Retos, Información General. Hace la función de acceso directo para las secciones que se nos muestran en el sidebar.

\subsubsection{Distritos}
En esta sección podremos crear nuevos distritos además de editarlos, eliminarlos y mostrar información relevante. Además podremos ingresar TDV a todos los miembros del distrito/clan.

\textbf{Padawans}
En esta sección podremos crear nuevos distritos además de editarlos, eliminarlos y mostrar información relevante.

Como funciones extra podemos destacar:
\begin{itemize}
	\item Visualización de todas las entregas que ha realizado el usuario
	\item Visualización de todos los movimientos bancarios relativos al usuario
	\item Acción de subir de nivel de forma manual si el docente lo considera oportuno
	\item Visualización de todos los resultados deportivos del usuario.
\end{itemize}

Estas secciones anteriormente mencionadas y explicadas a continuación ofrecen información relevante de un usuario al docente cuando se encuentre con el en tutorías.

\textbf{Entregas del usuario}
En esta sección, dentro del directorio del alumno podremos visualizar todas las entregas de todos los tipos de retos que ha realizado el alumno. Como información relevante se mostrará:

\begin{itemize}
	\item El estado en el que está su entrega (Comprado, Entregado, Corregido)
	\item La visualización del fichero entregado (si el estado es Entregado o Corregido)
	\item Información sobre el reto (Enunciado)
	\item Información sobre el tipo de reto (Comida, Agua, Proyecto de Felicidad, Proyecto de innovación, Prueba de nivel, Audiencia ante el Senado Galáctico)
\end{itemize}

\textbf{Movimientos bancarios de un usuario}
Esta sección es un show de los movimientos bancarios que tiene un usuario (Cobros e Ingresos).

\textbf{Subir de nivel a un usuario}
Para que un usuario suba de nivel debe cumplir dos requisitos, haber superado el reto deportivo de la semana y tener un badge mínimo especificado por el docente durante esa semana. Cuando esas dos condiciones se cumplen, el usuario sube de nivel e ingresa una cantidad de XP correspondiente al nivel que tenga. Esta tarea se ejecuta automáticamente al término de cada semana pero el profesor se reserva la opción de poder subir de nivel de forma manual si lo considera oportuno.

\textbf{Datos deportivos de un usuario}
Esta sección muestra un resumen semanal de los resultados deportivos de un usuario en particular. Los datos que se muestran son, por semana, la fase del reto deportivo que se le aplica y si ha superado o no dicha fase.

\subsubsection{Vacaciones}
Esta sección ofrece la posibilidad al docente de hacer un parón de la app en periodos no lectivos. Al asignar un periodo de vacaciones indica a la app que el TDV y las barras de energía no disminuirán durante el periodo especificado.

\subsubsection{Entregas}
Esta sección nos muestra todas las entregas en estado ``Entregado''. De esta forma el docente puede ir calificando esas entregas en una sola sección.

\subsubsection{Cuestionarios}
En esta sección podremos crear nuevos cuestionarios además de editarlos, eliminarlos y mostrar información relevante.\\

Al crear un cuestionario se le cobrará 1 día de vida a cada alumno recuperables si éste responde correctamente al mismo.

\subsubsection{Comida}
En esta sección podremos crear nuevos retos de comida además de editarlos, eliminarlos y mostrar información relevante.\\

\subsubsection{Agua}
En esta sección podremos crear nuevos retos de agua además de editarlos, eliminarlos y mostrar información relevante.\\

\subsubsection{Retos deportivos}
En esta sección podremos crear nuevas fases de retos deportivos además de editarlos, eliminarlos y mostrar información relevante.\\

\subsubsection{Retos de felicidad}
En esta sección podremos crear nuevos retos de felicidad además de editarlos, eliminarlos y mostrar información relevante.\\

\subsubsection{Pruebas de nivel}
En esta sección podremos crear nuevas pruebas de nivel además de editarlos, eliminarlos y mostrar información relevante.\\

\subsubsection{Pruebas de audiencia}
En esta sección podremos crear nuevas pruebas de audiencia además de editarlos, eliminarlos y mostrar información relevante.\\

\subsubsection{Calificaciones (Badges)}
En esta sección podremos crear nuevos badges además de editarlos, eliminarlos y mostrar información relevante.\\

\subsubsection{Constantes}
En esta sección podremos crear nuevas constantes además de editarlos, eliminarlos y mostrar información relevante.\\

Estas constantes nos sirven para modificar el comportamiento de la app, en esta sección podremos especificar: 

\begin{itemize}
	\item el tipo de interés que tienen los préstamos
	\item el número de días que debe transcurrir entre la entrega de un proyecto de felicidad y la entrega de las evidencias.
	\item la recompensa en segundos que se obtendrá al superar una fase deportiva.
	\item la recompensa en segundos por almacenar X Tweets diarios.
	\item el número mínimo de Tweets diarios que debe almacenar cada usuario.
	\item la velocidad a la que disminuye la barra de energía de comida.
	\item la velocidad a la que disminuye la barra de energía de agua.
\end{itemize}

\subsubsection{Rangos}
En esta sección el docente puede definir los rangos que van a existir en la app y la puntuación mínima en el ranking para conseguir ese distintivo.

\subsubsection{Crear Padawans}
En esta sección el docente puede dar de alta usuarios por lotes mediante una lista de DNIs.

\subsubsection{Cartas de privilegios}
En esta sección el docente puede ocultar o mostrar una carta de privilegios previamente implementada. Además de modificar el coste, imagen de la misma.

\subsubsection{Apuestas}
En esta sección el docente puede crear apuestas deportivas ante cualquier torneo deportivo entre clanes. Una vez publicado la apuesta los usuarios podrán apostar TDV en las diferentes opciones.\\

El docente puede cerrar la apuesta para que los alumnos no puedan apostar más. El siguiente paso es marcar la opción correcta y monetizar a los alumnos que hayan acertado.

\subsubsection{Minas}
En esta sección el docente puede crear una mina, especificar un código secreto que deben encontrar los alumnos mediante pistas que agrega el docente a la mina. Las pistas no son visibles a los alumnos a menos que las compren con cartas de privilegios.

\newpage

\subsection{Organización de secciones [Rol=Padawan]}
\dirtree{%
	.1 padawan/home[Rol=Padawan].
	.2 Padawan.
	.3 Hoja de personaje (Datos).
	.3 Rango Jedi (Clasificaciones).
	.3 Nivel - Cartas de privilegio (Cartas de privilegio).
	.3 Historial de aprendizaje (Movimientos bancarios).
	.2 Formación.
	.3 Seguimiento de maestro Jedi (Twitter).
	.3 Pruebas de nivel (Pruebas de superación).
	.3 Dominio de la fuerza (Cuestionarios sorpresa).
	.3 Audiencia ante el Senado Galáctico (Proyectos de innovación).
	.2 Víveres.
	.3 Comida (Retos individuales o por clanes).
	.3 Agua (Retos individuales o por clanes).
	.2 Comunidad de aprendizaje.
	.3 RCI: Red de Comunicación Intergaláctica (Amigos).
	.4 Hologramas (Chat).
	.4 Historial (Galería de fotos).
	.3 Viajes Interplanetarios (Deporte - Runtastic).
	.3 Detección de la fuerza (Apuestas deportivas).
	.3 Altruismo.
	.4 Donación (Donación de TDV).
	.4 Minas (Desactivación de minas).
	.2 Desconexión.
	.3 Cápsula del tiempo (Préstamos).
	.3 Retiro (Vacaciones).
	.2 Dagobah (Tutorías).
	.2 Felicidad (Proyectos de felicidad).
}

\newpage

\subsection{Funciones de cada sección [Rol=Ciudadano]}

\subsubsection{Todas las secciones}

\textbf{Mostrar el TdV}
\begin{figure}[ht]
	\centering
	\includegraphics[width=0.5\textwidth]{imagenes/Objetivo1_new.png}
	\caption{Objetivo 1 - Cuenta regresiva}
	\label{objetivoIm1}
\end{figure}

Es un reloj regresivo que muestra al usuario el TdV que le queda en tiempo real. Como pieza clave del proyecto debe mostrarse en cada una de las secciones que el usuario visualice ya que el padawan tomará sus decisiones en función de éste.\\

%\begin{table}[ht]
%\centering
%\resizebox{\textwidth}{!}{%
%\begin{tabular}{|
%>{\columncolor[HTML]{656565}}r |l|}
%\hline
%\textbf{Objetivo 1}  & \cellcolor[HTML]{9B9B9B}\textit{Mostrar siempre el TdV}                                                                                         \\ \hline
%\textbf{Sección}     & Todas                                                                                                                                           \\ \hline
%\textbf{Descripción} & \begin{tabular}[c]{@{}l@{}}En todas las plantillas se mostrará un reloj regresivo correspondiente \\ al TdV del ciudadano logueado\end{tabular} \\ \hline
%\end{tabular}%
%}
%\caption{Objetivo 1 - Mostrar TdV}
%\label{objetivo1}
%\end{table}

\textbf{Feedback para cada interacción}
\begin{figure}[ht]
	\centering
	\includegraphics[width=0.4\textwidth]{imagenes/Objetivo2_new.png}
	\includegraphics[width=0.4\textwidth]{imagenes/Objetivo2b_new.png}
	\caption{Objetivo 2 - Ejemplos de avisos}
	\label{objetivoIm2}
\end{figure}

Es importante que el alumno pueda visualizar estos avisos allá donde vaya para que sepa qué está ocurriendo en todo momento.

\textbf{Mostrar alertas}

El sistema de alertas permite avisar a un alumno de que se han producido nuevos eventos. Los eventos comunicados mediante este sistema de alertas son los siguientes:\\

\begin{table}[h!]
	\begin{center}
		\begin{tabular}{ c  c  c  c }
			\bottomrule
			\\
			\raisebox{-0.2ex}{\begin{tabular}[c]{@{}l@{}}\includegraphics[width=0.25\textwidth, height=40mm]{imagenes/avisos/dominio_de_la_fuerza.png}\\Avisa de dos\\nuevas publica-\\ciones de dominio\\ de la fuerza.\end{tabular}}
			
			& 
			\raisebox{-0.2ex}{\begin{tabular}[c]{@{}l@{}}\includegraphics[width=0.25\textwidth, height=40mm]{imagenes/avisos/prueba_nivel.png}\\Avisa de la pu-\\blicación de un\\nuevo reto de\\prueba de nivel.\end{tabular}}
			& 
			\raisebox{-0.2ex}{\begin{tabular}[c]{@{}l@{}}\includegraphics[width=0.25\textwidth, height=40mm]{imagenes/avisos/mina.png}\\Avisa de que una\\nueva mina está\\disponible para ser\\ desactivada.\end{tabular}}
			&
			\raisebox{-0.2ex}{\begin{tabular}[c]{@{}l@{}}\includegraphics[width=0.25\textwidth, height=40mm]{imagenes/avisos/chat.png}\\Avisa del número de\\mensajes de chat pen-\\dientes de ser leídos.\\ \\\end{tabular}}
			\\
			\bottomrule
		\end{tabular}
		\caption{Sistema de alertas}
		\label{tbl:Sistema de alertas}
	\end{center}
\end{table}

\newpage

%\textbf\left( {Home}
\begin{figure}[ht]
	\centering
	\includegraphics[width=0.2\textwidth]{imagenes/secciones/padawan/home.png}
	\caption{Objetivo 2 - Ejemplos de avisos}
	\label{objetivoIm2}
\end{figure}



\setcounter{chapter}{2}
\setcounter{section}{0}
\setcounter{subsection}{0}
\chapter{Planificación}

Para este proyecto hemos seguido una metodología de desarrollo XP (eXtreme Programming). En el destacamos simplicidad en el desarrollo, un feedback constante con el usuario (incluso con los usuarios finales) y un desarrollo iterativo e incremental.\\

\section{Trello}

Para el proceso de planificación hemos decidido usar la herramienta online \textbf{trello}, que es un software de administración de proyectos con interfaz web basada en \textit{cartas kanban}. Con esta herramienta dividimos tareas en columnas para crear un flujo de trabajo organizado, la organización de las tareas las dividiremos en las siguientes columnas:

\begin{itemize}
	\item \textbf{Backlog:} En esta categoría agruparemos las tareas o mejoras que no sean urgentes pero estaría bien desarrollar. Estas tareas tienen una prioridad mayormente baja o una fecha de desarrollo estipulada en un futuro próximo.
	\item \textbf{ToDo:} En esta categoría agruparemos las tareas que tengamos pendiente de desarrollo ordenadas por nivel de prioridad de alta a baja.
	\item \textbf{Doing:} En esta categoría agruparemos las tareas que estemos desarrollando. Cada desarrollador tendrá asignada una y solo una tarea bajo esta categoría siempre. Cuando termine esa tarea la pasará a la siguiente categoría y escogerá otra tarea de la categoría \textbf{ToDo}. En este caso, como es un desarrollo de una sola persona siempre habrá una sola tarea en esta categoría.
	\item \textbf{QA:} En esta categoría agruparemos las tareas ya terminadas. La función de esta es la de revisar que el nuevo desarrollo se adapta bien al código de la rama master. Si supera los test se hará un PR y se pasará a la siguiente categoría. En caso contrario esta tarea volverá a la categoría \textbf{Doing}.
	\item \textbf{Done: } En esta categoría agruparemos las tareas terminadas que hayan pasado los test. Se acepta el Pull Request y se hace un merge con la rama master. Esta rama no se desplegará hasta que llegue la fecha final del Sprint.
\end{itemize}

\begin{figure}[ht]
	\centering
	\includegraphics[width=0.8\textwidth]{imagenes/trello.png}
	\caption{Trello}
	\label{trello}
\end{figure}

\section{Sprints}
Los sprint son \textit{periodos de desarrollo} pactados con el cliente en el que definimos un paquete de nuevas funcionalidades. En este proyecto vamos a definir el Sprint tipo 2 + 1, esto es, dos semanas de desarrollo más una semana de testing.

\section{Técnica Pomodoro (Productividad) \cite{Pomodoro}}

En relación a la productividad del desarrollo del proyecto \textbf{SinTime}, se ha usado la técnica \textbf{Pomodoro} que es un método para gestionar el tiempo. Esta técnica se usa para combatir la ansiedad producida por la cantidad de tareas a desarrollar en un corto periodo de tiempo.\\

En lugar de trabajar con prisas a fin de conseguir terminar todo el trabajo en la fecha estipulada, esta técnica usa el tiempo como aliado para alcanzar el objetivo de un modo correcto y nos permite mejorar continuamente la manera en que trabajamos.

\subsection{Los objetivos}

La Técnica \textbf{Pomodoro} nos permite mejorar la productividad mediante un proceso para conseguir:\\

\begin{itemize}
	\item Aliviar la ansiedad a comenzar las tareas.
	\item Aumentar la concentración disminuyendo las interrupciones.
	\item Impulsar la motivación y nos mantiene constantes.
	\item Refina el proceso de estimación.
	\item Mejora el proceso de trabajo.
\end{itemize}

\subsection{Los principios}

Cada día se seleccionan las tareas a completar y las colocamos en la lista \textbf{ToDo} de la que hemos hablado anteriormente. Podemos usar el cronómetro de nuestro reloj o un timer de \textbf{Pomodoro} para, a continuación:

\begin{enumerate}
	\item Comenzar a trabajar: Iniciamos el timer con una duración de 25 minutos, que es el equivalente a una \textbf{sesión Pomodoro}. Una sesión Pomodoro es ininterrumpible por lo que no se debe pausar y reanudar, si se hace una pausa en mitad debe reiniciarse la sesión.
	\item Cuando el timer llegue a cero comenzamos un periodo de descanso llamado \textbf{Short break} de 3 a 5 minutos, después de este descanso se comienza otra sesión Pomodoro.
	\item Cada 4 sesiones Pomodoro completas se realiza un \textbf{Long Break}. Un tiempo entre 15-30 minutos.
	\item Seguir trabajando hasta que la tarea haya terminado.
\end{enumerate}




\setcounter{chapter}{3}
\setcounter{section}{0}
\setcounter{subsection}{0}
\chapter{Presupuestos}

\section{Hosting y Dominio}

En este proyecto se ha contratado un hosting cuyo proveedor es https://www.guebs.com con una licencia de tipo Startup y un registro del dominio https://www.sintime.es con este mismo proveedor durante 2 años. Los gastos generados son los que recoge la tabla a continuación:\\

\begin{table}[h]
	\centering
	\begin{tabular}{| p{10cm} | p{1.6cm} | p{2.3cm} |}
		\rowcolor[HTML]{329A9D} 
		{\color[HTML]{FFFFFF} \textbf{Tipo de artículo}} & {\color[HTML]{FFFFFF} \textbf{Cantidad}} & {\color[HTML]{FFFFFF} \textbf{Total}} \\ \hline
		Hosting Startup (sintime.es) desde 06/09/2017 hasta 06/09/2018 & 1 & 72,45 \euro \\ \hline 
		Renovación de Hosting Startup (sintime.es) desde 06/09/2018 hasta 06/09/2019 & 1 & 72,45 \euro \\ \hline
		Registro de dominio .ES (sintime.es) desde 13/06/2017 hasta 13/06/2018 & 1 & 14,52 \euro \\ \hline
		Renovación de Registro de dominio .ES (sintime.es) desde 13/06/2018 hasta 13/06/2019 & 1 & 14,52 \euro \\ \hline
		 &  & \textbf{173,94 \euro }\\ \hline
	\end{tabular}
\end{table}

\newpage

\section{Salarios}

El coste por hora de un Ingeniero Informático ronda ahora mismo en el mercado laboral sobre los 25\euro/hora. Teniendo en cuenta que el total de horas de desarrollo del proyecto es de 612 horas, entonces:\\

\begin{table}[h]
	\centering
	\begin{tabular}{| p{5cm} | p{2cm} | p{1.6cm} | p{2.3cm} |}
		\rowcolor[HTML]{329A9D} 
		{\color[HTML]{FFFFFF} \textbf{Tipo de Profesional}} & {\color[HTML]{FFFFFF} \textbf{Retribución por horas}} & {\color[HTML]{FFFFFF} \textbf{Total de horas}} & {\color[HTML]{FFFFFF} \textbf{Total}} \\ \hline
		Ingeniero Informático & 612 & 25 \euro & 15.300 \euro \\ \hline
		&  & & \textbf{15.300 \euro }\\ \hline
	\end{tabular}
\end{table}

\section{Equipo de trabajo}

\begin{table}[h]
	\centering
	\begin{tabular}{| p{5cm} | p{2cm} | p{1.6cm} | p{2.3cm} |}
		\rowcolor[HTML]{329A9D} 
		{\color[HTML]{FFFFFF} \textbf{Item}} & {\color[HTML]{FFFFFF} \textbf{Cantidad}} & {\color[HTML]{FFFFFF} \textbf{Coste}} & {\color[HTML]{FFFFFF} \textbf{Total}} \\ \hline
		MSI GV62 7RE & 1 & 1.249 \euro & 1.249 \euro \\ \hline
		&  & & \textbf{1.249 \euro }\\ \hline
	\end{tabular}
\end{table}

\section{Presupuesto total}

\begin{table}[h]
	\centering
	\begin{tabular}{| p{8.6cm} | p{2.3cm} |}
		\rowcolor[HTML]{329A9D} 
		{\color[HTML]{FFFFFF} \textbf{Item}} & {\color[HTML]{FFFFFF} \textbf{Total}} \\ \hline
		Hosting y Dominio & 173,94 \euro \\ \hline
		Equipo de trabajo & 15.300 \euro \\ \hline
		Equipo de trabajo & 1.249 \euro \\ \hline
		& \textbf{16.767,94 \euro }\\ \hline
	\end{tabular}
\end{table}

\setcounter{chapter}{4}
\setcounter{section}{0}
\setcounter{subsection}{0}
\chapter{Plan de entregas}

\section{Breve descripción del alcance del sistema}
El desarrollo del proyecto \textbf{SinTime} consiste en la implementación de una aplicación web cuya implantación se realizará en el marco universitario con el fin de gamificar la experiencia del alumnado mediante la inmersión de ellos en un universo ficticio.\\

Los objetivos más importantes que debe cumplir la aplicación serán:

\begin{itemize}
	\item Debe permitir la administración de todos los usuarios con independencia de su rol.
	\item Debe facilitar al docente la creación de todo tipo de retos para el alumnado.
	\item Debe facilitar al docente la información de los movimientos de un alumno dentro de la app.
	\item Debe permitir al alumno ser evaluado por el docente.
	\item Debe permitir al alumno ser evaluado por el sistema.
	\item El alumno debe poder obtener su calificación.
\end{itemize}

\section{Listado inicial de Historias de Usuario}

A continuación se muestran las Historias de Usuario obtenidas durante las reuniones de planificación y entregas del producto realizadas entre el cliente y el equipo de desarrollo. La lista se divide en 4 partes: un identificador, una descripción de la historia, una estimación en días ideales y una prioridad. La prioridad se medirá por el cliente en un rango de 0 a 100, siendo 100 la prioridad más alta.\\

\begin{table}[h]
	\centering
	\begin{tabular}{| p{2.3cm} | p{5.1cm} | p{2cm} | p{1.6cm} |}
		\rowcolor[HTML]{329A9D} 
		{\color[HTML]{FFFFFF} \textbf{Identificador}} & {\color[HTML]{FFFFFF} \textbf{Historias de Usuario}} & {\color[HTML]{FFFFFF} \textbf{Estimación}} & {\color[HTML]{FFFFFF} \textbf{Prioridad}} \\ \hline
		HU.1 & Ejemplo & 1 & 100 \\         
		\hline              
	\end{tabular}
\end{table}

\section{Cálculo de la velocidad del equipo}

El equipo inicial está formado por un programador con una dedicación al proyecto del 50\%. La duración de cada Sprint será, como ya se ha comentado, de 3 semanas de las cuales dos semanas se destinarán a desarrollo y una a testing.\\

La estimación está basada en \textit{días ideales}. Un día ideal se compone de 4 horas de trabajo por lo que el resultado en horas de cada  sprint será:\\

\[
1 \textrm{Sprint} = 3 \textrm{Semanas} \quad \textrm{y} \quad 1 \textrm{Semana} = 5 \textrm{días ideales}
\]

\[
1 \textrm{día ideal} = 4 \textrm{horas reales} \quad \textrm{entonces} \quad 1 \textrm{Spint} = 60 \textrm{horas reales}
\]

\section{Descripción de las entregas}

\textbf{Esfuerzo total del proyecto} = X PH
\textbf{Duración del proyecto} = X meses
\textbf{Velocidad del equipo} = X a Y PH

En base a las estimaciones de velocidad del equipo y al esfuerzo necesario para el desarrollo se ha decidido realizar X entregas. La semana laboral dedicada al proyecto se dividirá en 5 días (Lunes a Viernes) con una duración de 4 horas diarias que suman un total de 20 horas semanales. El desarrollo del proyecto comenzará el día \textbf{1 de Octubre de 2017}.\\


El plan de entregas es el siguiente:

\begin{table}[h]
	\centering
	\begin{tabular}{| p{1.4cm} | p{6.7cm} | p{3.2cm} |}
		\rowcolor[HTML]{329A9D} 
		{\color[HTML]{FFFFFF} \textbf{Entrega}} & {\color[HTML]{FFFFFF} \textbf{Objetivo}} & {\color[HTML]{FFFFFF} \textbf{Fecha de entrega}} \\    
		Sprint 1 & Ejemplo largo Ejemplo largo Ejemplo largo Ejemplo largo  & 20 Octubre 2017 \\ \hline
		Sprint 2 & Ejemplo largo Ejemplo largo Ejemplo largo Ejemplo largoEjemplo largo Ejemplo largo Ejemplo largo Ejemplo largo  & 10 Noviembre 2017 \\ 
		\hline              
	\end{tabular}
\end{table}

\section{Lista Inicial del Producto}

La lista del producto, con las historias que se usaran en el inicio del desarrollo es la siguiente:\\

\begin{table}[h]
	\centering
	\begin{tabular}{| p{2.3cm} | p{5.1cm} | p{2cm} | p{1.6cm} |}
		\rowcolor[HTML]{329A9D} 
		{\color[HTML]{FFFFFF} \textbf{Identificador}} & {\color[HTML]{FFFFFF} \textbf{Historias de Usuario}} & {\color[HTML]{FFFFFF} \textbf{Estimación}} & {\color[HTML]{FFFFFF} \textbf{Prioridad}} \\ \hline
		HU.1 & Ejemplo & 1 & 100 \\         
		\hline              
	\end{tabular}
\end{table}

\setcounter{chapter}{5}
\setcounter{section}{0}
\setcounter{subsection}{0}
\chapter{Tecnologías usadas}

En este proyecto se ha hecho uso de herramientas externas que nos han ayudado en diferentes aspectos durante el desarrollo del proyecto. A continuación vamos a nombrar algunas de ellas.

\newpage
\section{Google Analytics}
Esta famosa herramienta nos va a ofrecer un feedback a tiempo real del tráfico y flujo de usuarios que interactúan con nuestra plataforma así como análisis y comparativas de esos flujos con respecto a diferentes períodos de tiempo.

\begin{figure}[ht]
	\centering
	\includegraphics[width=0.6\textwidth]{imagenes/tecnologias/analytics.png}
	\caption{Google Analytics}
	\label{analytics}
\end{figure}

\newpage
\section{Rollbar}
Rollbar es un sistema de detección y diagnóstico de errores en tiempo real. En este proyecto está integrado mediante una \textbf{gema}, siguiendo una pequeña guía podremos conectar nuestra aplicación con su servicio. \cite{Rollbar1}.\\
Debemos tener presente en todo momento que el ecosistema tecnológico está en continuo cambio por lo que cualquiera de estos podría afectar al correcto funcionamiento de nuestra plataforma, así que, una vez publiquemos nuestra aplicación deberíamos estar alerta de todos los errores que se suceden. Por ello, esta herramienta nos va a dar un feedback en tiempo real de los errores que sucedan así como entender cómo ocurren, por qué y dónde.\\

A continuación vamos a poner un ejemplo de un error en tiempo real y la potencia de esta herramienta.

\begin{figure}[ht]
	\centering
	\includegraphics[width=0.6\textwidth]{imagenes/tecnologias/rollbar_error_mail.png}
	\caption{Rollbar - Notificación email}
	\label{rollbar_error_mail}
\end{figure}

\newpage

Si seguimos en enlace que nos marca la url podremos ver todos los detalles del error como la traza, las ocurrencias, navegadores y OS afectados y más detalles.\\


\begin{figure}[ht]
	\centering
	\includegraphics[width=0.4\textwidth]{imagenes/tecnologias/rollbar_error_traceback.png}
	\caption{Rollbar - Rollbar detalle Traceback}
	\label{rollbar_error_traceback}
\end{figure}

\begin{figure}[ht]
	\centering
	\includegraphics[width=0.4\textwidth]{imagenes/tecnologias/rollbar_error_ocurrences.png}
	\caption{Rollbar - Rollbar detalle Ocurrences}
	\label{rollbar_error_ocurrences}
\end{figure}

\begin{figure}[ht]
	\centering
	\includegraphics[width=0.4\textwidth]{imagenes/tecnologias/rollbar_error_browser_os.png}
	\caption{Rollbar - Rollbar detalle Browsers/OS}
	\label{rollbar_error_browser_os}
\end{figure}

\newpage

\section{Port Monitor \cite{PortMonitor}}
Servicio web creado por un \textbf{antiguo estudiante de la UGR}, Francisco Yañez. Port Monitor es una herramienta fácil y en línea que supervisa el monitoreo del sitio web y del servidor para los usuarios las 24 horas del día, los 7 días de la semana. Registra el tiempo de actividad, los tiempos de respuesta del sitio web / servidor (rendimiento) y las causas del tiempo de inactividad. Genera informes personalizados y envía alertas instantáneas por correo electrónico o informes semanales / mensuales.

\section{Runtastic}
Ya que uno de los principales objetivos de la plataforma es mejorar el ritmo cardiorrespiratorio de los usuarios del sistema, debemos tener un servicio que monitorice los retos enfocados a los ejercicios físicos. En este proyecto actualizamos continuamente las sesiones que realizan cada uno de los padawans para evaluar semanalmente los resultados obtenidos.\\

Para ello necesitamos que el usuario nos facilite su usuario y contraseña de Runtastic, los almacenamos en la base de datos

\newpage

\section{Twitter}
En esta aplicación hacemos uso de la API de Twitter ya que uno de los retos diarios consiste en almacenar 5 tweets diarios relevantes en el mundo de la Educación Física. Para ello tenemos que servir los tweets de usuarios de Twitter a los que los padawans sigan.\\


\begin{figure}[ht]
	\centering
	\includegraphics[width=1\textwidth]{imagenes/tecnologias/twitter.png}
	\caption{Twitter - Ejemplo}
	\label{twitter}
\end{figure}

\setcounter{chapter}{6}
\setcounter{section}{0}
\setcounter{subsection}{0}
\chapter{Diseño}
\section{Diagrama de clases}

\newpage
\includepdf{misc/class_diagram.pdf}

\setcounter{chapter}{7}
\setcounter{section}{0}
\setcounter{subsection}{0}
\chapter{Análisis}
\section{Historias de usuario}

\begin{longtable} {r l c c c}
	\hline
	\#	&	\textbf{Descripción}					&	\textbf{Dep.}	&	\textbf{Est.}	&	\textbf{Prio.}	\\
	\hline \hline
	\endhead
	\multicolumn{5}{l}{\textbf{Servidor de producción}} \\
	\hline 
	1.1.	&	Instalación de Rails				&	-				&	5				&	1	\\
	\hline
	1.2.	&	Configuración de la Base de Datos	&	1.1				&	1				&	1	\\
	\hline
	1.3.	&	Git ignore							&	1.1	1.2			&	1				&	1	\\
	\hline	
	1.4.	&	Sistema de alertas (\textbf{Rollbar})&	1.1				&	4				&	2	\\
	\hline	
	1.5.	&	Desplegar cambios locales a producción&	1.1	1.2 1.4		&	5				&	1	\\
	\hline	
	1.6.	&	Sistema de alertas (\textbf{Port Monitor})&	1.1			&	1				&	3	\\
	\hline
	\multicolumn{5}{l}{\textbf{Sistema}} \\
	\hline 
	2.1.	&	Configuración de namespaces				&	-				&	1				&	1	\\
	\hline
	2.2.	&	Gestión de sesiones	con \textbf{Devise}	&	-			&	4				&	3	\\
	\hline
	2.3.	&	Configuración de mailing			&	-				&	8				&	3	\\
	\hline
	\multicolumn{5}{l}{\textbf{Alumnos}} \\
	\hline 
	3.1.	&	Reloj regresivo						&	-				&	2				&	1	\\
	\hline
	\multicolumn{5}{l}{\textbf{Administración}} \\
	\hline 
	4.1.	&	Creación de usuarios(Lista DNIs)	&	-				&	4				&	1	\\
	\hline
	4.2.	&	Creación de usuarios				&	4.1.			&	4				&	3	\\
	\hline
	4.3.	&	Gestión de clanes					&	4.1.			&	2				&	2	\\
	\hline
	\\
	\caption{Historias de usuario}
	\label{tab:analisis/hus}
\end{longtable}

\newpage

\section{Tarjetas de Servidor de Producción}

\subsection{Instalación de Ruby on Rails}

\begin{table}[h]
	\begin{center}
		\begin{tabular} {l|c|l}
			\hline
			1.1. & \multicolumn{2}{c}{Instalación de Ruby on Rails} \\ \noalign{\hrule height 1pt}
			\multicolumn{3}{l}{Descripción} \\ \hline
			\multicolumn{3}{p{12cm}}{
				Se debe instalar en el servidor el Framework Ruby on Rails. Proveeremos también de datos todos los ficheros de configuración del framework. 
				\begin{itemize}
					\item \textbf{secrets.yml}: que contendrá todas las contraseñas, o referencias a ellas, que necesite el sistema.
					\item \textbf{database.yml}: que contendrá toda la información necesaria para acceder a la base de datos que hemos configurado en el servidor de producción.
				\end{itemize}
			} \\ \noalign{\hrule height 1pt}
			\multicolumn{2}{l|}{Estimación} & 5 \\ \hline
			\multicolumn{2}{l|}{Prioridad} & 1 \\ \hline
			\multicolumn{2}{l|}{Dependencias} & - \\ \noalign{\hrule height 1pt}
			\multicolumn{3}{l}{Pruebas de aceptación} \\ \hline
			\multicolumn{3}{p{12cm}}{ - } \\
			\hline
		\end{tabular}
	\end{center}
	\caption{Historia de usuario - Instalación de Ruby on Rails}
	\label{tab:analisis/instalacion-ruby-on-rails}
\end{table}

\subsection{Configuración de la Base de Datos}

\begin{table}[h]
	\begin{center}
		\begin{tabular} {l|c|l}
			\hline
			1.2. & \multicolumn{2}{c}{Configuración de la Base de Datos} \\ \noalign{\hrule height 1pt}
			\multicolumn{3}{l}{Descripción} \\ \hline
			\multicolumn{3}{p{12cm}}{
				Crearemos la base de datos del servidor de producción y obtendremos todos los datos necesarios para conectar con ella. La base de datos será de tipo MySQL.
			} \\ \noalign{\hrule height 1pt}
			\multicolumn{2}{l|}{Estimación} & 1 \\ \hline
			\multicolumn{2}{l|}{Prioridad} & 1 \\ \hline
			\multicolumn{2}{l|}{Dependencias} & 1.1 \\ \noalign{\hrule height 1pt}
			\multicolumn{3}{l}{Pruebas de aceptación} \\ \hline
			\multicolumn{3}{p{12cm}}{ - } \\
			\hline
		\end{tabular}
	\end{center}
	\caption{Historia de usuario - Configuración de la Base de Datos}
	\label{tab:analisis/configuracion-de-la-base-de-datos}
\end{table}

\subsection{Configuración de gitignore}

\begin{table}[h]
	\begin{center}
		\begin{tabular} {l|c|l}
			\hline
			1.3. & \multicolumn{2}{c}{Configuración de gitignore} \\ \noalign{\hrule height 1pt}
			\multicolumn{3}{l}{Descripción} \\ \hline
			\multicolumn{3}{p{12cm}}{
				Configuraremos el fichero \textit{gitignore} para que ignore los cambios en los ficheros de configuración anteriores. De esta forma evitaremos que se sustituyan estos ficheros de la parte local con los del servidor de producción, donde estarán las configuraciones reales. 
			} \\ \noalign{\hrule height 1pt}
			\multicolumn{2}{l|}{Estimación} & 1 \\ \hline
			\multicolumn{2}{l|}{Prioridad} & 1 \\ \hline
			\multicolumn{2}{l|}{Dependencias} & 1.1 1.2 \\ \noalign{\hrule height 1pt}
			\multicolumn{3}{l}{Pruebas de aceptación} \\ \hline
			\multicolumn{3}{p{12cm}}{ - } \\
			\hline
		\end{tabular}
	\end{center}
	\caption{Historia de usuario - Configuración de gitignore}
	\label{tab:analisis/configuracion-de-gitignore}
\end{table}

\subsection{Sistema de alertas Rollbar}

\begin{table}[h]
	\begin{center}
		\begin{tabular} {l|c|l}
			\hline
			1.4. & \multicolumn{2}{c}{Sistema de alertas (\textbf{Rollbar})} \\ \noalign{\hrule height 1pt}
			\multicolumn{3}{l}{Descripción} \\ \hline
			\multicolumn{3}{p{12cm}}{
				Haremos la conexión de nuestro sistema con el servicio Rollbar para recibir alertas sobre errores en tiempo real. Para ello tenemos que registrarnos en su página web a través de la url \cite{RollbarSingUp} y seguiremos la guía de configuración facilitada en \cite{Rollbar1}.
			} \\ \noalign{\hrule height 1pt}
			\multicolumn{2}{l|}{Estimación} & 1 \\ \hline
			\multicolumn{2}{l|}{Prioridad} & 1 \\ \hline
			\multicolumn{2}{l|}{Dependencias} & 1.1 \\ \noalign{\hrule height 1pt}
			\multicolumn{3}{l}{Pruebas de aceptación} \\ \hline
			\multicolumn{3}{p{12cm}}{ - } \\
			\hline
		\end{tabular}
	\end{center}
	\caption{Historia de usuario - Sistema de alertas (\textbf{Rollbar})}
	\label{tab:analisis/sistema-de-alertas-rollbar}
\end{table}

\newpage

\subsection{Desplegar cambios locales a producción}

\begin{table}[h]
	\begin{center}
		\begin{tabular} {l|c|l}
			\hline
			1.5. & \multicolumn{2}{c}{Desplegar cambios locales a producción} \\ \noalign{\hrule height 1pt}
			\multicolumn{3}{l}{Descripción} \\ \hline
			\multicolumn{3}{p{12cm}}{
				Configuraremos todas las acciones que va a realizar el servidor cuando demos la orden de despliegue. Para ello haremos uso de la gema Capistrano. Las acciones que vamos a realizar serán, por orden, las siguientes:
				\begin{itemize}
					\item Conectará con la rama \textbf{master} de nuestro proyecto en github y lo descargará.
					\item Instalará las nuevas gemas que hayamos incluido
					\item Compilará los assets y los comprimirá
					\item Ejecutará las migraciones pendientes de ejecutar
					\item Cambiará el puntero del directorio principal de la web para que apunte a los nuevos cambios
					\item Reiniciará la Web App
				\end{itemize}
			
				Si al ejecutar el despliegue se producen errores no se producirán cambios en el servidor de producción y \textbf{rollbar} nos notificará el error vía email.
			} \\ \noalign{\hrule height 1pt}
			\multicolumn{2}{l|}{Estimación} & 5 \\ \hline
			\multicolumn{2}{l|}{Prioridad} & 1 \\ \hline
			\multicolumn{2}{l|}{Dependencias} & 1.1 1.2 1.4 \\ \noalign{\hrule height 1pt}
			\multicolumn{3}{l}{Pruebas de aceptación} \\ \hline
			\multicolumn{3}{p{12cm}}{ - } \\
			\hline
		\end{tabular}
	\end{center}
	\caption{Historia de usuario - Desplegar cambios locales a producción}
	\label{tab:analisis/desplegar-cambios-locales-a-produccion}
\end{table}

\newpage
\subsection{Sistema de alertas Port Monitor}

\begin{table}[h]
	\begin{center}
		\begin{tabular} {l|c|l}
			\hline
			1.6. & \multicolumn{2}{c}{Sistema de alertas {Port Monitor}} \\ \noalign{\hrule height 1pt}
			\multicolumn{3}{l}{Descripción} \\ \hline
			\multicolumn{3}{p{12cm}}{
				Haremos la conexión de nuestro sistema con el servicio Port Monitor para recibir alertas sobre el estado de nuestro dominio. Para ello nos registraremos en su página web a través de la url \cite{PortMonitorSingUp} y seguiremos la navegación normal para que haga cheking de nuestro dominio.
			} \\ \noalign{\hrule height 1pt}
			\multicolumn{2}{l|}{Estimación} & 1 \\ \hline
			\multicolumn{2}{l|}{Prioridad} & 1 \\ \hline
			\multicolumn{2}{l|}{Dependencias} & 1.1 \\ \noalign{\hrule height 1pt}
			\multicolumn{3}{l}{Pruebas de aceptación} \\ \hline
			\multicolumn{3}{p{12cm}}{ - } \\
			\hline
		\end{tabular}
	\end{center}
	\caption{Historia de usuario - Sistema de alertas {Port Monitor}}
	\label{tab:analisis/sistema-de-alertas-port-monitor}
\end{table}

\newpage

\section{Sistema}

\subsection{Configuración de namespaces}

\begin{table}[h]
	\begin{center}
		\begin{tabular} {l|c|l}
			\hline
			2.1. & \multicolumn{2}{c}{Configuración de namespaces} \\ \noalign{\hrule height 1pt}
			\multicolumn{3}{l}{Descripción} \\ \hline
			\multicolumn{3}{p{12cm}}{
				Haremos una separación en la raíz del proyecto a partir de la cual van a crecer nuestros diferentes layouts. El sistema lo vamos a dividir en 3 partes: 
				\begin{enumerate}
					\item \textbf{Padawan}: Será la sección a la que podrán acceder sólo los \textit{usuarios de tipo estudiante}.
					\item \textbf{Manager}: Será la sección a la que podrán acceder sólo los \textit{usuarios de tipo admin}, es la sección en la que \textbf{el profesor} podrá crear contenido para los estudiantes y gestionar gran parte de las funciones de la aplicación.
					\item \textbf{Super Admin}: Será la sección a la que podrán acceder sólo los \textit{usuarios de tipo admin}. En ella podremos gestionar todos los aspectos posibles de la aplicación. Esta sección está reservada para el mantenimiento de la aplicación.
				\end{enumerate}
			} \\ \noalign{\hrule height 1pt}
			\multicolumn{2}{l|}{Estimación} & 1 \\ \hline
			\multicolumn{2}{l|}{Prioridad} & 1 \\ \hline
			\multicolumn{2}{l|}{Dependencias} & - \\ \noalign{\hrule height 1pt}
			\multicolumn{3}{l}{Pruebas de aceptación} \\ \hline
			\multicolumn{3}{p{12cm}}{ - Ninguna de estas secciones podrá ser pública.} \\ 
			\multicolumn{3}{p{12cm}}{ - Sólo se podrá acceder a cada una de las secciones previo registro.} \\
			\hline
		\end{tabular}
	\end{center}
	\caption{Historia de usuario - Configuración de namespaces}
	\label{tab:analisis/configuracion-de-namespaces}
\end{table}


%
%\input{capitulos/07_Implementacion}
%
%\input{capitulos/08_Pruebas}
%
%\setcounter{chapter}{9}
\setcounter{section}{0}
\setcounter{subsection}{0}
\chapter{Conclusiones}

Tras la realización del trabajo y posterior prueba en aulas reales durante dos cursos consecutivos puedo afirmar que la gamificación de una asignatura estimula al alumnado y despierta en éstos un interés por participar. Durante el desarrollo de los cursos se aprecia una creciente interacción entre los alumnos y los retos propuestos por el docente.\\

Por otro lado, durante dos años consecutivos se ha abordado un problema real en las instituciones educativas de manera que se demuestra que el uso de las tecnologías se adaptan completamente al ámbito educativo.

\section{Líneas futuras}

Como próximos desarrollos que me gustaría implementar en el proyecto \textbf{SinTime} destacaría:

\begin{enumerate}
	\item Desarrollar una versión móvil
	\item Apuestas
	\item Tutorías
	\item Romper el sistema
\end{enumerate}

\subsubsection{Desarrollar una versión móvil}

Todos conocemos la creciente demanda de aplicaciones móviles por lo que sería una evolución natural desarrollar una versión para dispositivos móviles.\\

Para llegar a ese punto se debe realizar un nuevo desarrollo de una API en el entorno del servidor. De esta forma, cualquier desarrollo podría consumir de este servicio.\\

Por otro lado el desarrollo móvil podría realizarse en Ionic para dar soporte a diferentes plataformas como Android y Iphone.

\subsubsection{Apuestas}

No sigue una línea educativa pero realmente expandería la gama de toma de decisiones dentro del entorno SinTime. Cada semana el sistema calcularía un número al azar del 1 al 20. El usuario Padawan que lo desee podría comprar un numero durante la semana por una cantidad de TdV y, al final de la semana, el sistema abonaría el bote recaudado a los ganadores.\\

\subsubsection{Tutorías}

Una sección bastante útil para el docente. La gestión de tutorías desde la misma plataforma sería bastante útil ya que, si la asignatura se basa en la aplicación, podría tener toda la información del alumno en una vista de todo el desarrollo del alumno en la aplicación.\\

El docente podría marcar los huecos libres que tendría durante la semana para ofrecer tutorías. \\
\\
El alumno podría visualizar el horario con esos huecos y distinguir cuáles de ellos están disponibles o están ya ocupados por otro alumno. Podría a su vez solicitar un slot de tiempo para una cita que el tutor tendría que aceptar posteriormente y ese hueco de tiempo tendría un coste simbólico en TdV.\\

Por otro lado el docente podría aceptar esa cita y devolver esa cantidad de TdV pagada por el alumno si la tutoría ha merecido la pena.\\

\subsubsection{Romper el sistema}

Como punto final, el docente podría marcar el final de una etapa docente con un circuito de retos que el alumno podrá completar opcionalmente. Es más bien un reto global para todos los alumnos.\\

El sistema iría acumulando a lo largo del curso todo el TdV cobrado a los alumnos en un \textbf{Banco del Tiempo}. Una vez abierto el circuito, el que lo complete podrá acceder a un apartado donde se mostrará todo el TdV acumulado por el sistema a lo largo del curso. Este TdV del Banco del Tiempo estará dividido en slots de TdV equitativos y el usuario que acceda podrá repartir cada slot con usuarios diferentes incluyéndose a sí mismo.

\setcounter{chapter}{8}
\setcounter{section}{0}
\setcounter{subsection}{0}
\chapter{Prensa}

\section{Prensa digital}

\begin{vtimeline}[description={text width=20cm}, 
	row sep=4ex, 
	use timeline header,
	timeline pretitle={Prensa},
	timeline title={Digital},
	timeline color=cyan!80!red, add bottom line, line offset=8pt
	]
	20/05/2017 & \textbf{Granada Hoy}: \url{http://www.granadahoy.com/vivir/In-Time-mejora-fitness-cardiorrespiratorio\_0\_1148885257.html} \endlr
	16/06/2017 & \textbf{Canal Sur Andalucía}: \url{http://www.canalsur.es/noticias/tecnicas-innovadoras-que-aumentan-el-rendimiento-universitario/1177803.html}\endlr
	26/06/2017 & \textbf{Canal UGR}: \url{https://canal.ugr.es/noticia/proyecto-gamificacion-in-time} \endlr
	28/06/2017 & \textbf{Aula Magna}: \url{http://www.aulamagna.com.es/in-time-innovacion-docente-granada} \endlr
	28/06/2017 & \textbf{Info Costa Tropical}: \url{http://www.infocostatropical.com/mb/noticia.asp?id\_noticia=76610} \endlr
	07/07/2017 & \textbf{Esto no es Finlandia}: \url{https://estonoesfinlandia.wordpress.com/2017/07/07/in-time} \endlr
	23/05/2018 & \textbf{Canal UGR}: \url{https://canal.ugr.es/noticia/los-primeros-jedi-ugr} \endlr
	28/05/2018 & \textbf{Granada Hoy}: \url{http://www.granadahoy.com/vivir/Star-Wars-Episodio-Primeros-Jedi\_0\_1248175685.html} \endlr
	28/05/2018 & \textbf{Europa press}: \url{http://www.europapress.es/andalucia/noticia-profesor-facultad-ciencias-deporte-granada-utiliza-star-wars-mejorar-motivacion-20180523203636.html} \endlr
	28/05/2018 & \textbf{Ideal}: \url{http://www.ideal.es/miugr/profesor-facultad-ciencias-20180523122848-nt.html} \endlr
	28/05/2018 & \textbf{Ahora Granada}: \url{https://www.ahoragranada.com/noticia/los-primeros-jedi-de-la-facultad-de-ciencias-del-deporte} \endlr
\end{vtimeline}

\newpage

\section{Radio}

\begin{vtimeline}[description={text width=20cm}, 
	row sep=4ex, 
	use timeline header,
	timeline pretitle={Prensa},
	timeline title={Radio},
	timeline color=red!80!red, add bottom line, line offset=8pt
	]
	24/05/2018 & \textbf{Canal Sur Radio}: \url{https://drive.google.com/open?id=1n\_DUIsL96aHi8KbvjBdT81gOmWLVccXU}\endlr
\end{vtimeline}

\newpage

\section{Prensa escrita}

\subsection{27/06/2017 - Granada Hoy}
\begin{figure}[ht]
	\centering
	\includegraphics[width=0.8\textwidth]{prensa/GRANADA_HOY_27_06_2017_b.png}
	\caption{Prensa 27/06/2017 - Granada Hoy}
	\label{prensa1}
\end{figure}

\newpage

\subsection{28/06/2017 - Ideal}
\begin{figure}[ht]
	\centering
	\includegraphics[width=0.8\textwidth]{prensa/IDEAL_28_06_2017_b.png}
	\caption{Prensa 28/06/2017 - Ideal}
	\label{prensa2}
\end{figure}

\newpage

\subsection{27/07/2017 - Ideal}
\begin{figure}[ht]
	\centering
	\includegraphics[width=0.8\textwidth]{prensa/IDEAL_27_07_2017_b.png}
	\caption{Prensa 27/07/2017 - Ideal}
	\label{prensa3}
\end{figure}

\newpage




%%\chapter{Conclusiones y Trabajos Futuros}
%
%
%%\nocite{*}
%\bibliography{bibliografia/bibliografia}\addcontentsline{toc}{chapter}{Bibliografía}
%\bibliographystyle{miunsrturl}
%
%\appendix
%\input{apendices/manual_usuario/manual_usuario}
%%\input{apendices/paper/paper}
%\input{glosario/entradas_glosario}
% \addcontentsline{toc}{chapter}{Glosario}
% \printglossary

\chapter*{}
\thispagestyle{empty}
\newpage
\bibliographystyle{acm} % hay varias formas de citar
\bibliography{citas} %archivo citas.bib que contiene las entradas 

\end{document}
