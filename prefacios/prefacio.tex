\chapter*{}
%\thispagestyle{empty}
%\cleardoublepage

%\thispagestyle{empty}

\input{portada/portada_2}



\cleardoublepage
\thispagestyle{empty}

\begin{center}
{\large\bfseries Proyecto \$inTime: Aplicación Web Gamificada de Docencia }\\
\end{center}
\begin{center}
Álvaro Fernández-Alonso Araluce (alumno)\\
\end{center}

%\vspace{0.7cm}
\noindent{\textbf{Palabras clave}: \$inTime, docencia, gamificación }\\

\vspace{0.7cm}
\noindent{\textbf{Resumen}}\\

En este documento se desarrollará una idea dirigida a incrementar la atención, participación, dedicación y esfuerzo por parte del alumnado en sus obligaciones como estudiante.\\

Las actividades lúdicas provocan que focalicemos toda nuestra atención en superar retos de forma constante provocando incluso, por medio de historias; escenarios y experiencias, la inmersión total de la persona que las realice. La idea consiste en extraer todo ese mecanismo e implantarlo en la educación haciendo que los retos a superar sean retos educativos.\\

En este trabajo se presenta un escenario inmersivo en el que científicos han conseguido detener el gen de crecimiento. Dado que ya nadie fallece por causas naturales se implanta en cada persona un reloj regresivo con 10 años de vida. A partir de ese momento el trabajo se cobra en tiempo, los alimentos cuestan tiempo, toda la economía se basa en el tiempo. Si un reloj llega a cero provocará un infarto instantáneo.\\

El escenario anteriormente descrito es una breve sinopsis de la película \textbf{InTime} del director neozelandés \textbf{Andrew Niccol} y es en esta trama donde van a verse inmersos \textbf{los estudiantes de Fundamentos de la Educación Física} del curso 2016/17 por medio de una \textbf{Aplicación Web} que va a simular este escenario.\\

En este caso cada estudiante (ciudadano de \$inTime a partir de ahora) va a comenzar su experiencia con 15 días de vida en distritos independientes, cada segundo que pasa es un segundo menos que tienen y tendrán que realizar diversos retos para obtener más tiempo.

\cleardoublepage


\thispagestyle{empty}


\begin{center}
{\large\bfseries \$inTime Project: Gamified Web Application for Teaching}\\
\end{center}
\begin{center}
Álvaro Fernández-Alonso Araluce (alumno)\\
\end{center}

%\vspace{0.7cm}
\noindent{\textbf{Keywords}: \$inTime, Gamification, Teaching}\\

\vspace{0.7cm}
\noindent{\textbf{Abstract}}\\

Write here the abstract in English.

\chapter*{}
\thispagestyle{empty}

\noindent\rule[-1ex]{\textwidth}{2pt}\\[4.5ex]

Yo, \textbf{Álvaro Fernández-Alonso Araluce (alumno)}, alumno de la titulación TITULACIÓN de la \textbf{Escuela Técnica Superior
de Ingenierías Informática y de Telecomunicación de la Universidad de Granada}, con DNI 75167394J, autorizo la
ubicación de la siguiente copia de mi Trabajo Fin de Grado en la biblioteca del centro para que pueda ser
consultada por las personas que lo deseen.

\vspace{6cm}

\noindent Fdo: Álvaro Fernández-Alonso Araluce (alumno)

\vspace{2cm}

\begin{flushright}
Granada a Mayo de mes de 2017.
\end{flushright}


\chapter*{}
\thispagestyle{empty}

\noindent\rule[-1ex]{\textwidth}{2pt}\\[4.5ex]

D. \textbf{Nombre Apellido1 Apellido2 (tutor1)}, Profesor del Área de XXXX del Departamento YYYY de la Universidad de Granada.

\vspace{0.5cm}

D. \textbf{Nombre Apellido1 Apellido2 (tutor2)}, Profesor del Área de XXXX del Departamento YYYY de la Universidad de Granada.


\vspace{0.5cm}

\textbf{Informan:}

\vspace{0.5cm}

Que el presente trabajo, titulado \textit{\textbf{Título del proyecto, Subtítulo del proyecto}},
ha sido realizado bajo su supervisión por \textbf{Nombre Apellido1 Apellido2 (alumno)}, y autorizamos la defensa de dicho trabajo ante el tribunal
que corresponda.

\vspace{0.5cm}

Y para que conste, expiden y firman el presente informe en Granada a X de mes de 201 .

\vspace{1cm}

\textbf{Los directores:}

\vspace{5cm}

\noindent \textbf{Nombre Apellido1 Apellido2 (tutor1) \ \ \ \ \ Nombre Apellido1 Apellido2 (tutor2)}

\chapter*{Agradecimientos}
\thispagestyle{empty}

       \vspace{1cm}


A \textbf{Soraya} por su apoyo incondicional.\\

A \textbf{Juan Manuel Fernández Luna} por su paciencia y consejos.\\

A \textbf{Isaac José Pérez López} por contar conmigo para desarrollar una idea tan original y a \textbf{sus alumnos de Fundamentos de la Educación Física} por la experiencia vivida.

