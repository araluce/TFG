\chapter*{}
%\thispagestyle{empty}
%\cleardoublepage

%\thispagestyle{empty}

\input{portada/portada_2}



\cleardoublepage
\thispagestyle{empty}

\begin{center}
{\large\bfseries Proyecto SinTime: Aplicación Web Gamificada de Docencia }\\
\end{center}
\begin{center}
Álvaro Fernández-Alonso Araluce\\
\end{center}

%\vspace{0.7cm}
\noindent{\textbf{Palabras clave}: SinTime, \#SinTime\_wARs, docencia, gamificación }\\

\vspace{0.7cm}
\noindent{\textbf{Resumen}}\\

En este documento se desarrollará una idea dirigida a incrementar la atención, participación, dedicación y esfuerzo por parte del alumnado en sus obligaciones como estudiante.\\

Las actividades lúdicas provocan que focalicemos toda nuestra atención en superar retos de forma constante provocando incluso, por medio de historias; escenarios y experiencias, la inmersión total de la persona que las realice. La idea consiste en extraer todo ese mecanismo e implantarlo en la educación haciendo que los retos a superar sean retos educativos.\\

En este trabajo se presenta un escenario inmersivo en el que científicos han conseguido detener el gen de crecimiento. Dado que ya nadie fallece por causas naturales se implanta en cada persona un reloj regresivo con 10 años de vida. A partir de ese momento el trabajo se cobra en tiempo, los alimentos cuestan tiempo, toda la economía se basa en el tiempo. Si un reloj llega a cero provocará un infarto instantáneo.\\

El escenario anteriormente descrito es una breve sinopsis de la película \textbf{InTime} del director neozelandés \textbf{Andrew Niccol} y es en esta trama donde van a verse inmersos \textbf{los estudiantes de Fundamentos de la Educación Física} de los cursos 2016/2017 \& 2017/2018 por medio de una \textbf{Aplicación Web} que va a simular este escenario.\\

En este caso cada estudiante (ciudadano de SinTime a partir de ahora) va a comenzar su experiencia con 15 días de vida en distritos independientes, cada segundo que pasa es un segundo menos que tienen y tendrán que realizar diversos retos para obtener más tiempo.

\cleardoublepage


\thispagestyle{empty}


\begin{center}
{\large\bfseries \$inTime Project: Gamified Web Application for Teaching}\\
\end{center}
\begin{center}
Álvaro Fernández-Alonso Araluce\\
\end{center}

%\vspace{0.7cm}
\noindent{\textbf{Keywords}: SinTime, \#SinTime\_wARs, Gamification, Teaching}\\

\vspace{0.7cm}
\noindent{\textbf{Abstract}}\\

This document will develop an idea aimed at increasing attention, participation, dedication and effort on the part of students in their obligations as a student.\\

Playful activities cause us to focus all our attention on constantly overcoming challenges, even provoking them through stories; scenarios and experiences, the total immersion of the person who makes them. The idea is to extract all this mechanism and implement it in education, making the challenges to be overcome educational challenges. \\

In this work an immersive scenario is presented in which scientists have managed to stop the growth gene. Since no one dies due to natural causes, a regressive clock with 10 years of life is implanted in each person. From that moment work is charged in time, food costs time, the whole economy is based on time. If a clock reaches zero it will cause an instantaneous infarction.\\

The scenario described above is a brief synopsis of the film \textbf{InTime} by the New Zealand director \textbf{Andrew Niccol} and it is in this plot where the \textbf{students of Fundamentals of Physical Education will be immersed} 2016/2017  \& 2017/2018 through a \textbf{Web Application} that will simulate this scenario.\\

In this case each student (citizen of SinTime from now on) will start their experience with 15 days of life in independent districts, every second that passes is a second less than they have and they will have to perform various challenges to get more time.

\chapter*{}
\thispagestyle{empty}

\noindent\rule[-1ex]{\textwidth}{2pt}\\[4.5ex]

Yo, \textbf{Álvaro Fernández-Alonso Araluce}, alumno de la titulación TITULACIÓN de la \textbf{Escuela Técnica Superior
de Ingenierías Informática y de Telecomunicación de la Universidad de Granada}, con DNI 75167394J, autorizo la
ubicación de la siguiente copia de mi Trabajo Fin de Grado en la biblioteca del centro para que pueda ser
consultada por las personas que lo deseen.

\vspace{6cm}

\noindent Fdo: Álvaro Fernández-Alonso Araluce (alumno)

\vspace{2cm}

\begin{flushright}
Granada a Mayo de mes de 2017.
\end{flushright}


\chapter*{}
\thispagestyle{empty}

\noindent\rule[-1ex]{\textwidth}{2pt}\\[4.5ex]

D. \textbf{Juan Manuel Fernández Luna}, Profesor del Área de XXXX del Departamento YYYY de la Universidad de Granada.

\vspace{0.5cm}

D. \textbf{Isaac José Pérez López}, Profesor del Área de XXXX del Departamento YYYY de la Universidad de Granada.


\vspace{0.5cm}

\textbf{Informan:}

\vspace{0.5cm}

Que el presente trabajo, titulado \textit{\textbf{SinTime, Aplicación Web de Docencia Gamificada}},
ha sido realizado bajo su supervisión por \textbf{Álvaro Fernández-Alonso Araluce}, y autorizamos la defensa de dicho trabajo ante el tribunal
que corresponda.

\vspace{0.5cm}

Y para que conste, expiden y firman el presente informe en Granada a X de Septiembre de 2018.

\vspace{1cm}

\textbf{Los directores:}

\vspace{5cm}

\noindent \textbf{Juan Manuel Fernández Luna \ \ \ \ \ Isaac José Pérez López}

\chapter*{Agradecimientos}
\thispagestyle{empty}

       \vspace{1cm}


A \textbf{Soraya} por su apoyo incondicional.\\

A \textbf{Juan Manuel Fernández Luna} por su paciencia y consejos.\\

A \textbf{Isaac José Pérez López} por contar conmigo para desarrollar una idea tan original y a \textbf{sus alumnos de Fundamentos de la Educación Física} por la experiencia vivida.

